\documentclass[a4paper, twoside]{Thesis}
% \documentclass[8pt, a4paper, twoside]{Thesis} % this generated warning about unused option [8pt]
\usepackage[export]{adjustbox}
\usepackage{afterpage}
%\usepackage{packages/algorithm2e}
\usepackage[ruled,linesnumbered,vlined]{algorithm2e}
\usepackage{amsmath}
\usepackage{amssymb}
\usepackage{dcolumn}
\usepackage{dsfont}
\usepackage{etoolbox} % for the bibliography ``badness''
\usepackage{multirow}
\usepackage[section]{placeins}
\usepackage{setspace}
%\usepackage{subcaption}
\usepackage[hang,center]{subfigure}
\usepackage{tocbibind}
\usepackage{tocvsec2} % for \settocdepth
\usepackage{xcolor}

\apptocmd{\sloppy}{\hbadness 10000\relax}{}{}

\definecolor{dark-blue}{rgb}{0.2,0.2,0.6}

\setlength{\parindent}{8pt}
\makeatletter
\DeclareRobustCommand\onedot{\futurelet\@let@token\@onedot}
\def\@onedot{\ifx\@let@token.\else.\null\fi\xspace}

\def\eg{\emph{e.g}\onedot} \def\Eg{\emph{E.g}\onedot}
\def\ie{\emph{i.e}\onedot} \def\Ie{\emph{I.e}\onedot}
\def\cf{\emph{c.f}\onedot} \def\Cf{\emph{C.f}\onedot}
\def\etc{\emph{etc}\onedot}
\def\vs{\emph{vs}\onedot}
\def\wrt{w.r.t\onedot}
\def\dof{d.o.f\onedot}
\def\etal{\emph{et al}\onedot}
\makeatother

\newcommand{\argmax}{\operatornamewithlimits{argmax}}
\newcommand{\argmin}{\operatornamewithlimits{argmin}}
\newcommand{\abs}[1]{\left\lvert#1\right\rvert}
\newcommand{\norm}[1]{\left\lVert#1\right\rVert}
\newcommand{\sign}{\operatorname{sign}}

% Zori's workaround: from packages/algorithm2e.sty
\newcommand{\dontprintsemicolon}{\DontPrintSemicolon}%
\newcommand{\incmargin}[1]{\IncMargin{#1}}%
\newcommand{\decmargin}[1]{\DecMargin{-#1}}%
\newcommand{\Setnlsty}[3]{\SetNlSty{#1}{#2}{#3}}%

% fancy paragraph labelling
\renewcommand{\theparagraph}{\S\arabic{paragraph}}
\setcounter{secnumdepth}{5} % give numbering up to depth 5: 1) section 2) subsection 3) subsubsection 4) paragraph 5)subparagraph

\begin{document}
% *************** Front matter ***************
\frontmatter

\title  {Structured Forests: {F}rom Edges to Contours}

\authors  {\texorpdfstring
            {{Zornitsa Kostadinova}}
            {Zornitsa Kostadinova}
            }
\addresses  {\groupname\\\deptname\\\univname}  % Do not change this here, instead these must be set in the "Thesis.cls" file, please look through it instead
%\usdate
\date       {Saarbr\"ucken, \today }
\subject    {}
\keywords   {}

\maketitle

%\newpage
%\mbox{}
%\thispagestyle{empty}
%\newpage

\setstretch{1}

\thispagestyle{empty}

\section*{Eidesstattliche Erkl\"{a}rung}
Ich erkl\"{a}re hiermit an Eides Statt, dass ich die vorliegende Arbeit selbstst\"{a}ndig verfasst und keine
anderen als die angegebenen Quellen und Hilfsmittel verwendet habe.

\vspace{0.60cm}
\section*{Statement in Lieu of an Oath}
I hereby confirm that I have written this thesis on my own and that I have not used any other media or
materials than the ones referred to in this thesis.
\vspace{1.5cm}

\section*{Einverst\"{a}ndniserkl\"{a}rung}
Ich bin damit einverstanden, dass meine (bestandene) Arbeit in beiden Versionen in die Bibliothek der
Informatik aufgenommen und damit ver\"{o}ffentlicht wird.

\vspace{0.60cm}
\section*{Declaration of Consent}
I agree to make both versions of my thesis (with a passing grade) accessible to the public by having
them added to the library of the Computer Science Department.
\vspace{3cm}

\begin{flushright}
\noindent Saarbr\"{u}cken, \today
\hfill
Zornitsa Kostadinova
\end{flushright}

\clearpage  % Declaration ended, now start a new page
%% ----------------------------------------------------------------
\newpage
\mbox{}
\thispagestyle{empty}
\newpage
%% ----------------------------------------------------------------

% \newpage
% \mbox{}
% \thispagestyle{empty}
% \newpage

% The Abstract Page
\addtotoc{Abstract}  % Add the "Abstract" page entry to the Contents
\abstract{
\addtocontents{toc}{\vspace{1em}}  % Add a gap in the Contents, for aesthetics
\vspace{-0.8cm}
Recently, good edge detection results were achieved using structured forests. 
The concept of an image edge is a critical component of all image segmentation algorithms. However, relatively few segmentation techniques base themselves off the output of an edge detector.

In this project we use the strong edge detections obtained with structured forests as a basis for hierarchical image segmentation. % through structured learning
We explore ways to leverage the information learnt by the structured forest in order to compute a hierarchical segmentation, in effect closing the extracted edges to contours.

% Edge
% detection is a critical component of many
% image segmentation algorithms. In this project,
% we explore ways to leverage the information in
% the Structured Forest in order to do image
% segmentation.
}
\clearpage

% %Video segmentation has become one of the core areas in computer vision with a wide range of applications.
% Despite the recent progress, video segmentation research is currently limited by the lack of study of low-level features.
% The computational complexity of video data and variations of color, texture and motion over time introduce additional challenges to the task of video segmentation. 
% % The computational complexity of the video data and inherent difficulties such as 
% % variations of color, texture and motion over time and hence temporal consistency of segments introduce additional challenges to the problem. 
% 
% %Many segmentation algorithms have come to rely increasingly on spectral clustering in past years. 
% Spectral %relaxation 
% methods build the basis of many state-of-the-art image and video segmentation
% techniques. However, in contrast to image segmentation extension of spectral clustering to video %segmentation 
% is far less researched. % due to inherent difficulties such as temporal consistency of segments and requirement of computational resources. !!!!!
% Little attention has been paid to the effects of the 2-norm relaxation 
% with different cut criteria applied to video segmentation and the tight 1-norm relaxation, which in practice yields %much 
% better partitions than %the standard 
% spectral clustering, has not been yet adopted to video processing. 
% 
% In this work we contribute with an experimental analysis %of the quality 
% of low-level features used for video segmentation. 
% Our results show that the features, integrated by local grouping cues, provide good performance with low error rate and are sufficient for obtaining high-quality segmentations.
% 
% In an effort to understand the benefits and drawbacks of recently developed spectral methods applied to video, we provide an extensive empirical comparison of the 1-norm and 2-norm relaxation techniques
% with different graph cut objective functions, examining their impact on segmentation performance. We show that applied to video the standard spectral method outperforms 1-spectral clustering and 
% the optimal in the sense of balanced graph cuts solution is not always reached by %both 
% relaxation techniques.
% 
% We propose to employ constrained spectral clustering to video segmentation, where must-link constraints are integrated into spectral %clustering 
% framework via sparsification preserving all balanced %graph 
% cuts in the reduced graph. We present a novel methodology for discriminative learning of must-links from low-level features.
% The proposed method allows to reduce runtime and memory consumptions and improve the performance. 
% The experimental results on the Berkeley motion segmentation dataset demonstrate the relevance and accuracy of our method as compared to other existing video segmentation algorithms.
% }
% \clearpage

%% ----------------------------------------------------------------
\newpage
\mbox{}
\thispagestyle{empty}
\newpage

\setstretch{1.3}
\begin{spacing}{0.1}
\pagestyle{fancy}
\tableofcontents
\newpage

% \listoffigures
% \newpage
% \listoftables
% \newpage
% \listofalgorithmes
\end{spacing}

\clearpage

\singlespacing
\setlength{\parskip}{6pt}
% *************** Main matter ***************
\mainmatter

\chapter{Introduction}
\label{Chapter1}
\section{Motivation}

\section{Related Work}
\subsubsection*{Edge Detection}
\subsubsection*{Segmentation}
\section{Outline}
% The rest of this work is structured as follows:
% \begin{itemize}
% \item Chapter~\ref{SpectRelax} starts the thesis with a brief introduction to balanced graph cuts and spectral relaxation techniques.
% Section~\ref{sec:ch2_clgrpart} shows that clustering can be seen as a graph partitioning problem. The minimum cut approach often yields useless results where clusters are highly unbalanced, hence
% the balanced graph cut criteria are described in Section~\ref{sec:ch2_balgrcut}.  To solve the NP-hard balanced graph cut problem
% the relaxation methods are applied. Section~\ref{sec:ch2_spectclus} presents the standard spectral clustering approach, which is known to be loose.
% The tight relaxation, called 1-spectral clustering, is described in Section~\ref{sec:ch2_1spectclus}. 
% Section~\ref{ch2:disc} concludes the chapter and discusses the relevance of proposed methods to video segmentation.
% \item Chapter~\ref{chapter3} gives an overview of the video segmentation framework and provides the analysis of low-level features.
% Section~\ref{sec:ch3_framework} introduces the proposed video segmentation model, which employs a two-step approach:
% a graph is constructed on pre-computed superpixels and then a spectral clustering technique is applied. 
% %In graph-based algorithms, in order to produce high-quality segmentation results powerful superpixel similarity measures must be defined.
% Section~\ref{sec:ch3_affinities} gives a description of the graph affinities used in this work.
% To evaluate video segmentation performance and analyze the features of the proposed model we chose the Berkeley motion segmentation dataset, which is presented in Section~\ref{sec:ch3_dataset}.  
% The examination of the quality of the low-level video features is reported in Section~\ref{sec:ch3_aff} and the results are discussed in Section~\ref{ch3:disc}.
% \item Chapter~\ref{Chapter4} provides an experimental comparison of spectral relaxations and analyzes the effects of different balanced graph cuts applied to video segmentation. 
% We start with a brief recap of the main theoretical aspects of 1-norm and 2-norm relaxations in Section~\ref{ch4:recap}.
% Section~\ref{ch4:bench} presents the evaluation benchmark for video segmentation.
% Section~\ref{sec:ch4_1sc_vs_sc} shows the comparison of the performance of spectral clustering and 1-spectral clustering with different balanced graph cut objectives in the task of video segmentation. 
% In order to explore further the balanced cut criteria and the quality of the solutions obtained from the relaxation techniques, we tried to find a better partition by a trivial greedy search optimizing different balanced graph cut functions and see if the ground truth corresponds
% with the minimum cut criterion. The results of the experiments are reported in Section~\ref{sec:ch4_GTexp}.
% Section~\ref{ch4:disc} gives the discussion of the obtained results.
% \item In Chapter~\ref{Chapter5} a methodology for discriminative learning of must-link constraints and their incorporation in the video segmentation framework are proposed.
% Section~\ref{sec:ch5_cosc} shows a way of integrating prior information in the form of must-link constraints into spectral clustering while preserving all the
% balanced graph cuts.
% Section~\ref{sec:llf} presents evaluation of the low-level features as must-link constraints and the connections in the graph based on the ground truth.
% In Section~\ref{sec:ch4_ML} we propose to learn must-links with Random Forest from the affinities.
% We show that even a naive learning approach on the restricted feature space improves video segmentation performance for both relaxations: spectral clustering and 1 spectral clustering. 
% The proposed model is compared to state-of-the-art methods.
% Section~\ref{ch5:disc} discusses the achieved results.
% \item Chapter~\ref{Chapter6} concludes the thesis summarizing all the results of our work and proposing directions for possible improvements.
% \end{itemize}

\clearpage

\chapter{Spectral Relaxations of Balanced Graph Cuts}%{Spectral Methods for Video Segmentation}
\label{SpectRelax}
Spectral clustering techniques have become one of the major clustering methods over the past few years. 
The reasons are their well-understood theoretical foundation and generality. Spectral clustering can be applied to any kind of data where similarity measure is available
to build a neighbourhood graph and can be solved efficiently by standard linear algebra software. Results obtained by spectral clustering often outperform the traditional clustering approaches.
Spectral methods have proved to be successful in many video segmentation algorithms~\cite{Brox10,GalassoCS12,Galasso14}. They convince by the ability to include long-range affinities and
the globalization effect~\cite{Fowlkes04}.

In this chapter we focus on the motivation of spectral clustering based on balanced graph cut criteria. We start by presenting different balanced graph cut objectives and continue with overview of 
relaxation techniques of the NP-hard balanced cut problem. We introduce a standard 2-norm relaxation approach, which known to be loose and may lead to a solution far from the optimal one of the original problem~\cite{guattery1998},
and a tight 1-norm relaxation technique, called 1-spectral clustering~\cite{Buhler09,Hein10,HeinS11}, which in practice usually outperforms the standard loose relaxation and yields much better graph cuts.
%Further the generalization of 1-spectral clustering which integrates must-link constraints, the method called COSC~\cite{RangapuramH12}, is presented. 
%
% 
% The method, called COSC, is based on a tight relaxation of the constrained normalized cut into
% an unconstrained continuous optimization problem and is the first one, where all given constraints are fulfilled.  
% 
% Solving balanced graph cut problems is well-known
% to be NP-hard~\cite{Wagner93}. 
% 
% Spectral clustering as a graph-based approach to clustering is originally derived as a relaxation
% of the NP-hard balanced graph cut problem. 
% 
% We introduce a standard 2-norm relaxation approach, which known to be loose and may lead to a solution far from the optimal one of the original problem~\cite{guattery1998},
% and a tight 1-norm relaxation technique, called 1-spectral clustering~\cite{Buhler09,Hein10,HeinS11}. In practice it usually outperforms the standard loose relaxation and yields much better graph cuts.
% 
% leads to an eigenproblem for the graph Laplacian, where the second eigenvectors of the unnormalized and normalized graph Laplacians correspond to relaxations
% of the ratio cut~\cite{HagenK91} and normalized cut~\cite{Shi00}. However, it is often quite loose and may lead to a solution far from the optimal one of the original problem~\cite{guattery1998}.
% 
% Recently, it has been shown that all balanced graph cut problems have a tight relaxation into continuous optimization problem using the nonlinear 1-graph Laplacian~\cite{Buhler09,Hein10,HeinS11}. This tight relaxation is called 
% 1-spectral clustering. Although this technique provides no guarantee of convergence to the globally optimal solution, in practice it usually outperforms the standard loose relaxation and yields much better graph cuts.
% 
% In further work the generalization of 1-spectral clustering technique which integrates must-link constraints has been shown in~\cite{RangapuramH12}. 
% The method, called COSC, is based on a tight relaxation of the constrained normalized cut into
% an unconstrained continuous optimization problem and is the first one, where all given constraints are fulfilled.  
\section{Clustering as Graph Partitioning}
\label{sec:ch2_clgrpart}
%In this thesis to obtain the final video segmentation we consider clustering as a graph partitioning problem.
Clustering can be seen as a graph partitioning problem.
Suppose we are given a set of data points $\{x_i\}_{i=1}^n$ and some notion of similarity measure $W\colon X\times X\rightarrow \mathbb{R}$. 
The intuition behind the clustering is to group points into different sets according to their similarities. 

The data can be transformed into a weighted, undirected graph $G =(V,W)$, where the vertices $V$ of the graph represent the data points and the positive edge weights $W$ encode the similarity between pairs of points.
The matrix $W = \{w_{ij}\}_{i,j=1}^n$ is called the affinity matrix. When constructing similarity graphs the goal is to model local neighbourhood relationships between the points, to build the global structure inherent in the data 
from the local structure.

The problem of clustering can be reformulated using the similarity graph: we want to find the partition such that the edges inside the group of points have very high weights (high within-cluster similarity) and the edges between different groups of points 
have very low weights (low between-cluster similarity). A clustering of points is then equivalent to a partition of $V$ into subsets $C_1,\dots,C_k$, where $k$ is the desired number of clusters. 

Let us introduce some notions. The degree of vertex $v_i\in V$ is defined by the degree function $d\colon V\rightarrow \mathbb{R}$, $d_i = \sum_{j=1}^n w_{ij}$. The degree matrix $D$ is defined as $D_{ij} = d_i\delta_{ij}$. 
Given a subset of vertices $A\subset V$, its complement is denoted as $\overline{A} = V\backslash A$. For two not necessarily disjoint sets $A,B\subset V$ we define
\begin{equation*}
W(A,B) = \sum\limits_{i\in A, j\in B} w_{ij}.
\end{equation*}
And the cut value of $A\subset V$ and $\overline{A}$ is defined as 
%\begin{equation*}
$\mathrm{cut}(A,\overline{A}) = W(A,\overline{A})$. %\sum\limits_{i\in A, j\in \overline{A}} w_{ij}.
%\end{equation*}

Using these definitions several criteria for graph partitioning can be defined.
One way to obtain the partition of the graph is to solve the min-cut problem. For a given number of clusters $k$, the min-cut approach consists in choosing the partition  $C_1,\dots,C_k$ which minimizes
\begin{equation}
\label{eq:mincut}
\mathrm{MinCut}(C_1,\dots,C_k) = \frac{1}{2} \sum\limits_{i=1}^k \mathrm{cut}(C_i,\overline{C_i}).
\end{equation}
The min-cut problem is well researched and can be solved efficiently in polynomial time (notably the Edmonds-Karp algorithm). However, in practice solving~\eqref{eq:mincut} often leads to trivial partitions. 
The solution of min-cut may just disconnect one individual point or a few points from the set. This is not what we want to see in the clustering solution, as clusters 
should be reasonably large group of points. The way to deal with this problem is to explicitly request that the clusters should be of equal size. Therefore one has to introduce some balancing term which biases the cut criterion 
towards balanced partitions.  
In the unsupervised clustering when the focus is on pairwise terms the balancing property of graph cuts is favourable, whereas the graph partitioning techniques based on the min-cut formulation are suited for problems
with strong unary terms.
\section{Balanced Graph Cut Criteria}
\label{sec:ch2_balgrcut}
In our work we aim to investigate the influence of different balanced graph cut objectives on the final video segmentation result.
Several balanced graph cut criteria have been proposed in the literature, all of them request the size of the clusters to be ``reasonably large``.
%One way to obtain the desirable solution is to balance the size of clusters. 

There are two different approaches how to measure the size of a set in a graph. We can favour solutions which give clusters with equal cardinality $\lvert A \rvert$ 
(cardinality of set $A$) or with equal volume $\mathrm{vol}(A) = \sum_{i \in A} d_i$ (see fig.~\ref{fig:bal_cr}).
\begin{figure}[h!]
 \centering
 \subfigure[Balancing of cardinality]{%
 \includegraphics[width=0.4\textwidth]{images/png/card.png}
\label{fig:subfigure1}}
\quad
\qquad
 \subfigure[Balancing of volume]{%
 \includegraphics[width=0.4\textwidth]{images/png/vol.png}
\label{fig:subfigure2}}
 \caption[Two different ways of balancing the size of clusters]{
  {\bf Two different ways of balancing the size of clusters} (courtesy of~\cite{HeinBuh09}).}
\label{fig:bal_cr}
\end{figure}
Using these two balancing terms one can define several balanced graph cut criteria which can be used for graph partitioning. Commonly used balanced graph cut criteria are the ratio cut $\mathrm{RCut}(C,\overline{C})$~\cite{HagenK91} and 
the normalized cut $\mathrm{NCut}(C,\overline{C})$~\cite{Shi00},
which are defined as
\begin{equation}
\mathrm{RCut}(C,\overline{C}) =  \frac{\mathrm{cut}(C,\overline{C})}{\lvert C \rvert} + \frac{\mathrm{cut}(C,\overline{C})}{\lvert \overline{C} \rvert},
\end{equation}
\begin{equation}
\mathrm{NCut}(C,\overline{C}) =  \frac{\mathrm{cut}(C,\overline{C})}{\mathrm{vol}(C)} + \frac{\mathrm{cut}(C,\overline{C})}{\mathrm{vol}(\overline{C})}.
\end{equation}
Related criteria with slightly different balancing behaviour are the corresponding ratio Cheeger cut $\mathrm{RCC}(C,\overline{C})$~\cite{c70} and normalized Cheeger cut $\mathrm{NCC}(C,\overline{C})$ defined as
\begin{equation}
  \displaystyle \mathrm{RCC}(C,\overline{C}) =  \frac{\mathrm{cut}(C,\overline{C})}{\min{\{\lvert C \rvert,\lvert \overline{C} \rvert\}} }, 
\end{equation}
\begin{equation}
  \displaystyle \mathrm{NCC}(C,\overline{C}) =  \frac{\mathrm{cut}(C,\overline{C})}{\min{\{\mathrm{vol}(C),\mathrm{vol}(\overline{C})\}} }. 
\end{equation}
One has the following simple relation between the normalized cut $\mathrm{NCut}(C,\overline{C})$ and the normalized Cheeger cut $\mathrm{NCC}(C,\overline{C})$:
\begin{equation*}
 \mathrm{NCC}(C,\overline{C})\leq \mathrm{NCut}(C,\overline{C})\leq \mathrm{2NCC}(C,\overline{C}).
\end{equation*}
The same result holds for the ratio cut $\mathrm{RCut}(C,\overline{C})$ and the ratio Cheeger cut $\mathrm{RCC}(C,\overline{C})$.

Generally, clustering has two different objectives:
\begin{enumerate}
\item We want to find the partition which minimizes the between-cluster similarity. This means to minimize the cut value $\mathrm{cut}(C,\overline{C})$.
\item We want to find the partition which maximizes the within-cluster similarities $W(C,C)$ and $W(\overline{C},\overline{C})$.
\end{enumerate}

Both the ratio and the normalized cut, as well as the Cheeger cuts, directly fulfill the first objective by explicitly incorporating the value of $\mathrm{cut}(C,\overline{C})$ in the objective function. 
However, concerning the second point the algorithms behave differently due to its balancing terms. As $W(C,C)= W(C,V)-W(\overline{C},\overline{C}) = \mathrm{vol}(C)-\mathrm{cut}(C,\overline{C})$, 
the within-cluster similarity is maximized if $\mathrm{vol}(C)$ is large and the cut value is small. This is exactly
achieved by the normalized cut $\mathrm{NCut}$. In order to fulfill the second objective, another balanced graph cut criterion can be considered, namely the $\mathrm{MinMaxCut}$ criterion~\cite{DingHZGS01}:
\begin{equation}
 \mathrm{MinMaxCut}(C,\overline{C}) =  \frac{\mathrm{cut}(C,\overline{C})}{W(C,C)} + \frac{\mathrm{cut}(C,\overline{C})}{W(\overline{C},\overline{C})}.
\end{equation}
In practice, $\mathrm{NCut}$ and $\mathrm{MinMaxCut}$ are often minimized by similar partitions, as a good solution will have a small value of $\mathrm{cut}(C,\overline{C})$ and hence
their denominators are not so different. Moreover, relaxing $\mathrm{MinMaxCut}$ leads exactly to the same optimization problem as relaxing
$\mathrm{NCut}$~\cite{Luxb07}.
As for the ratio cut, here the objective is to maximize cardinality instead of volume. But the within-cluster similarity depends on the edges and not on the number of vertices in the cluster. Therefore by minimizing $\mathrm{RCut}$ the
second objective is not achieved.

For multipartition of $V$ into $k$ clusters $C_1,\dots,C_k$ the ratio, normalized cuts and the $\mathrm{MinMaxCut}$ can be generalized~\cite{Luxb07} as
\begin{equation}
\mathrm{RCut}(C_1,\dots,C_k) =  \sum \limits_{i=1}^k \frac{\mathrm{cut}(C_i,\overline{C_i})}{\lvert C_i \rvert} ,
\end{equation}
\begin{equation}
\mathrm{NCut}(C_1,\dots,C_k) =  \sum \limits_{i=1}^k \frac{\mathrm{cut}(C_i,\overline{C_i})}{\mathrm{vol}(C_i)}.
\end{equation}
\begin{equation}
 \mathrm{MinMaxCut}(C_1,\dots,C_k)) = \sum \limits_{i=1}^k \frac{\mathrm{cut}(C_i,\overline{C_i})}{W(C_i,C_i)}.
\end{equation}
The commonly recognized multipartition version of the Cheeger cuts does not exist.

% Both objective functions take a small value if clusters $C_i$ are not too small: the minima of $\sum_{i=1}^k 1\backslash{\lvert C_i \rvert}$ and $\sum_{i=1}^k 1\backslash{\mathrm{vol}(C_i)}$ are achieved if all $\lvert C_i \rvert$ and
% $\mathrm{vol}(C_i)$ coincide correspondingly.
One should also mention that balancing term only has an impact 
if the size of cardinality or volume is small, usually the cut value itself dominates what is selected for the partition. It simply helps to avoid the extreme
cases where the clusters are highly unbalanced.

We are interested in the optimal $\mathrm{RCut}$/$\mathrm{NCut}$ and search for the minimum among all possible partitions,
\begin{equation}
\label{eq:np}
\min\limits_{C_1,\dots,C_k} \mathrm{NCut}(C_1,\dots,C_k).
\end{equation}
However, introducing balancing terms makes the graph partitioning problem~\eqref{eq:np} NP-hard~\cite{Wagner93}. In practice we use spectral relaxations of this problem. 
We can write the normalized cut as
\begin{equation*}
\mathrm{NCut}(C_1,\dots,C_k) = \sum \limits_{i=1}^k \frac{\sum _{r,s=1}^n \mathds{1}_{r \in C_i} w_{ij} (1 - \mathds{1}_{s \in C_i)}}{\sum_{r=1}^n \mathds{1}_{r \in C_i} d_r},
\end{equation*}
so that finding the optimal partition can be seen as a combinatorial optimization problem. We will see below that a slightly different way of writing balanced graph cut criteria suggests a relaxation, where 
the constraints are relaxed. Thus the set of possible solutions becomes larger and the optimal value of the relaxed problem is smaller than the one of the original problem. We go from combinatorial 
to a continuous problem which is potentially easier to solve. Spectral clustering algorithms are based on this kind of relaxation.
%For this relaxation we could even find the globally optimal solution. 
\section{Spectral Clustering}
\label{sec:ch2_spectclus}
In our work we want to study the effects of the 2-norm relaxation of balanced graph cuts which leads to spectral clustering on video segmentation.
This section presents a short overview of the spectral relaxation technique, two variants of spectral clustering algorithmic schemes and argues about the quality of the solution
obtained by the standard spectral method. 

The spectral relaxation of the balanced graph cut problem results in the linear eigenproblem for the graph Laplacian~\cite{HagenK91,Shi00,Luxb07}. Therefore in order to proceed 
with the 2-norm relaxation, we first briefly recall some notions of the graph Laplacian and point out the most important properties.
%In order to proceed with relaxations of balanced graph cut problem, we first briefly recall some notions of the graph Laplacian and point out the most important properties.
\subsubsection*{Graph Laplacian}
In the following we always assume that $G$ is an 
undirected, weighted graph with affinity matrix $W$.
%\subsubsection*{The unnormalized graph Laplacian}
The unnormalized graph Laplacian is defined as 
\begin{equation*}
\Delta_{2}^{(u)} = D-W.
\end{equation*}
There are two matrices which are called normalized graph Laplacians in the literature. Since the first one is closely related to a random work on the graph, it is called normalized random walk graph Laplacian 
and the other one normalized graph Laplacian. Both of them are closely related and are defined as
\begin{equation*}
\begin{aligned}
&\Delta_{2}^{(rw)} = I-D^{-1}W,\\
&\Delta_{2}^{(n)} = I-D^{-1/2}WD^{-1/2}.
\end{aligned}
\end{equation*} 
Note that $\Delta_{2}^{(u)} = D\Delta_{2}^{(rw)}$ and $\Delta_{2}^{(n)} = D^{-1/2}\Delta_{2}^{(u)}D^{-1/2}$.

An overview over the properties of graph Laplacians can be found in~\cite{Luxb07}. The most important properties are summarized in the following proposition~\cite{HeinNotes}.
\begin{proposition}
\begin{enumerate}
\item For every vector $f\in\mathbb{R}^n$ 
\begin{equation*}
\begin{aligned}
\langle f,\Delta_{2}^{(u)}f\rangle &= \langle f,\Delta_{2}^{(rw)}f\rangle= \frac{1}{2}\sum \limits_{i,j=1}^n w_{ij} ( f_i - f_j )^2 ,\\
\langle f,\Delta_{2}^{(n)}f\rangle &= \frac{1}{2}\sum \limits_{i,j=1}^n w_{ij} ( \frac{f_i}{\sqrt{d_i}} - \frac{f_j}{\sqrt{d_j}} )^2,
\end{aligned}
\end{equation*}
where $\langle \cdot,\cdot\rangle$ denotes the inner product in $\mathbb{R}^n$.
\item All graph Laplacians are positive semi-definite and self-adjoint.
\item $\Delta_{2}^{(rw)} = I-D^{-1}W$ and $\Delta_{2}^{(n)} = I-D^{-1/2}WD^{-1/2}$ are similar, $\Delta_{2}^{(rw)} =D^{-1/2}\Delta_{2}^{(n)}D^{1/2}$.
Therefore $\Delta_{2}^{(n)}u_i = \lambda_i u_i \iff v_i = D_{-1/2}u_i,\quad\Delta_{2}^{(rw)} v_i = \lambda_i v_i$.
\item Let $A_i$, $i=1,\dots,K$ be the connected components of the graph. $\mathds{1}_{j \in A_i}$ are eigenvectors of $\Delta_{2}^{(rw)}$ and $\Delta_{2}^{(u)}$ to the eigenvalue 0.
$\sqrt{d_j} \mathds{1}_{j \in A_i}$ are eigenvectors of $\Delta_{2}^{(n)}$ to the eigenvalue 0.
\item The eigenvectors of $\Delta_{2}^{(u)}$ and $\Delta_{2}^{(n)}$ define an orthonormal basis on $\mathbb{R}^V$.
\end{enumerate}
\end{proposition}
% 
% The associated regularization functionals with the graph Laplacian are:

The following properties motivate the use of eigenvectors of the graph Laplacian for finding the best partition of the graph. Later we will see that the minimum and minimizer of the relaxed balanced graph cut problem are exactly 
the second eigenvalue and second eigenvector of the graph Laplacian, which leads to the formulation of spectral clustering - the relaxation of the balanced graph cut criteria problem.   
\subsubsection*{Relaxation of Balanced Graph Cuts for Laplacian}%{Approximating $RCut$ and $NCut$}
For simplicity we start with the case of $\mathrm{RCut}$ and $k=2$. Our goal is to solve the following optimization problem
\begin{equation}
\label{eq:minrcut}
\min\limits_{C \in V} \mathrm{RCut}(C,\overline{C}).
\end{equation} 
\begin{lemma}
\label{lem:rcut}
Given a partition of $V$ into $C$ and $\overline{C}$ define,
\begin{equation}
\label{eq:frcut}
 f_i = \begin{cases} 
                      \sqrt{\frac{\lvert \overline{C} \rvert}{\lvert C \rvert}}   & \quad i \in C, \\
                      - \sqrt{\frac{\lvert C \rvert}{\lvert \overline{C} \rvert}} & \quad i \in \overline{C}.
                     \end{cases}
\end{equation}
Then 
\begin{equation*}
\mathrm{RCut}(C,\overline{C}) = \{\langle f, \Delta_{2}^{(u)} f \rangle\ |\langle f, \mathds{1} \rangle=0, \lVert f \rVert = \sqrt{n}, \text{ $f$  has form as in Eq.~\eqref{eq:frcut}} \}.
\end{equation*}
\end{lemma}
The proof can be found in~\cite{Luxb07}. 
Now the $\mathrm{RCut}$ objective function is conveniently rewritten using the unnormalized graph Laplacian. With the formulation of the lemma~\eqref{lem:rcut} we deal with a discrete optimization problem~\eqref{eq:minrcut} and 
it is still NP-hard.
However, now there exists a direct relaxation. Instead of allowing for $f$ to take only two possible values, we relax the problem by discarding the discreteness condition and allowing $f$ to take arbitrary values in $\mathbb{R}^V$. 
This leads to the relaxed optimization problem
\begin{equation}
\label{eq:rel}
\min \limits_{f \in \mathbb{R}^V} \{\langle f, \Delta_{2}^{(u)} f \rangle\ |\langle f, \mathds{1} \rangle=0, \lVert f \rVert = \sqrt{n}\}.
\end{equation}
By the Rayleigh-Ritz principle it can be seen that the solution of~\eqref{eq:rel} is just the second smallest eigenvalue $\lambda_2$ of $\Delta_{2}^{(u)}$. However, we are more interested
in the minimizer, which is just the second eigenvector $u_2$ of $\Delta_{2}^{(u)}$. The remaining step is to transform the eigenvector $u_2$ into a partition of the graph. There are two ways~\cite{HeinNotes}:
\begin{itemize}
\item thresholding: one defines for $t\geq 0$ the set $C_t = \{j \in V | u_2(j) > t\}$ and finds the threshold $t$ by optimizing the ratio cut criterion
\begin{equation*}
t^* = \min\limits_{t\geq 0} \mathrm{RCut}(C,\overline{C}).
\end{equation*}   
And return $C_{t^*}$ and $\overline{C}_{t^*}$ as clusters. This transformation of the second eigenvector into a graph partition provides upper and lower bounds in terms of the optimal cut~\cite{Buhler09}. 
\item k-means: one assumes that the eigenvector is basically constant on the cluster $C$ and $\overline{C}$. Under this condition k-means will give a partition similar to the one obtained by thresholding. However, for two clusters
thresholding should be preferred. k-means should only be used for more than two clusters.
\end{itemize}
Note, as long as the clusters are well-separated, the eigenvectors represent the structure well, so thresholding and k-means should provide almost the same clustering.
The same relaxation technique can be applied to the normalized cut criterion.
\begin{lemma}
\label{lem:ncut}
Given a partition of $V$ into $C$ and $\overline{C}$ define,
\begin{equation}
\label{eq:fncut}
 f_i = \begin{cases} 
                      \sqrt{\frac{\mathrm{vol}(\overline{C})}{\mathrm{vol}(C)}}   & \quad i \in C, \\
                      - \sqrt{\frac{\mathrm{vol}(C)}{\mathrm{vol}(\overline{C})}} & \quad i \in \overline{C}.
                     \end{cases}
\end{equation}
Then 
\begin{equation*}
\mathrm{\mathrm{NCut}}(C,\overline{C}) = \{\langle f, \Delta_{2}^{(u)} f \rangle\ |\langle Df, \mathds{1} \rangle=0, \langle f, Df \rangle = \mathrm{vol}(V) =\sqrt{n}, \text{ $f$ has form as in Eq.~\eqref{eq:fncut}} \}.
\end{equation*}
\end{lemma}
The proof can be found in~\cite{Luxb07}. 
Again we relaxing the problem by allowing $f$ to take arbitrary real values:
\begin{equation}
\min \limits_{f \in \mathbb{R}^V} \{\langle f, \Delta_{2}^{(u)} f \rangle\ | \langle Df, \mathds{1} \rangle=0, \langle f, Df \rangle = \mathrm{vol}(V)\}.
\end{equation}
Now by substituting $g = D^{1/2}f$ we obtain
\begin{equation}
\label{eq:minncut}
\min \limits_{g \in \mathbb{R}^V} \{\langle g, \Delta_{2}^{(n)} g \rangle\ |\langle g, D^{1/2}\mathds{1} \rangle=0, \lVert g \rVert^2 = \mathrm{vol}(V)\}.
\end{equation}
Observe that $\Delta_{2}^{(n)} = D^{-1/2} \Delta_{2}^{(u)} D^{1/2}$, $D^{1/2}\mathds{1}$ is the first eigenvector of $\Delta_{2}^{(n)}$ and $\mathrm{vol}(V)$ is a constant. 
Hence,~\eqref{eq:minncut} is in the form of the Rayleigh-Ritz principle,
and its solution $g$ is given by the second eigenvector of $\Delta_{2}^{(n)}$. By re-substituting $f = D^{-1/2}g$ we see that $f$ is the second eigenvector of the generalized eigenproblem $\Delta_{2}^{(u)}u= \lambda Du$.  
Until now, we have shown relaxations of different balanced graph cut criteria for only two clusters. This principle is also used to motivate the use of higher-order eigenvectors of the graph 
Laplacian for finding graph partitions.
The relaxation of the $\mathrm{RCut}$ and $\mathrm{NCut}$ minimization problem in the case of general number of clusters $k$ can be done in the similar way.
\begin{lemma}
\label{lem:multircut}
Given a partition of $V$ into $C_1,\dots,C_k$ define,
\begin{equation}
\label{eq:fmultircut}
\text{for $i = 1,\dots,k$ and $j=1,\dots,n$, }
 h_i(j) = \begin{cases} 
                      \sqrt{1}{\sqrt{\rvert C_i\lvert}}   & \quad j \in C_i, \\
                      0 & \quad j \in \overline{C_i}.
                     \end{cases}
\end{equation}
Then the general ratio cut can be written as
\begin{equation*}
\mathrm{RCut}(C_1,\dots,C_k) = \{Tr(H\Delta_{2}^{(u)}H^T) |HH^T = \mathds{1}_k\}.
\end{equation*}
\end{lemma}
Similar to above we relax the problem by allowing the entries of matrix $H$ to take arbitrary real values. Then the relaxed problem becomes:
\begin{equation}
\min\limits_{H \in \mathbb{R}^{n\times k}} \{Tr(H\Delta_{2}^{(u)}H^T) |HH^T = \mathds{1}_k\}.
\end{equation}
This is the standard form of a trace minimization problem, and again some version of the Rayleigh-Ritz principle tells us that the solution is given by choosing $H$ as the matrix which contains the first $k$ eigenvectors of 
$\Delta_{2}^{(u)}$ as columns. This leads to the general unnormalized spectral clustering algorithm.

For the case of finding more than two clusters for normalized cut:
\begin{lemma}
\label{lem:multincut}
Given a partition of $V$ into $C_1,\dots,C_k$ define,
\begin{equation}
\label{eq:fmultincut}
\text{for $i = 1,\dots,k$ and $j=1,\dots,n$, }
 h_i(j) = \begin{cases} 
                      \sqrt{1}{\sqrt{\mathrm{vol}(C_i)}}   & \quad j \in C_i, \\
                      0 & \quad j \in \overline{C_i}.
                     \end{cases}
\end{equation}
Then the general normalized cut can be written as
\begin{equation*}
\mathrm{NCut}(C_1,\dots,C_k) = \{Tr(H\Delta_{2}^{(u)}H^T) |HDH^T = \mathds{1}_k\}.
\end{equation*}
\end{lemma}
Relaxing the discreteness condition and substituting $T = D^{1/2}H$ we obtain the relaxed problem
\begin{equation}
\label{eq:mint}
\min\limits_{T \in \mathbb{R}^{n\times k}} \{Tr(T\Delta_{2}^{(n)}T^T) |TT^T = \mathds{1}_k\}.
\end{equation}
Again~\eqref{eq:mint} is solved by the matrix $T$ which contains the first $k$ eigenvectors of $\Delta_{2}^{(n)}$ as columns. Re-substituting $H = D^{-1/2}T$ the solution of $H$ consists of the first $k$ generalized eigenvectors
of $\Delta_{2}^{(u)}u= \lambda Du$. This yields the normalized spectral clustering.
The proof for the ratio and normalized cuts can be found in~\cite{Luxb07}. 
Note, that we still need to transform the set of eigenvectors into a partitioning of $V$. This is where k-means is usually applied.
\subsubsection*{Spectral Clustering Algorithms}
There are two variants of spectral clustering~\cite{HeinNotes}. The main difference is how the eigenvectors of the graph Laplacian are used in order to construct the graph partitioning.

The first variant of spectral clustering uses the eigenvectors to construct a new representation of the data. In this new representation one applies the standard k-means clustering. The data is assumed to be quite well clustered 
so that k-means yields reasonable results. The advantage of this approach is that it aggregates all information about the k eigenvectors. 
The disadvantage is that k-means prefers blob-like clusters which is not always the case for the embedded clusters.

The general scheme of spectral clustering is given below in Algorithm~\ref{alg:SCv1}.
\incmargin{1em}
\begin{algorithm}[hbtp]
\caption{Spectral Clustering - Variant 1}
\label{alg:SCv1}
\dontprintsemicolon
\BlankLine
\Indm  
\KwIn{affinity matrix $W$, number $k$ of clusters to construct, choice of the graph Laplacian (normalized, unnormalized).}
\Setnlsty{textbf}{}{:}
\Indp
\BlankLine
compute the graph Laplacian;\\
compute the first $k$ eigenvectors $\{ u_i \}_{i=1}^k$ (each eigenvector is normalized, $\lVert u_i\rVert=1$, $i=1,\dots,k$);\\
construct the Laplacian eigenmap by embedding $\phi: V\rightarrow \mathbb{R}^k$, $i\rightarrow z_i =(u_1(i),\dots,u_k(i))$;\\
cluster the points $\{z_i\}_{i=1}^n$ by k-means into clusters $C_1,\dots,C_k$.
\BlankLine
\Indm  
\KwOut{Clusters $C_1,\dots,C_k$.} 
\end{algorithm}
\decmargin{1em}

The alternative method is to use a sequential splitting procedure with thresholding to get more than two clusters. The outline of this method is listed in Algorithm~\ref{alg:SCv2}.
\incmargin{1em}
%\linesnumberedhidden
\begin{algorithm} [hbtp]
\caption{Spectral Clustering - Variant 2}
\label{alg:SCv2}
\dontprintsemicolon
\BlankLine
\Indm  
\KwIn{affinity matrix $W$, number $k$ of clusters to construct, choice of the graph Laplacian (normalized, unnormalized).}
\Setnlsty{textbf}{}{:}
\Indp
\BlankLine
%\showln
initialization: cluster $C_1 = V$, number of clusters $s=1$;\\
%\showln
\Repeat{
%\showln
number of clusters $s=k$}
{
build on each element of the current partition $C_i$ the graph Laplacian;\;
compute the second eigenvector on each partition;\;
compute the optimal threshold for dividing each cluster $C_i$;\;
choose to split the cluster $C_i$ so that the total multipartition cut criterion is minimized;\;
$s\Longleftarrow s+1$.
}
\BlankLine
\Indm  
\KwOut{Clusters $C_1,\dots,C_k$.} 
\end{algorithm}
\decmargin{1em}

There is still an open question how good is the partition obtained by the spectral relaxation. The answer to this is the isoperimetric inequality~\cite{Chung:1997}, which provides upper and lower bounds on the ratio and normalized
Cheeger cuts in terms of the second eigenvalue of the graph Laplacian. The optimal ratio and normalized Cheeger cuts values are defined as
\begin{equation*}
h_{\mathrm{RCC}} = \inf_{C} \mathrm{RCC}(C,\overline{C}), \quad h_{\mathrm{NCC}} = \inf_{C} \mathrm{NCC}(C,\overline{C}).
\end{equation*}
The standard isoperimetric inequality is given as
\begin{equation}
 \frac{h_{\mathrm{NCC}}^2}{2} \leq \lambda_{2} \leq  2 h_{\mathrm{NCC}},
\end{equation}
where $\lambda_2$ is the second eigenvalue of the normalized graph Laplacian. Note, while the second eigenvector of the graph Laplacian can be efficiently computed, this spectral relaxation is known to be 
loose %far from being tight
and provides non-optimal result.
\section{1-Spectral Clustering}
\label{sec:ch2_1spectclus}
In the recent line of work~\cite{Buhler09,Hein10,HeinS11} it has been shown that a tight relaxation of the balanced graph cuts exists and can be achieved by moving from linear eigenproblem to a nonlinear eigenproblem associated 
to the nonlinear graph p-Laplacian. This tight relaxation is called 1-spectral clustering. 
In this thesis we would like to adopt the 1-norm relaxation to video segmentation and study the relevance of this technique for video processing in comparison with the standard spectral clustering. 

In order to proceed with introduction to 1-spectral clustering, we first briefly recall the graph p-Laplacian and point out the most important properties.
\subsubsection*{Graph p-Laplacian}
The standard graph Laplacian induces the quadratic form, but there exists a nonlinear generalization of the graph Laplacian $\Delta_{p}$, called graph p-Laplacian~\cite{Amghibech03}
\begin{equation*}
\langle f,\Delta_{p} f\rangle = \frac{1}{2} \sum \limits_{i,j=1}^n w_{ij} \rvert f_i - f_j\rvert^p. 
\end{equation*}
Similar to the graph Laplacian we can define the unnormalized and normalized p-Laplacian $\Delta_{p}^{(u)}$ and $\Delta_{p}^{(n)}$. Let $i \in V$, then
\begin{equation*}
\begin{aligned}
(\Delta_{p}^{(u)}f)_i &= \sum \limits_{j\in V} w_{ij}\phi_p ( f_i - f_j ),\\
(\Delta_{p}^{(n)}f)_i &= \frac{1}{d_i}\sum \limits_{j\in V} w_{ij}\phi_p ( f_i - f_j ),
\end{aligned} 
\end{equation*}
where $\phi_p:\mathbb{R}\longrightarrow \mathbb{R}$ is defined for $x \in \mathbb{R}$ as
\begin{equation*}
 \phi_p(x) = \rvert x \lvert^{p-1} \sign(x).
\end{equation*}
Note, that for $p=2$ the standard graph Laplacian is recovered. As shown in~\cite{Buhler09}, one can obtain the eigenvalues of the %unnormalized 
p-Laplacian $\Delta_{p}^{(u)}$ as local minima of the functional
$F_p: \mathbb{R}^V\longrightarrow \mathbb{R}$,
\begin{equation*}
 F_p(f) = \frac{\langle f, \Delta_{p}f\rangle}{\lVert f\rVert_p^p} = \frac{1}{2} \frac{ \sum \limits_{i,j=1}^n w_{ij} \rvert f_i - f_j\rvert^p}{\lVert f\rVert_p^p}. 
\end{equation*}
\begin{theorem}
The functional $F_p$ has a critical point at $v \in \mathbb{R}^V$ if and only if $v$ is a p-eigenvector of $\Delta_{p}$. The corresponding eigenvalue $\lambda_p$ is given as $\lambda_p = F_p(v)$.
\end{theorem}
Similar to the standard case we need at least the second eigenvector to construct the partition. To obtain the second eigenvalue, we consider the functional $F_p^{(2)}: \mathbb{R}^V\longrightarrow \mathbb{R}$,
\begin{equation*}
 F_p^{(2)}(f) = \frac{\langle f, \Delta_{p}f\rangle}{\min\limits_{c \in \mathbb{R}}\lVert f-c\mathds{1}\rVert_p^p} = \frac{1}{2} \frac{ \sum \limits_{i,j=1}^n w_{ij} \rvert f_i - f_j\rvert^p}{\min\limits_{c \in \mathbb{R}}\lVert f-c\mathds{1}\rVert_p^p}. 
\end{equation*}
\begin{theorem}
The second eigenvalue $\lambda_p^{(2)}$ of the graph p-Laplacian $\Delta_{p}$ is equal to the global minimum of the functional $F_p^{(2)}$. The corresponding eigenvector $v_p^{(2)}$ can be computed using the global minimizer of $F_p^{(2)}$.
\end{theorem}
Proof can be found in~\cite{Buhler09}.
\subsubsection*{Relaxation of Balanced Graph Cuts for p-Laplacian}
The second eigenvector $v_p^{(2)}$ of the p-Laplacian can also be seen as the relaxation of balanced graph cuts~\cite{Buhler09}. For $p>1$ the second eigenvalue $\lambda_p^{(2)}$ is the solution of the relaxation problem
\begin{equation*}
 \min_{C\subset V} \mathrm{cut}(C, \overline{C}) \biggr\rvert \frac{1}{\rvert C\lvert^{\frac{1}{p-1}}} + \frac{1}{\rvert \overline{C}\lvert^{\frac{1}{p-1}}}\biggl\lvert^{p-1}.
\end{equation*}
With the special cases, 
\begin{equation*}
 \begin{aligned}
  p=2, \quad \min_{C\subset V} \mathrm{RCut}(C, \overline{C}),\\
  p\rightarrow 1, \quad \min_{C\subset V} \mathrm{RCC}(C, \overline{C}).
 \end{aligned}
\end{equation*}
In order to obtain the partitioning of the graph we use the standard procedure and threshold the second eigenvector $v_p^{(2)}$.  The optimal threshold is determined by minimizing the corresponding Cheeger cut.
For the unnormalized graph p-Laplacian $\Delta_{p}^{(u)}$ we determine 
\begin{equation}
 h_{\mathrm{RCC}}^* = \min_{C_t = \{i \in V | v_p^{(2)}(i) > t\}} \mathrm{RCC}(C_t, \overline{C}_t),
\end{equation}
and similarly for the normalized graph p-Laplacian $\Delta_{p}^{(n)}$ 
\begin{equation}
 h_{\mathrm{NCC}}^* = \min_{C_t = \{i \in V | v_p^{(2)}(i) > t\}} \mathrm{NCC}(C_t, \overline{C}_t).
\end{equation}
The question is how good the cut values obtained by thresholding the second eigenvector of the p-Laplacian compared to optimal Cheeger cuts. There are extensions of isoperimetric inequalities for $p>1$ for the normalized 
p-Laplacian~\cite{Amghibech03}
\begin{equation*}
 2^{p-1} \biggr ( \frac{h_{\mathrm{NCC}}}{p} \biggl )^p \leq \lambda_p^{(2)} \leq 2^{p-1} h_{\mathrm{NCC}},
\end{equation*}
 and for the unnormalized p-Laplacian~\cite{Buhler09}
 \begin{equation*}
 \biggr (\frac{2}{\max_i d_i} \biggl)^{p-1} \biggr ( \frac{h_{\mathrm{RCC}}}{p} \biggl)^p \leq \lambda_p^{(2)} \leq 2^{p-1} h_{\mathrm{RCC}}.
 \end{equation*}
It has been shown in~\cite{Buhler09}, that for $p>1$ 
\begin{equation}
\label{eq:ineq}
 \begin{aligned}
  h_{\mathrm{RCC}} &\leq h_{\mathrm{RCC}}^* \leq p(\max_{i \in V} d_i)^{\frac{p-1}{p}} (h_{\mathrm{RCC}})^{\frac{1}{p}},\\
  h_{\mathrm{NCC}} &\leq h_{\mathrm{NCC}}^* \leq p( h_{\mathrm{NCC}})^{\frac{1}{p}}.
 \end{aligned}
\end{equation}
When considering the limit $p\rightarrow 1$, one can observe that the upper bound for both inequalities~\eqref{eq:ineq} gets tight, which means that the cut obtained by thresholding the second eigenvalue of the p-Laplacian converges to 
the optimal Cheeger cut. 
\subsubsection*{1-Spectral Clustering Algorithm}
To obtain the desired partition into $k$ clusters one uses the sequential splitting with thresholding of the second eigenvector of the p-Laplacian. Applying k-means in this setting is not possible, since at the moment 
we are not able to compute higher-order eigenvectors of the p-Laplacian. 

The algorithmic scheme for 1-spectral clustering is presented below.
\incmargin{1em}
%\linesnumberedhidden
\begin{algorithm}[hbtp]
\caption{1-Spectral Clustering}
\label{SCv1}
\dontprintsemicolon
\BlankLine
\Indm  
\KwIn{affinity matrix $W$, number $k$ of clusters to construct, choice of the graph 1-Laplacian (normalized, unnormalized).}
\Setnlsty{textbf}{}{:}
\Indp
\BlankLine
%\showln
initialization: cluster $C_1 = V$, number of clusters $s=1$;\\
%\showln
\Repeat{
%\showln
number of clusters $s=k$}
{
minimize $F_p^{(2)}: \mathbb{R}^{C_i}\longrightarrow \mathbb{R}$ for the chosen p-Laplacian for each cluster $C_i$, $i=1,\dots,s$;\;
compute the optimal threshold for dividing each cluster $C_i$;\;
choose to split the cluster $C_i$ so that the total multipartition cut criterion is minimized;\;
$s\Longleftarrow s+1$.
}
\BlankLine
\Indm  
\KwOut{Clusters $C_1,\dots,C_k$.} 
\end{algorithm}
\decmargin{1em}
\newpage

The functional $F_p^{(2)}$ is non-convex and thus we cannot guarantee to reach the global minimum. Although a direct minimization for small values of $p$ leads often to very fast convergence to a non-optimal local minimum.
Empirical observations show that the cut found by 1-spectral clustering is always at least as good as the cut found by standard spectral clustering - often much better (see fig.~\ref{fig:1sc_sc_fig}).
\begin{figure}[htbp]
 \centering
 \subfigure[A simple graph in two dimensions]{%
 \includegraphics[width=0.32\textwidth]{images/png/sc_vs_1sc_1.png}
\label{fig:subfigure1}}
\qquad
 \subfigure[Eigenvector of the graph Laplacian]{%
 \includegraphics[width=0.32\textwidth]{images/png/sc_vs_1sc_2.png}
\label{fig:subfigure2}}
\qquad
\subfigure[Partition obtained by thresholding the second eigenvector of the graph Laplacian]{%
 \includegraphics[width=0.32\textwidth]{images/png/sc_vs_1sc_3.png}
\label{fig:subfigure1}}
\qquad
 \subfigure[Eigenvector of the graph 1-Laplacian]{%
 \includegraphics[width=0.32\textwidth]{images/png/sc_vs_1sc_4.png}
\label{fig:subfigure2}}
 \caption[1-spectral clustering versus standard spectral clustering]{
  {\bf 1-Spectral Clustering vs. Spectral Clustering} (courtesy of~\cite{Setzer12}).}
\label{fig:1sc_sc_fig}
\end{figure}
\sloppy
  
In further work it was shown by~\cite{Hein10} that 1-spectral clustering can be naturally formulated as nonlinear eigenproblem and the inverse power method can be applied to compute the eigenvectors of the 1-Laplacian. 
The method is guaranteed to converge to a non-constant eigenvector of the 1-Laplacian. However, the convergence to the second eigenvector is not guaranteed. Thus it is recommended to use multiple random initializations and 
use the result which achieves the best cut value.

In~\cite{HeinS11} the generalization of the spectral relaxation for almost any balanced graph cut criterion was proposed. A characterization of all balanced graph cuts has been provided, which allows a tight relaxation into 
continuous problem. Although the resulting optimization problem is non-convex and non-smooth, it can be efficiently solved by the method for the minimization of ratios of differences of convex functions, called RatioDCA. 
\section{Discussion}
\label{ch2:disc}
In this chapter we motivated spectral clustering as a relaxation of a NP-hard balanced graph cut problem and presented two relaxation methods with different graph cut objectives.

The standard 2-norm relaxation leads to a linear eigenproblem for the graph Laplacian, where the second eigenvectors
of the unnormalized and normalized graph Laplacians correspond to relaxations of the ratio cut and normalized cut. 
This spectral relaxation is known to be loose and may lead to a solution far from the optimal one of the original problem.
However, there exists a tight relaxation of a balanced graph cut problem using the nonlinear 1-graph Laplacian, called 1-spectral clustering.
This generalized non-linear eigenproblem in practice provides much better graph cuts. 
In contrast to spectral clustering, 1-spectral clustering employs a recursive splitting scheme for multipartitioning instead of a multiway approach. The reason for this is
that the computation of higher-order eigenvectors of the graph 1-Laplacian is not feasible at present.

Many successful video segmentation algorithms are based on the spectral relaxation techniques. 
But despite of all the achieved progress, adaptation of spectral clustering to video processing requires further researching.
Little attention has been paid to the effects of the 2-norm relaxation 
with different cut criteria applied to video segmentation and the tight 1-norm relaxation has not been yet adopted to video processing. 
In this thesis we aim to study the relevance of these methods for video segmentation.
\clearpage

\chapter{From edges to contours} % - Current State-of-the-Art}
\label{Chapter3}
\section{gPb-owt-ucm algorithm pipeline}
The gPb algorithm utilises local cues - colour, brightness and texture to build features. Those are then globalised using spectral clustering. Multiple-scale approach gives competitive edge detection results.

it is straightforward to close contours using a watershed operation (basic morphological operation)

watershed contours is a binary image (c.f. edge map) indicating the highest level of recall

\begin{figure}[ht!]
\centering
 \includegraphics[width=1\textwidth]{images/gPb-OWT-UCM/gPb-OWT-UCM_pipeline.png}
\caption{gPb-OWT-UCM algorithm. We have expanded the pipeline to explicitly include the ``watershed transform'' operation and its output - watershed contours.}
\label{fig:gPb-OWT-UCM-pipeline}
\end{figure}

\subsection{Quantised oriented probability of boundary}
\subsection{Weighted watershed}
\section{Limitations of quantisation} % or drawbacks, don't say 'flaws'
\clearpage

\chapter{Experimental Study of Spectral Clustering Techniques}
\label{Chapter4}

In this chapter we aim to study the effect of different spectral methods applied to video segmentation. 
We start with a brief recap of the main theoretical aspects of two relaxation techniques of the balanced graph cut problem - spectral clustering and 1-spectral clustering 
and then introduce the boundary and volume oriented metric of~\cite{Galasso13} for evaluation of the video segmentation results.

We continue with a number of experiments comparing the performance of 1-norm and 2-norm relaxations.
To make the comparison fair, we use an identical setting for both methods. The results are tested on the BMDS (see sec.~\ref{sec:ch3_dataset}). 

Finally, in order to analyze the behaviour of different balanced cut criteria and the solution produced by mentioned above methods we present several experiments with the ground truth, where we tried to find a better partition by 
a trivial greedy search optimizing different objective functions: NCut, RCut, NCC, RCC, MinMaxCut and Cut (see sec.~\ref{sec:ch2_balgrcut}) and see if the optimal in the sense of balanced graph cuts solution is coincides with the 
human annotated segmentation.  
%%
\section{Spectral Methods for Video Segmentation}
\label{ch4:recap}
In the proposed video segmentation model (see sec.~\ref{sec:ch3_framework}) first the graph is constructed on pre-computed superpixels and then the spectral method is applied to obtain the final segmentation. 
In this work we consider two relaxation techniques of balanced graph cuts: spectral clustering (see sec.~\ref{sec:ch2_spectclus}) 
and 1-spectral clustering (see sec.~\ref{sec:ch2_1spectclus}).

The standard 2-norm approach leads to a linear eigenproblem for the graph Laplacian, where the second eigenvector of the graph Laplacian corresponds to relaxation of the balanced graph cut.
Spectral clustering has proved to be successful in segmentation. However, it is known to be loose and may yield solutions far from the optimal one.
1-spectral clustering gives a tight relaxation and in practice provides much better graph cuts than standard spectral clustering approach.
This relaxation leads to a non-linear eigenproblem for the graph 1-Laplacian and the method does not guarantee convergence to the global optimum.

In contrast to spectral clustering, in 1-spectral clustering for multipartitioning the sequential splitting procedure with thresholding of the second eigenvector of the 1-Laplacian is used.
Applying k-means like for the 2-norm relaxation is not possible here, as at the moment computation of higher-order eigenvectors of the 1-Laplacian is not feasible.

Both methods can optimize different balanced graph cut objectives (see sec.~\ref{sec:ch2_balgrcut}). However, the NCut is recommended itself by showing the state-of-the-art performance in image and video segmentation.
In the experiments we consider normalized and unnormalized spectral clustering, which relaxes the NCut and RCut respectively. 
Concerning 1-spectral clustering we use the following multipartition criteria: NCut, RCut, NCC and RCC.

For all the experiments for 1-spectral clustering the MATLAB implementation by~\cite{Buhler09} was used. The source code can be found at
\url{http://www.ml.uni-saarland.de/code/oneSpectralClustering/1SpectralClustering_V1_1.rar}. For the standard spectral clustering we used
an implementation by~\cite{GalassoCS12} which is available at \url{http://www.mpi-inf.mpg.de/~galasso/files/VSS.zip}.
%In this section our goal is to investigate the benefits and shortcomings of employing different relaxation methods in video segmentation. 
\section{Evaluation Benchmark for Video Segmentation}
\label{ch4:bench}
In order to examine video segmentation results of different spectral methods we first need to choose a metric that could deal with segmentation hierarchies and reflect the tradeoff between over-segmentation and segmentation accuracy.

Recently it was proposed by~\cite{Galasso13} to evaluate video segmentation performance with a boundary and volume oriented metrics against human ground-truth.  
This new benchmark can evaluate under- and over-segmentation and also takes into account temporal consistency of the segmentation. 
\subsection{Boundary precision-recall (BPR)}
The boundary-based evaluation metric developed by~\cite{Martin01,Arbelaez11} on the Berkeley segmentation dataset (BSDS) has become a standard for image segmentation. It estimates the quality of a segmentation boundary map
in the precision-recall framework. Precision measures the fraction of true positives in the produced contours and recall the fraction of ground truth boundary pixels detected:
\begin{equation*}
\begin{aligned}
 P=\frac{ \lvert S\bigcap \bigr ( \bigcup_{i=1}^M G_i \bigl ) \rvert}{\rvert S \lvert},\\
 R =\frac{ \sum_{i=1}^M \lvert S\bigcap G_i \rvert}{\sum_{i=1}^M\rvert G_i \lvert},
\end{aligned}
\end{equation*}
where $S$ is the computer generated boundary map and $\{ G_i\}_{i=1}^M$ are the sets of ground-truth boundaries.
To estimate the total performance the global F-measure is defined as the harmonic mean of precision and recall:
\begin{equation*}
 F = \frac{2PR}{P+R}.
\end{equation*}

The main limitation of this metric is that it evaluates every frame independently and does not consider consistency of segmentation across frames. Although it estimates the localization accuracy of segmentation boundaries quite well,
the methodology that directly measures the quality of segments is desirable. Therefore the following volume-based metric is considered.
\subsection{Volume precision-recall (VPR)}
The volume metric developed by~\cite{Galasso13} measures the spatio-temporal overlap between the machine generated segmentation $S$ and the human annotated segmentations $\{ G_i\}_{i=1}^M$. 
To avoid high scores with this metric for over- and under-segmentation (every pixel being a separate segment or a single segment for the whole video sequence) a lower bounds are subtracted from the overlap score: 
\begin{equation*}
\begin{aligned}
 P=\frac{ \sum_{i=1}^M \bigl [ \{ \sum_{s\in S} \max_{g \in G_i} \lvert s \bigcap g \rvert\} - \max_{g \in G_i} \rvert g\lvert \bigr ]}{M\rvert S \lvert - \sum_{i=1}^M \max_{g \in G_i}},\\
\qquad
\quad
 R =\frac{ \sum_{i=1}^M  \sum_{g\in G_i} \{\max_{s \in S} \lvert s\bigcap g \rvert-1 \}}{\sum_{i=1}^M \{ \rvert G_i \lvert - \Gamma_{G_i} \}},
\end{aligned}
\end{equation*}
where $\Gamma_{G_i}$ is the number of volumes in the ground truth $G_i$.

For both BPR and VPR  three different quantities are reported: the Optimal Dataset Scale (ODS) - the best F-measure on the dataset for a fixed scale, the Optimal Segmentation Scale (OSS) - the aggregate F-measure over the dataset for 
the best scale and the Average Precision (AP) on full recall range - the area under the precision-recall curve. %All precision-recall curves in the experiments are obtained by hierarchical clustering

For comparison we also consider different complementary region-based metrics: the Variation of Information (VI)~\cite{Meila05}, which measures the distance between two segmentations in terms of their average conditional entropy and mutual information, the 
Probabilistic Rand Index (PRI)~\cite{rand1971, UnnikrishnanPH07}, which counts the number of pixel pairs which labels are consistent between the machine generated and human annotated segmentations, and the Segmentation Covering (SG)
~\cite{Arbelaez09}, which estimates the best possible covering of the ground truth by segments.  

The benchmark code proposed by~\cite{Galasso13} can be found at  \url{http://www.mpi-inf.mpg.de/~galasso/files/Benchmark.zip}.
\section{Experiments on 1-Spectral Clustering vs. Spectral Clustering}
\label{sec:ch4_1sc_vs_sc}
In this section our goal is to investigate the benefits and shortcomings of employing 1-norm and 2-norm relaxation methods with different balanced graph cut objectives in video segmentation.
As a first step we directly apply our proposed video segmentation model, in which to obtain the final segmentation we use spectral relaxation techniques. 
% Here we consider two techniques: spectral clustering (see sec.~\ref{sec:ch2_spectclus}) 
% and 1-spectral clustering (see sec.~\ref{sec:ch2_1spectclus}). 2-norm relaxation has proved to be successful in segmentation, but may yield solutions far from the optimal one. 
% Theoretically, 1-spectral clustering gives a tight relaxation, which should lead to a better solution than standard spectral clustering approach. 
% Even though, it does not guarantee convergence to the global optimum.
% Both methods can optimize different objective functions. However, the NCut is recommended itself by showing the state-of-the-art performance in image and video segmentation.
% 
% In our setup we consider normalized and unnormalized spectral clustering, which relaxes the NCut and RCut respectively. 
% Concerning 1-spectral clustering we use the following multipartition criteria: NCut, RCut, NCC and RCC.

Due to the fact that in video segmentation we deal with superpixels in order to reduce computational complexity, in the balancing term of the ratio cuts the cardinality of cluster could be understood in two ways.
Originally by the cardinality we consider the number of vertices in the cluster, which is in our work the number of superpixels. This could lead to very unbalanced clusters as the actual size of
superpixels may vary significantly, for instance for the video sequence ``Marple4`` from 4 to 8203 pixels. To avoid this potential problem and strengthen the balancing term we suggest to understand by the size of clusters 
the actual number of pixels.
Therefore, in our experiments for 1-norm relaxation we distinguish the ratio cut and the ratio Cheeger cut balanced by the number and the actual size of superpixels. 
   
%As a first step we directly apply our proposed video segmentation model, in which to obtain the final segmentation we use 1-norm and 2-norm relaxations with different cut criteria.
% For all the experiments with 1-spectral clustering the MATLAB implementation by~\cite{Buhler09} was used. The source code can be found at
% \url{http://www.ml.uni-saarland.de/code/oneSpectralClustering/1SpectralClustering_V1_1.rar}.
Figure~\ref{fig:seg_res} illustrates boundary and volume precision-recall curves (BPR and VPR respectively) on the BMDS for the suggested experimental settings. 
The curves are obtained by hierarchical clustering, 
starting from 2 till 600 clusters, where the top-left part of BPR and the down-right part of VPR corresponds to a small number of clusters. 
Numerical comparison is presented in Table~\ref{tab:sc_comparison}.
\begin{figure}[htbp]
\centering
\subfigure[BPR]{%
\includegraphics[trim=0cm 0cm 0cm 0.5cm, clip=true, width=0.48\textwidth]{images/basic/2.png}}
%\hfill
\quad
% \subfigure{%
% \includegraphics[trim=0cm 0cm 0cm 0.8cm, clip=true, width=0.47\textwidth]{images/basic/3.png}}
\subfigure[VPR]{%
\includegraphics[trim=0cm 0cm 0cm 0.5cm, clip=true, width=0.48\textwidth]{images/basic/4.png}}

\caption[Boundary precision-recall (BPR) and volume precision-recall (VPR) curves for spectral clustering (SC) and 1-spectral clustering (1SC) with different cut criteria]{
{\bf Boundary precision-recall (BPR) and volume precision-recall (VPR) curves for spectral clustering (SC) and 1-spectral clustering (1SC) with different cut criteria.}}
\label{fig:seg_res}
\end{figure}

\begin{table}[htbp]
\renewcommand{\arraystretch}{1.3}
\centering
\scriptsize
%\sffamily
\begin{tabular}{|l||c|c|c||c|c|c||c|c|c||}
\hline 
\multirow{2}{*}{\textbf{Method}} & \multicolumn{3}{c||}{\textbf{BPR}} & \multicolumn{3}{c||}{\textbf{VPR}}& \multicolumn{3}{c||}{\textbf{Region}}\\
\cline{2-10}
& \textbf{ODS}  & \textbf{OSS} & \textbf{AP}
& \textbf{ODS} & \textbf{OSS} & \textbf{AP}
& \textbf{SC} & \textbf{PRI} & \textbf{VI} \\
\hline
\hline
\textbf{SC, NCut} & 0.36 & 0.39 & 0.21 & 0.59 &\textbf{0.72} & \textbf{0.59} & \textbf{0.89} & \textbf{0.78} & \textbf{0.67} \\
\hline
\textbf{SC, RCut} & 0.36 & 0.40 & 0.21 & 0.55 & 0.65 & 0.51 & 0.87 & 0.73 & \textbf{0.67} \\
\hline
\hline
\textbf{1SC, NCut} & 0.32 & 0.33 & 0.17 & 0.55 & 0.64 & 0.47 & 0.69 & 0.63 & 1.13 \\
\hline
\textbf{1SC, NCC} & 0.25 & 0.29 & 0.12 & 0.50 & 0.56 & 0.40 & 0.61 & 0.58 & 1.31 \\
\hline
\textbf{1SC, RCut (\# of spx)} &\textbf{0.40} & \textbf{0.45} & 0.25 & \textbf{0.62} & 0.69 & 0.58 & 0.82 & 0.70 & 0.86 \\
\hline
\textbf{1SC, RCC (\# of spx)} & 0.38 & 0.42 & 0.22 & 0.58 & 0.65 & 0.52 & 0.76 & 0.68 & 0.97 \\
\hline
\textbf{1SC, RCut (size of spx)} & \textbf{0.40} & \textbf{0.45} & \textbf{0.28} & 0.61 & 0.69 & 0.57 & 0.77 & 0.71 & 0.93 \\
\hline
\textbf{1SC, RCC (size of spx)} & 0.38 & 0.43 & 0.26 & 0.58 & 0.65 & 0.51 & 0.72 & 0.66 & 1.06 \\
\hline
\end{tabular}
 \caption{{\bf Comparison of spectral clustering (SC) and 1-spectral clustering (1SC) optimizing different cut criteria.} 
The table shows aggregate measures (ODS, OSS, AP) for boundary precision-recall (BPR), volume precision-recall (VPR) and 
includes region statistics (SC, PRI, VI).}
\label{tab:sc_comparison}
\end{table}
As it can be seen from the quantitative results normalized spectral clustering outperforms 1-norm relaxation despite its theoretical properties. The performance of the NCut and NCC for 1-spectral clustering is quite low,
especially for small number of clusters. Surprisingly the ratio cuts perform better than normalized ones. They improve upon standard spectral clustering in boundary metric and give comparable results in volume oriented metric, 
particularly for a bigger number of clusters. Our proposed for the RCut and RCC balancing by the size of superpixels enhances the results compared to the balancing by number of superpixels. 
The unnormalized spectral clustering yields comparable results with the normalized one in BPR, but performs poorly in VPR. 

These quantitative results are supported by qualitative results. Figures~\ref{fig:seg_res_C1} and~\ref{fig:seg_res_M4} illustrate final segmentations for video sequences ''Cars1'' for 2 clusters and ``Marple4'' for 3 clusters.

It was observed that the RCut and RCC, normalized by the number of superpixels, have a tendency to disconnect one small superpixel in one of the frames and usually require more clusters
to obtain meaningful solution where objects are well distinguished. 
The standard spectral clustering and tight relaxation for the ratio cuts, balanced by the size of superpixels, usually provide comparable segmentation results, where the RCut is slightly better than the RCC . 
The NCut and NCC tend to separate the sequence into clusters of equal size by the horizontal or vertical line, which could be explained by the strong balancing term of the normalized cuts. This results in poor 
overall performance.
%\subsection{Qualitative results}
\begin{figure}[ht!]
\begin{minipage}[t]{1\textwidth}
 \centering
\hfill \hfill  \hfill
\footnotesize Frame 1
\hfill  \hfill
\footnotesize Frame 9
\hfill  \hfill
\footnotesize Frame 11
\hfill  \hfill
\footnotesize Frame 19
\hfill \hfill  \hfill
\end{minipage}
% 

\begin{minipage}[t]{1\textwidth}
\centering
\hfill \hfill  \hfill
 \subfigure{%
 \includegraphics[width=0.16\textwidth]{images/pres/C1/cars1_01.jpg}}
\hfill
\subfigure{%
 \includegraphics[width=0.16\textwidth]{images/pres/C1/cars1_09.jpg}}
\hfill
\subfigure{%
 \includegraphics[width=0.16\textwidth]{images/pres/C1/cars1_11.jpg}}
\hfill
\subfigure{%
 \includegraphics[width=0.16\textwidth]{images/pres/C1/cars1_19.jpg}}
\hfill \hfill  \hfill

\footnotesize (a) Marple4
\end{minipage}
\begin{minipage}[t]{1\textwidth}
\centering
\hfill \hfill  \hfill
 \subfigure{%
 \includegraphics[width=0.16\textwidth]{images/pres/C1/cars1_01.png}}
\hfill
\subfigure{%
 \includegraphics[width=0.16\textwidth]{images/pres/C1/cars1_09.png}}
\hfill
\subfigure{%
 \includegraphics[width=0.16\textwidth]{images/pres/C1/cars1_11.png}}
\hfill
\subfigure{%
 \includegraphics[width=0.16\textwidth]{images/pres/C1/cars1_19.png}}
\hfill \hfill  \hfill

\footnotesize (b) Ground Truth
\end{minipage}
\begin{minipage}[t]{1\textwidth}
\centering
\hfill \hfill  \hfill
 \subfigure{%
 \includegraphics[trim=0cm 0cm 0cm 1cm, clip=true, width=0.17\textwidth]{images/pres/C1/sc/001.jpg}}
\hfill
\subfigure{%
 \includegraphics[trim=0cm 0cm 0cm 1cm, clip=true, width=0.17\textwidth]{images/pres/C1/sc/009.jpg}}
\hfill
\subfigure{%
 \includegraphics[trim=0cm 0cm 0cm 1cm, clip=true, width=0.17\textwidth]{images/pres/C1/sc/011.jpg}}
\hfill
\subfigure{%
 \includegraphics[trim=0cm 0cm 0cm 1cm, clip=true, width=0.17\textwidth]{images/pres/C1/sc/019.jpg}}
\hfill \hfill  \hfill

\footnotesize (c) Spectral Clustering
\end{minipage}
\begin{minipage}[t]{1\textwidth}
\centering
\hfill \hfill  \hfill
 \subfigure{%
 \includegraphics[trim=0cm 0cm 0cm 1cm, clip=true, width=0.17\textwidth]{images/pres/C1/ncut/001.jpg}}
\hfill
\subfigure{%
 \includegraphics[trim=0cm 0cm 0cm 1cm, clip=true, width=0.17\textwidth]{images/pres/C1/ncut/009.jpg}}
\hfill
\subfigure{%
 \includegraphics[trim=0cm 0cm 0cm 1cm, clip=true, width=0.17\textwidth]{images/pres/C1/ncut/011.jpg}}
\hfill
\subfigure{%
 \includegraphics[trim=0cm 0cm 0cm 1cm, clip=true, width=0.17\textwidth]{images/pres/C1/ncut/019.jpg}}
\hfill \hfill  \hfill

\footnotesize (d) 1-Spectral Clustering, NCut
\end{minipage}
\begin{minipage}[t]{1\textwidth}
\centering
\hfill \hfill  \hfill
 \subfigure{%
 \includegraphics[trim=0cm 0cm 0cm 1cm, clip=true, width=0.17\textwidth]{images/pres/C1/ncc/001.jpg}}
\hfill
\subfigure{%
 \includegraphics[trim=0cm 0cm 0cm 1cm, clip=true, width=0.17\textwidth]{images/pres/C1/ncc/009.jpg}}
\hfill
\subfigure{%
 \includegraphics[trim=0cm 0cm 0cm 1cm, clip=true, width=0.17\textwidth]{images/pres/C1/ncc/011.jpg}}
\hfill
\subfigure{%
 \includegraphics[trim=0cm 0cm 0cm 1cm, clip=true, width=0.17\textwidth]{images/pres/C1/ncc/019.jpg}}
\hfill \hfill  \hfill

\footnotesize (e) 1-Spectral Clustering, NCC
\end{minipage}
\begin{minipage}[t]{1\textwidth}
\centering
\hfill \hfill  \hfill
 \subfigure{%
 \includegraphics[trim=0cm 0cm 0cm 1cm, clip=true, width=0.17\textwidth]{images/pres/C1/rcut_old/001.jpg}}
\hfill
\subfigure{%
 \includegraphics[trim=0cm 0cm 0cm 1cm, clip=true, width=0.17\textwidth]{images/pres/C1/rcut_old/009.jpg}}
\hfill
\subfigure{%
 \includegraphics[trim=0cm 0cm 0cm 1cm, clip=true, width=0.17\textwidth]{images/pres/C1/rcut_old/011.jpg}}
\hfill
\subfigure{%
 \includegraphics[trim=0cm 0cm 0cm 1cm, clip=true, width=0.17\textwidth]{images/pres/C1/rcut_old/019.jpg}}
\hfill \hfill  \hfill

\footnotesize (f) 1-Spectral Clustering, RCut (balanced by number of superpixels)
\end{minipage}
\begin{minipage}[t]{1\textwidth}
\centering
\hfill \hfill  \hfill
 \subfigure{%
 \includegraphics[trim=0cm 0cm 0cm 1cm, clip=true, width=0.17\textwidth]{images/pres/C1/rcc_old/001.jpg}}
\hfill
\subfigure{%
 \includegraphics[trim=0cm 0cm 0cm 1cm, clip=true, width=0.17\textwidth]{images/pres/C1/rcc_old/009.jpg}}
\hfill
\subfigure{%
 \includegraphics[trim=0cm 0cm 0cm 1cm, clip=true, width=0.17\textwidth]{images/pres/C1/rcc_old/011.jpg}}
\hfill
\subfigure{%
 \includegraphics[trim=0cm 0cm 0cm 1cm, clip=true, width=0.17\textwidth]{images/pres/C1/rcc_old/019.jpg}}
\hfill \hfill  \hfill

\footnotesize (g) 1-Spectral Clustering, RCC (balanced by number of superpixels)
\end{minipage}
\begin{minipage}[t]{1\textwidth}
\centering
\hfill \hfill  \hfill
 \subfigure{%
 \includegraphics[trim=0cm 0cm 0cm 1cm, clip=true, width=0.17\textwidth]{images/pres/C1/rcut/001.jpg}}
\hfill
\subfigure{%
 \includegraphics[trim=0cm 0cm 0cm 1cm, clip=true, width=0.17\textwidth]{images/pres/C1/rcut/009.jpg}}
\hfill
\subfigure{%
 \includegraphics[trim=0cm 0cm 0cm 1cm, clip=true, width=0.17\textwidth]{images/pres/C1/rcut/011.jpg}}
\hfill
\subfigure{%
 \includegraphics[trim=0cm 0cm 0cm 1cm, clip=true, width=0.17\textwidth]{images/pres/C1/rcut/019.jpg}}
\hfill \hfill  \hfill

\footnotesize (h) 1-Spectral Clustering, RCut (balanced by size of superpixels)
\end{minipage}
\begin{minipage}[t]{1\textwidth}
\centering
\hfill \hfill  \hfill
 \subfigure{%
 \includegraphics[trim=0cm 0cm 0cm 1cm, clip=true, width=0.17\textwidth]{images/pres/C1/rcc/001.jpg}}
\hfill
\subfigure{%
 \includegraphics[trim=0cm 0cm 0cm 1cm, clip=true, width=0.17\textwidth]{images/pres/C1/rcc/009.jpg}}
\hfill
\subfigure{%
 \includegraphics[trim=0cm 0cm 0cm 1cm, clip=true, width=0.17\textwidth]{images/pres/C1/rcc/011.jpg}}
\hfill
\subfigure{%
 \includegraphics[trim=0cm 0cm 0cm 1cm, clip=true, width=0.17\textwidth]{images/pres/C1/rcc/019.jpg}}
\hfill \hfill  \hfill

\footnotesize (i) 1-Spectral Clustering, RCC (balanced by size of superpixels)
\end{minipage}
 \caption[Segmentation results for the video sequence ``Cars1`` for 2 clusters]{
  {\bf Segmentation results for the video sequence ``Cars1`` for 2 clusters.}}
\label{fig:seg_res_C1}
\end{figure}
\begin{figure}[ht!]
\begin{minipage}[t]{1\textwidth}
 \centering
\hfill \hfill  \hfill
\footnotesize Frame 1
\hfill  \hfill
\footnotesize Frame 10
\hfill  \hfill
\footnotesize Frame 20
\hfill  \hfill
\footnotesize Frame 30
\hfill \hfill  \hfill
\end{minipage}
% 

\begin{minipage}[t]{1\textwidth}
\centering
\hfill \hfill  \hfill
 \subfigure{%
 \includegraphics[width=0.16\textwidth]{images/pres/M4/marple4_324.jpg}}
\hfill
\subfigure{%
 \includegraphics[width=0.16\textwidth]{images/pres/M4/marple4_333.jpg}}
\hfill
\subfigure{%
 \includegraphics[width=0.16\textwidth]{images/pres/M4/marple4_343.jpg}}
\hfill
\subfigure{%
 \includegraphics[width=0.16\textwidth]{images/pres/M4/marple4_353.jpg}}
\hfill \hfill  \hfill

\footnotesize (a) Marple4
\end{minipage}
\begin{minipage}[t]{1\textwidth}
\centering
\hfill \hfill  \hfill
 \subfigure{%
 \includegraphics[width=0.16\textwidth]{images/pres/M4/marple4_324.png}}
\hfill
\subfigure{%
 \includegraphics[width=0.16\textwidth]{images/pres/M4/marple4_333.png}}
\hfill
\subfigure{%
 \includegraphics[width=0.16\textwidth]{images/pres/M4/marple4_343.png}}
\hfill
\subfigure{%
 \includegraphics[width=0.16\textwidth]{images/pres/M4/marple4_353.png}}
\hfill \hfill  \hfill

\footnotesize (b) Ground Truth
\end{minipage}
\begin{minipage}[t]{1\textwidth}
\centering
\hfill \hfill  \hfill
 \subfigure{%
 \includegraphics[trim=0cm 0cm 0cm 1cm, clip=true, width=0.17\textwidth]{images/pres/M4/sc/001.jpg}}
\hfill
\subfigure{%
 \includegraphics[trim=0cm 0cm 0cm 1cm, clip=true, width=0.17\textwidth]{images/pres/M4/sc/010.jpg}}
\hfill
\subfigure{%
 \includegraphics[trim=0cm 0cm 0cm 1cm, clip=true, width=0.17\textwidth]{images/pres/M4/sc/020.jpg}}
\hfill
\subfigure{%
 \includegraphics[trim=0cm 0cm 0cm 1cm, clip=true, width=0.17\textwidth]{images/pres/M4/sc/030.jpg}}
\hfill \hfill  \hfill

\footnotesize (c) Spectral Clustering
\end{minipage}
\begin{minipage}[t]{1\textwidth}
\centering
\hfill \hfill  \hfill
 \subfigure{%
 \includegraphics[trim=0cm 0cm 0cm 1cm, clip=true, width=0.17\textwidth]{images/pres/M4/ncut/001.jpg}}
\hfill
\subfigure{%
 \includegraphics[trim=0cm 0cm 0cm 1cm, clip=true, width=0.17\textwidth]{images/pres/M4/ncut/010.jpg}}
\hfill
\subfigure{%
 \includegraphics[trim=0cm 0cm 0cm 1cm, clip=true, width=0.17\textwidth]{images/pres/M4/ncut/020.jpg}}
\hfill
\subfigure{%
 \includegraphics[trim=0cm 0cm 0cm 1cm, clip=true, width=0.17\textwidth]{images/pres/M4/ncut/030.jpg}}
\hfill \hfill  \hfill

\footnotesize (d) 1-Spectral Clustering, NCut
\end{minipage}
\begin{minipage}[t]{1\textwidth}
\centering
\hfill \hfill  \hfill
 \subfigure{%
 \includegraphics[trim=0cm 0cm 0cm 1cm, clip=true, width=0.17\textwidth]{images/pres/M4/ncc/001.jpg}}
\hfill
\subfigure{%
 \includegraphics[trim=0cm 0cm 0cm 1cm, clip=true, width=0.17\textwidth]{images/pres/M4/ncc/010.jpg}}
\hfill
\subfigure{%
 \includegraphics[trim=0cm 0cm 0cm 1cm, clip=true, width=0.17\textwidth]{images/pres/M4/ncc/020.jpg}}
\hfill
\subfigure{%
 \includegraphics[trim=0cm 0cm 0cm 1cm, clip=true, width=0.17\textwidth]{images/pres/M4/ncc/030.jpg}}
\hfill \hfill  \hfill

\footnotesize (e) 1-Spectral Clustering, NCC
\end{minipage}
\begin{minipage}[t]{1\textwidth}
\centering
\hfill \hfill  \hfill
 \subfigure{%
 \includegraphics[trim=0cm 0cm 0cm 1cm, clip=true, width=0.17\textwidth]{images/pres/M4/rcut_old/001.jpg}}
\hfill
\subfigure{%
 \includegraphics[trim=0cm 0cm 0cm 1cm, clip=true, width=0.17\textwidth]{images/pres/M4/rcut_old/010.jpg}}
\hfill
\subfigure{%
 \includegraphics[trim=0cm 0cm 0cm 1cm, clip=true, width=0.17\textwidth]{images/pres/M4/rcut_old/020.jpg}}
\hfill
\subfigure{%
 \includegraphics[trim=0cm 0cm 0cm 1cm, clip=true, width=0.17\textwidth]{images/pres/M4/rcut_old/030.jpg}}
\hfill \hfill  \hfill

\footnotesize (f) 1-Spectral Clustering, RCut (balanced by number of superpixels)
\end{minipage}
\begin{minipage}[t]{1\textwidth}
\centering
\hfill \hfill  \hfill
 \subfigure{%
 \includegraphics[trim=0cm 0cm 0cm 1cm, clip=true, width=0.17\textwidth]{images/pres/M4/rcc_old/001.jpg}}
\hfill
\subfigure{%
 \includegraphics[trim=0cm 0cm 0cm 1cm, clip=true, width=0.17\textwidth]{images/pres/M4/rcc_old/010.jpg}}
\hfill
\subfigure{%
 \includegraphics[trim=0cm 0cm 0cm 1cm, clip=true, width=0.17\textwidth]{images/pres/M4/rcc_old/020.jpg}}
\hfill
\subfigure{%
 \includegraphics[trim=0cm 0cm 0cm 1cm, clip=true, width=0.17\textwidth]{images/pres/M4/rcc_old/030.jpg}}
\hfill \hfill  \hfill

\footnotesize (g) 1-Spectral Clustering, RCC (balanced by number of superpixels)
\end{minipage}
\begin{minipage}[t]{1\textwidth}
\centering
\hfill \hfill  \hfill
 \subfigure{%
 \includegraphics[trim=0cm 0cm 0cm 1cm, clip=true, width=0.17\textwidth]{images/pres/M4/rcut/001.jpg}}
\hfill
\subfigure{%
 \includegraphics[trim=0cm 0cm 0cm 1cm, clip=true, width=0.17\textwidth]{images/pres/M4/rcut/010.jpg}}
\hfill
\subfigure{%
 \includegraphics[trim=0cm 0cm 0cm 1cm, clip=true, width=0.17\textwidth]{images/pres/M4/rcut/020.jpg}}
\hfill
\subfigure{%
 \includegraphics[trim=0cm 0cm 0cm 1cm, clip=true, width=0.17\textwidth]{images/pres/M4/rcut/030.jpg}}
\hfill \hfill  \hfill

\footnotesize (h) 1-Spectral Clustering, RCut (balanced by size of superpixels)
\end{minipage}
\begin{minipage}[t]{1\textwidth}
\centering
\hfill \hfill  \hfill
 \subfigure{%
 \includegraphics[trim=0cm 0cm 0cm 1cm, clip=true, width=0.17\textwidth]{images/pres/M4/rcc/001.jpg}}
\hfill
\subfigure{%
 \includegraphics[trim=0cm 0cm 0cm 1cm, clip=true, width=0.17\textwidth]{images/pres/M4/rcc/010.jpg}}
\hfill
\subfigure{%
 \includegraphics[trim=0cm 0cm 0cm 1cm, clip=true, width=0.17\textwidth]{images/pres/M4/rcc/020.jpg}}
\hfill
\subfigure{%
 \includegraphics[trim=0cm 0cm 0cm 1cm, clip=true, width=0.17\textwidth]{images/pres/M4/rcc/030.jpg}}
\hfill \hfill  \hfill

\footnotesize (i) 1-Spectral Clustering, RCC (balanced by size of superpixels)
\end{minipage}
 \caption[Segmentation results for the video sequence ``Marple4`` for 3 clusters]{
  {\bf Segmentation results for the video sequence ``Marple4`` for 3 clusters.}}
\label{fig:seg_res_M4}
\end{figure}

\clearpage
\newpage
\subsection{Recursive Splitting vs. Multiway Spectral Clustering}
In our work we tried to investigate why 1-spectral clustering yields unexpectedly worse results than the standard relaxation, particularly for the case of the normalized cut criteria.
At the moment computation of higher-order eigenvectors of the 1-Laplacian is not feasible, which is one of the main limitations of the 1-norm relaxation. Therefore, the method 
uses sequential splitting for multipartitioning and relies only on the second eigenvector. Whereas in the standard spectral clustering we employ multiway scheme and construct the Laplacian eigenmap, using 6 eigenvectors. 
Working with the first 6 eigenvectors has been empirically proven to be the best choice. Besides the number of different objects varies from 2 to 6 depending on the video sequence and it is recommended to choose the
number of eigenvectors equal to the desired number of clusters. 

In order to have a fair comparison, we conducted several experiments, where we even up the settings of two methods.
In Figure~\ref{fig:bipart} the results where the bipartitioning scheme is applied for spectral clustering can be seen. 
And Figure~\ref{fig:2dim} reports the results in which the construction of the Laplacian eigenmap of
the spectral method is based only on the second eigenvector. 
\begin{figure}[htbp]
 \centering
\subfigure[BPR]{%
\includegraphics[trim=0cm 0cm 0cm 0.5cm, clip=true, width=0.41\textwidth]{images/bipart/2.png}}
\quad%\hfill
% \subfigure{%
% \includegraphics[trim=0cm 0cm 0cm 0.8cm, clip=true, width=0.47\textwidth]{images/basic/3.png}}
\subfigure[VPR]{%
\includegraphics[trim=0cm 0cm 0cm 0.5cm, clip=true, width=0.41\textwidth]{images/bipart/4.png}}
\caption[Evaluation of spectral clustering (SC), spectral clustering with bipartitioning scheme (SC, bipart) and 1-spectral clustering (1SC) with different cut criteria 
on boundary precision-recall (BPR) and volume precision-recall (VPR) curves]{
{\bf Evaluation of spectral clustering (SC), spectral clustering with bipartitioning scheme (SC, bipart) and 1-spectral clustering (1SC) with different cut criteria 
on BPR and VPR curves.}}
% \caption[BPR and VPR curves for spectral clustering (SC), spectral clustering with bipartitioning (SC, bipart) and 1-spectral clustering (1SC) with different cut criteria]{
% {\bf BPR and VPR curves for spectral clustering (SC), spectral clustering with bipartitioning (SC, bipart) and 1-spectral clustering (1SC) with different cut criteria}.}
\label{fig:bipart}
% \end{figure}
\qquad
\vfill
% \begin{figure}[htbp]
%  \centering
\subfigure[BPR]{%
\includegraphics[trim=0cm 0cm 0cm 0.5cm, clip=true, width=0.41\textwidth]{images/2d/2.png}}
\quad%\hfill
% \subfigure{%
% \includegraphics[trim=0cm 0cm 0cm 0.8cm, clip=true, width=0.47\textwidth]{images/basic/3.png}}
\subfigure[VPR]{%
\includegraphics[trim=0cm 0cm 0cm 0.5cm, clip=true, width=0.41\textwidth]{images/2d/4.png}}
\caption[Evaluation of spectral clustering (SC), spectral clustering with 2D Laplacian eigenmap (SC, 2 dim) and 1-spectral clustering (1SC) with different cut criteria 
on boundary precision-recall (BPR) and volume precision-recall (VPR) curves]{
{\bf Evaluation of spectral clustering (SC), spectral clustering with 2D Laplacian eigenmap (SC, 2 dim) and 1-spectral clustering (1SC) with different cut criteria 
on BPR and VPR curves.}}
% \caption[BPR and VPR curves for spectral clustering (SC), spectral clustering with 2D Laplacian eigenmap (SC, 2 dim) and 1-spectral clustering (1SC) with different cut criteria]{
% {\bf BPR and VPR curves for spectral clustering (SC), spectral clustering with 2D Laplacian eigenmap (SC, 2 dim) and 1-spectral clustering (1SC) with different cut criteria}.}
\label{fig:2dim}
\end{figure}

As it can be observed in both settings the performance of normalized spectral clustering drops significantly, while the unnormalized case is less sensitive. 
And in this setup 1-spectral clustering with normalized cuts outperforms the 2-norm relaxation. One could expect a great improvement of the segmentation results for 1-norm relaxation provided that the computation of higher-order
eigenvectors is possible.
\subsection{Unbalanced vs. Balanced}
Examining the qualitative results of the proposed approaches, it was observed that algorithms perform differently depending on the size of the foreground objects and the background of the video sequence. Thus to get the 
better understanding
of the balancing factor, we conducted another experiment, where the dataset was divided into two parts - balanced and unbalanced, each consisting of 13 video sequences.
The decision was made according to the statistics of the video sequence where we considered the foreground-to-background ratio (see fig.~\ref{fig:hist}). The unbalanced part of the dataset includes
sequences where foreground takes in less than 17\% of background.
\begin{figure}[!h]
\centering
\includegraphics[width=0.4\textwidth]{images/1.png}
\caption[Histogram of the foreground-to-background ratio for all video sequences]{
{\bf Histogram of the foreground-to-background ratio for all video sequences}.}
\label{fig:hist}
\end{figure}

The results of the experiment are illustrated in Figure~\ref{fig:bal} and~\ref{fig:unbal}.
\begin{figure}[!hb]
\centering
\subfigure[BPR]{%
\includegraphics[trim=0cm 0cm 0cm 0.5cm, clip=true, width=0.41\textwidth]{images/bal/2.png}}
\quad%\hfill
% \subfigure{%
% \includegraphics[trim=0cm 0cm 0cm 0.8cm, clip=true, width=0.47\textwidth]{images/basic/3.png}}
\subfigure[VPR]{%
\includegraphics[trim=0cm 0cm 0cm 0.5cm, clip=true, width=0.41\textwidth]{images/bal/4.png}}
\caption[BPR and VPR curves on video sequences with balanced sizes of foreground and background objects]{
{\bf BPR and VPR curves on video sequences with balanced sizes of foreground and background objects.} The dashed curves are the one obtained from video sequences with roughly equal objects.
The solid curves are obtained from the whole dataset.}
\label{fig:bal}
\end{figure}
\begin{figure}[!ht]
 \centering
\subfigure[BPR]{%
\includegraphics[trim=0cm 0cm 0cm 0.5cm, clip=true, width=0.41\textwidth]{images/unbal/2.png}}
\quad%\hfill
% \subfigure{%
% \includegraphics[trim=0cm 0cm 0cm 0.8cm, clip=true, width=0.47\textwidth]{images/basic/3.png}}
\subfigure[VPR]{%
\includegraphics[trim=0cm 0cm 0cm 0.5cm, clip=true, width=0.41\textwidth]{images/unbal/4.png}}
\caption[BPR and VPR curves on video sequences with unbalanced sizes of foreground and background objects]{
{\bf BPR and VPR curves on video sequences with unbalanced sizes of foreground and background objects.} The dashed curves are the one obtained from video sequences with objects of highly unbalanced size.
The solid curves are obtained from the whole dataset.}
\label{fig:unbal}
\end{figure}

One can report that both spectral methods with the ratio cut objective perform significantly better on the balanced part of the dataset.
Here the RCut balanced by the size of superpixels achieves the highest result. Whereas for the normalized spectral clustering the drop in the performance can be observed. And vice versa for the unbalanced sequences. 
One can see the increase in performance for normalized cuts, particularly for the spectral clustering. 
Therefore we consider that it would be of our interest to analyze the behaviour of the normalized and ratio graph cut criteria on different types of video sequences.  
\section{Experiments with Ground Truth and Greedy Search}
\label{sec:ch4_GTexp}
In the next step of our research the goal was to explore further the balanced graph cut criteria and the quality of the solutions obtained by the relaxation techniques. 
We wanted to find an answer for the following questions:
\begin{itemize}
\item Could we find a better partition by the trivial greedy search optimizing one of the cut criteria?
\item Which objective function would give the best result for different types of sequences?
\item Would be the obtained solution close to the ground truth?
\item Does the ground truth correspond to the minimum cut value?
\end{itemize}
To have a better understanding we conducted several experiments with the ground truth in the same manner.
In order to reduce the size of the graph, we restrict ourselves to 5 frames and choose 2 video sequences: ''Marple4`` from the balanced part of the dataset and ''Cars6`` from the unbalanced part.
For each sequence we construct the similarity graph using the same affinities as before and partition it into 2 (foreground and background) or 3 clusters according to 
the human annotated segmentations or just use the output segmentation of the algorithm
as a starting point.

Next our aim was to see if we can find a better solution by a greedy search. However, looking through all possible partitions to find the optimal one is not feasible, as the problem is NP-hard. 
Therefore we apply the following simplified scheme.
We iteratively change the size of the foreground object in two directions: dilation and erosion, either adding to or subtracting a superpixel from the foreground cluster. 
In each step for dilation or erosion we choose one of the superpixels which yields the partition with minimum value of the objective function.
So in the end if a partition with a smaller cut criterion value is found by this simple search, we can determine that the obtained solution is far from the optimal one.

The basic scheme for the simplified greedy search is illustrated in Figure~\ref{fig:dil_er}.
\label{sec:ch4_bgc}
\begin{figure}[!h]
\begin{minipage}[t]{1\textwidth}
\centering
 \subfigure{%
 \includegraphics[width=0.16\textwidth]{images/gt_spx/marple4_363.jpg}}
\hfill
 \subfigure{%
 \includegraphics[width=0.16\textwidth]{images/gt_spx/marple4_363_2.jpg}}
\hfill
\subfigure{%
 \includegraphics[width=0.16\textwidth]{images/gt_spx/marple4_363_3.jpg}}
\hfill
\subfigure{%
 \includegraphics[width=0.16\textwidth]{images/gt_spx/marple4_363_4.jpg}}
\hfill
\subfigure{%
 \includegraphics[width=0.16\textwidth]{images/gt_spx/marple4_363_5.jpg}}
\hfill
\subfigure{%
 \includegraphics[width=0.16\textwidth]{images/gt_spx/marple4_363_6.jpg}}
\begin{picture}(1000,2)
\put(0,0){\vector(1,0){420}}
\end{picture}
\footnotesize (a) Dilation
\end{minipage}
\begin{minipage}[t]{1\textwidth}
\centering
 \subfigure{%
 \includegraphics[width=0.16\textwidth]{images/gt_spx/marple4_363.jpg}}
\hfill
 \subfigure{%
 \includegraphics[width=0.16\textwidth]{images/gt_spx/marple4_363_2.jpg}}
\hfill
\subfigure{%
 \includegraphics[width=0.16\textwidth]{images/gt_spx/marple4_363_7.jpg}}
\hfill
\subfigure{%
 \includegraphics[width=0.16\textwidth]{images/gt_spx/marple4_363_8.jpg}}
\hfill
\subfigure{%
 \includegraphics[width=0.16\textwidth]{images/gt_spx/marple4_363_9.jpg}}
\hfill
\subfigure{%
 \includegraphics[width=0.16\textwidth]{images/gt_spx/marple4_363_10.jpg}}
\begin{picture}(1000,2)
\put(0,0){\vector(1,0){420}}
\end{picture}
\footnotesize (a) Erosion
\end{minipage}
% 
% 
% \begin{minipage}[t]{1\textwidth}
% \centering
% \subfigure{%
% \includegraphics[width=0.15\textwidth]{images/gt_spx/marple4_363_8.jpg}}
% \subfigure{%
% \includegraphics[width=0.15\textwidth]{images/gt_spx/marple4_363_7.jpg}}
% \subfigure{%
% \includegraphics[width=0.15\textwidth]{images/gt_spx/marple4_363_2.jpg}}
% \subfigure{%
% \includegraphics[width=0.15\textwidth]{images/gt_spx/marple4_363.jpg}}
% \subfigure{%
% \includegraphics[width=0.15\textwidth]{images/gt_spx/marple4_363_2.jpg}}
% \subfigure{%
% \includegraphics[width=0.15\textwidth]{images/gt_spx/marple4_363_3.jpg}}
% \subfigure{
% \includegraphics[width=0.15\textwidth]{images/gt_spx/marple4_363_4.jpg}}
% 
% \footnotesize (a) Marple4
% \end{minipage}
 \caption[Basic scheme for a greedy search of the optimal partition]{
  {\bf Basic scheme for a greedy search of the optimal partition}. We find the best partition according to the balanced graph cut criteria by adding to (dilation) or subtracting (erosion) superpixels from foreground cluster.}
\label{fig:dil_er}
\end{figure}
\subsection{Analysis of Balanced Graph Cut Criteria}
In these experiments we consider the following cut criteria as our objectives: NCut, RCut, NCC, RCC, MinMaxCut and Cut (see sec.~\ref{sec:ch2_balgrcut}). For each of them we obtain the unique path by the greedy search described above.

Figure~\ref{fig:C6_cut} and~\ref{fig:M4_cut} illustrate the results for the video sequences ''Cars6`` and ''Marple4`` respectively. 
In the plots the x-axis represents the size in superpixels of dilative (positive) or erosive (negative) area. The zero corresponds to the starting point - the ground truth.
And the value of the cut criteria is represented on the y-axis in logarithmic scale.
\begin{figure}[!ht]
%\begin{minipage}[t]{0.6\linewidth}
\centering
\subfigure[Greedy search for optimal value of the graph cut ]{%
\includegraphics[width=0.425\textwidth]{images/C6_2cl/C6_2cl.png}}
\quad
%\end{minipage}
%\begin{minipage}[t]{0.4\linewidth}
%\centering
\subfigure[Cars6]{%
\includegraphics[width=0.196\textwidth]{images/C6_2cl/cars6_003.jpg}} 
\quad
\subfigure[Ground truth]{%
\includegraphics[width=0.19\textwidth]{images/C6_2cl/ground_truth.png}} 
\quad
\newline
%\end{minipage}
%\begin{minipage}[t]{1\textwidth}
\subfigure[Global minimum for all]{%
\includegraphics[width=0.19\textwidth]{images/C6_2cl/Global_minimum(all).png}} 
\quad
\subfigure[Local minimum for RCC, NCut, NCC, Cut, MinMaxCut]{%
\includegraphics[width=0.19\textwidth]{images/C6_2cl/Local_minimum(cut,ncc,rcc,ncut,nof2,minmaxcut).png}} 
\quad
\subfigure[Local minimum for RCut]{%
\includegraphics[width=0.19\textwidth]{images/C6_2cl/Local_minimum(rcut,rof1,rof2,nof1).png}} 
%\end{minipage}
\caption[Graph cut results for the video sequence ``Cars6`` for 2 clusters]{
{\bf Graph cut results for the video sequence ``Cars6`` for 2 clusters.}}
\label{fig:C6_cut}
\end{figure}
\begin{figure}[!ht]
%\begin{minipage}[t]{0.6\linewidth}
\centering
\quad
\subfigure[Greedy search for optimal value of the graph cut ]{%
\includegraphics[width=0.425\textwidth]{images/M4_2cl/M4_2cl.png}}
\quad 
%\end{minipage}
%\begin{minipage}[t]{0.4\linewidth}
%\centering
\subfigure[Marple4]{%
\includegraphics[width=0.18\textwidth]{images/M4_2cl/marple4_363.jpg}} 
\quad
\subfigure[Ground truth]{%
\includegraphics[width=0.19\textwidth]{images/M4_2cl/ground_truth.png}} 
\quad
\newline
%\end{minipage}
%\begin{minipage}[t]{1\textwidth}
\subfigure[Global minimum for RCC, NCC]{%
\includegraphics[width=0.19\textwidth]{images/M4_2cl/Global_minimum(rcc,ncc).png}} 
\quad
\subfigure[Global minimum for RCut, NCut, MinMaxCut, Cut]{%
\includegraphics[width=0.19\textwidth]{images/M4_2cl/Global_minimum(rcut,ncut,minmaxcut,cut,rof1,rof2,nof1,nof2).png}} 
\quad
\subfigure[Local minimum for RCC, NCC]{%
\includegraphics[width=0.19\textwidth]{images/M4_2cl/Local_minimum(ncc,rcc).png}} 
\quad
\subfigure[Local minimum for RCut, NCut, MinMaxCut, Cut]{%
\includegraphics[width=0.19\textwidth]{images/M4_2cl/Local_minimum(rcut,ncut,minmaxcut,cut,rof1,rof2,nof1,nof2).png}} 
%\end{minipage}
\caption[Graph cut results for the video sequence ``Marple4`` for 2 clusters]{
{\bf Graph cut results for the video sequence ``Marple4`` for 2 clusters.}}
\label{fig:M4_cut}
\end{figure}

The results for the sequence ''Cars6`` portray the ideal case. The cut value has a prominent minimum and the minimizer is close to the ground truth. Therefore all the balanced graph cuts have the same behaviour.
They coincide in the global minimum, which partition represents the moving car and the background, and with exception of the RCut in the local minimum, which minimizer is 2 clusters, divided along the horizon.
Here all the balanced graph cut objectives perform equally good. The affinity matrix provides powerful representation of within- and between-frame similarities of superpixels, which is not surprising since
the sequence has a strong translational motion and colour difference.

However, this is not always the case. For the video sequence ''Marple4`` the global and local minima of the cut value are close to each other and the minimizer is far from the ground truth.
So the better solution is given by the local minimizer. This video sequence is more challenging, the motion
is slow here and also such difficulties as occlusion and illumination change are present.
Here the balancing term has a stronger impact and the behaviour of graph cut criteria is also different. The solution closest to the ground truth is
achieved by the Cheeger cuts due to its more precise balancing term, while all other graph cut objectives get stuck in the global minimum of the cut value. 
\subsection{Analysis of the Convergence of Methods to Global Optimum}
For the next group of experiments the goal was to analyze the output solution of the algorithms in terms of the graph cut objective functions: NCut and RCut.
The idea is to see whether the methods converge to the optimal solution or we could find a better partitions in the sense of minimum cut criteria by a trivial greedy search.

We restrict ourselves to the video sequence "Marple4'', given that ``Cars6'' is not challenging enough and will lead to obvious results.
In these experiments the final segmentations of different methods as well as the ground truth are used as a starting point.
Here we consider the output of 1-spectral clustering with the NCut and RCut objectives, normalized spectral clustering with Laplacian eigenmap constructed on the second eigenvector and
six eigenvectors. The greedy search is carried out in the same manner as before.

Figure~\ref{fig:NCut_2} and~\ref{fig:RCut_2} report the results for two clusters for the NCut and RCut objectives respectively. In the plots the circle indicates
the starting point for each method with respect to the ground truth, which is zero. 
The x-axis represents for the NCut the volume  and for the RCut the size of dilative or erosive area. 
\begin{figure}[htbp]
\begin{minipage}[t]{0.5\linewidth}
\centering
\subfigure[Greedy search for the optimal NCut with the segmentation for 2 clusters as a starting point]{%
\includegraphics[width=\textwidth]{images/gt_M4_2cl/M4_NcutT_2cl.png}}

\end{minipage}
\begin{minipage}[t]{0.5\linewidth}
\centering
\begin{minipage}[t]{1\textwidth}
 \centering
\hfill \hfill 
\footnotesize GT
\hfill \hfill \hfill 
\footnotesize 1SC
\hfill  \hfill 
\footnotesize SC, 6 dim
\hfill 
\footnotesize SC, 2 dim
\hfill 
\end{minipage}
\begin{minipage}[t]{1\textwidth}
\centering
\hfill \hfill   \hfill 
\subfigure{%
\includegraphics[width=0.23\textwidth]{images/gt_M4_2cl/ground_truth.png}} 
\hfill  
\subfigure{%
\includegraphics[width=0.23\textwidth]{images/gt_M4_2cl/1SC_glob_gt.png}} 
\hfill 
\subfigure{%
\includegraphics[width=0.23\textwidth]{images/gt_M4_2cl/SC_6d_gt.png}} 
\hfill  
\subfigure{%
\includegraphics[width=0.23\textwidth]{images/gt_M4_2cl/SC_2d_gt.png}} 

\footnotesize (b) Initialization
\end{minipage}
\begin{minipage}[t]{1\textwidth}
\centering
\hfill \hfill   \hfill 
\subfigure{%
\includegraphics[width=0.23\textwidth]{images/gt_M4_2cl/glob_gt.png}} 
\hfill 
\subfigure{%
\includegraphics[width=0.23\textwidth]{images/gt_M4_2cl/1SC_glob_gt.png}} 
\hfill 
\subfigure{%
\includegraphics[width=0.23\textwidth]{images/gt_M4_2cl/SC_6d_glob.png}} 
\hfill 
\subfigure{%
\includegraphics[width=0.23\textwidth]{images/gt_M4_2cl/SC_2d_glob.png}} 

\footnotesize (c) Global minimum
\end{minipage}
\begin{minipage}[t]{1\textwidth}
\centering
\hfill \hfill   \hfill 
\subfigure{%
\includegraphics[width=0.23\textwidth]{images/gt_M4_2cl/loc_gt.png}} 
\hfill 
\subfigure{%
\includegraphics[width=0.23\textwidth]{images/gt_M4_2cl/1SC_loc.png}} 
\hfill 
\subfigure{%
\includegraphics[width=0.23\textwidth]{images/gt_M4_2cl/SC_6d_loc.png}} 
\hfill 
\subfigure{%
\includegraphics[width=0.23\textwidth]{images/gt_M4_2cl/SC_2d_loc.png}} 

\footnotesize (d) Local minimum
\end{minipage}
\end{minipage}
\caption[NCut results for the video sequence ``Marple4`` for 2 clusters with the output segmentation as a starting point]{
{\bf NCut results for the video sequence ``Marple4`` for 2 clusters with the output segmentation as a starting point.}}
\label{fig:NCut_2}
\end{figure}
\newpage
\begin{figure}[htbp]
\begin{minipage}[t]{0.5\linewidth}
\centering
\subfigure[Greedy search for the optimal RCut with the segmentation for 2 clusters as a starting point]{%
\includegraphics[width=\textwidth]{images/gt_M4_2cl/M4_RcutT_2cl.png}}

\end{minipage}
\begin{minipage}[t]{0.5\linewidth}
\centering
\begin{minipage}[t]{1\textwidth}
 \centering
\hfill \hfill 
\footnotesize GT
\hfill \hfill \hfill 
\footnotesize 1SC
\hfill  \hfill 
\footnotesize SC, 6 dim
\hfill 
\footnotesize SC, 2 dim
\hfill 
\end{minipage}
\begin{minipage}[t]{1\textwidth}
\centering
\hfill \hfill   \hfill 
\subfigure{%
\includegraphics[width=0.23\textwidth]{images/gt_M4_2cl/ground_truth.png}} 
\hfill  
\subfigure{%
\includegraphics[width=0.23\textwidth]{images/gt_M4_2cl/1SC_glob_gt.png}} 
\hfill 
\subfigure{%
\includegraphics[width=0.23\textwidth]{images/gt_M4_2cl/SC_6d_gt.png}} 
\hfill  
\subfigure{%
\includegraphics[width=0.23\textwidth]{images/gt_M4_2cl/SC_2d_gt.png}} 

\footnotesize (b) Initialization
\end{minipage}
\begin{minipage}[t]{1\textwidth}
\centering
\hfill \hfill   \hfill 
\subfigure{%
\includegraphics[width=0.23\textwidth]{images/gt_M4_2cl/glob_gt.png}} 
\hfill 
\subfigure{%
\includegraphics[width=0.23\textwidth]{images/gt_M4_2cl/1SC_glob_gt.png}} 
\hfill 
\subfigure{%
\includegraphics[width=0.23\textwidth]{images/gt_M4_2cl/SC_6d_glob_rcut.png}} 
\hfill 
\subfigure{%
\includegraphics[width=0.23\textwidth]{images/gt_M4_2cl/SC_2d_glob_rcut.png}} 

\footnotesize (c) Global minimum
\end{minipage}
\begin{minipage}[t]{1\textwidth}
\centering
\hfill \hfill   \hfill 
\subfigure{%
\includegraphics[width=0.23\textwidth]{images/gt_M4_2cl/loc_gt.png}} 
\hfill 
\subfigure{%
\includegraphics[width=0.23\textwidth]{images/gt_M4_2cl/1SC_loc.png}} 
\hfill 
\subfigure{%
\includegraphics[width=0.23\textwidth]{images/gt_M4_2cl/SC_6d_loc_rcut.png}} 
\hfill 
\subfigure{%
\includegraphics[width=0.23\textwidth]{images/gt_M4_2cl/SC_2d_loc_rcut.png}} 

\footnotesize (d) Local minimum
\end{minipage}

\end{minipage}
\caption[RCut results for the video sequence ``Marple4`` for 2 clusters with the output segmentation as a starting point]{
{\bf RCut results for the video sequence ``Marple4`` for 2 clusters with the output segmentation as a starting point.}}
\label{fig:RCut_2}
\end{figure}

One can observe that the solution of the 1-spectral clustering coincides with the global optimum in terms of both the normalized and ratio cuts. Although the optimizer is far from the ground truth. Here the better solution
would be obtained by the local minimum.
The results for the spectral clustering with six eigenvectors is different depending on the graph cut criterion. The global minimum is achieved only in terms of the ratio cut. For the normalized cut
the output partition is just the local optimum and is far from both the ground truth and the global minimum.
A better result is obtained for the spectral relaxation with the second eigenvector. Although the method does not converge and get stuck in local minimum for both the NCut and the RCut, the minimizer is 
closer to the ground truth than for all other methods.


Figure~\ref{fig:NCut_3} and~\ref{fig:RCut_3} reports the same experiments for 3 clusters with the multipartition cut criteria NCut and RCut. 
\begin{figure}[htbp]
\begin{minipage}[t]{0.5\linewidth}
\centering
\subfigure[Greedy search for the optimal NCut with the segmentation for 3 clusters as a starting point]{%
\includegraphics[width=\textwidth]{images/gt_M4_3cl/M4_NcutT_3cl.png}}

\end{minipage}
\begin{minipage}[t]{0.5\linewidth}
\centering
\begin{minipage}[t]{1\textwidth}
 \centering
\hfill \hfill 
\footnotesize GT
\hfill \hfill \hfill 
\footnotesize 1SC
\hfill  \hfill 
\footnotesize SC, 6 dim
\hfill 
\footnotesize SC, 2 dim
\hfill 
\end{minipage}
\begin{minipage}[t]{1\textwidth}
\centering
\hfill \hfill   \hfill 
\subfigure{%
\includegraphics[width=0.23\textwidth]{images/gt_M4_3cl/gt.png}} 
\hfill  
\subfigure{%
\includegraphics[width=0.23\textwidth]{images/gt_M4_3cl/gt_ncut.png}} 
\hfill 
\subfigure{%
\includegraphics[width=0.23\textwidth]{images/gt_M4_3cl/gt_adhoc.png}} 
\hfill  
\subfigure{%
\includegraphics[width=0.23\textwidth]{images/gt_M4_3cl/gt_adhoc_2dim_not1.png}} 

\footnotesize (b) Initialization
\end{minipage}
\begin{minipage}[t]{1\textwidth}
\centering
\hfill \hfill   \hfill 
\subfigure{%
\includegraphics[width=0.23\textwidth]{images/gt_M4_3cl/glob_min_gt.png}} 
\hfill 
\subfigure{%
\includegraphics[width=0.23\textwidth]{images/gt_M4_3cl/global_min_ncut.png}} 
\hfill 
\subfigure{%
\includegraphics[width=0.23\textwidth]{images/gt_M4_3cl/glob_min_adhoc.png}} 
\hfill 
\subfigure{%
\includegraphics[width=0.23\textwidth]{images/gt_M4_3cl/glob_min_adhoc_2dim_not1.png}} 

\footnotesize (c) Global minimum
\end{minipage}
\begin{minipage}[t]{1\textwidth}
\centering
\hfill \hfill   \hfill 
\subfigure{%
\includegraphics[width=0.23\textwidth]{images/gt_M4_3cl/loc_min_gt.png}} 
\hfill 
\subfigure{%
\includegraphics[width=0.23\textwidth]{images/gt_M4_3cl/local_min_ncut.png}} 
\hfill 
\subfigure{%
\includegraphics[width=0.23\textwidth]{images/gt_M4_3cl/loc_min_adhoc.png}} 
\hfill 
\subfigure{%
\includegraphics[width=0.23\textwidth]{images/gt_M4_3cl/loc_min_adhoc_2dim_not1.png}} 

\footnotesize (d) Local minimum
\end{minipage}

\end{minipage}
\caption[NCut results for the video sequence ``Marple4`` for 3 clusters with the output segmentation as a starting point]{
{\bf NCut results for the video sequence ``Marple4`` for 3 clusters with the output segmentation as a starting point.}}
\label{fig:NCut_3}
\end{figure}

\begin{figure}[htbp]
\begin{minipage}[t]{0.5\linewidth}
\centering
\subfigure[Greedy search for the optimal RCut with the segmentation for 3 clusters as a starting point]{%
\includegraphics[width=\textwidth]{images/gt_M4_3cl/M4_RcutT_3cl.png}}

\end{minipage}
\begin{minipage}[t]{0.5\linewidth}
\centering
\begin{minipage}[t]{1\textwidth}
 \centering
\hfill \hfill 
\footnotesize GT
\hfill \hfill \hfill 
\footnotesize 1SC
\hfill  \hfill 
\footnotesize SC, 6 dim
\hfill 
\footnotesize SC, 2 dim
\hfill 
\end{minipage}
\begin{minipage}[t]{1\textwidth}
\centering
\hfill \hfill   \hfill 
\subfigure{%
\includegraphics[width=0.23\textwidth]{images/gt_M4_3cl/gt.png}} 
\hfill  
\subfigure{%
\includegraphics[width=0.23\textwidth]{images/gt_M4_3cl/gt_rcc_rand.png}} 
\hfill 
\subfigure{%
\includegraphics[width=0.23\textwidth]{images/gt_M4_3cl/gt_adhoc.png}} 
\hfill  
\subfigure{%
\includegraphics[width=0.23\textwidth]{images/gt_M4_3cl/gt_adhoc_2dim_not1.png}} 

\footnotesize (b) Initialization
\end{minipage}

\begin{minipage}[t]{1\textwidth}
\centering
\hfill \hfill   \hfill 
\subfigure{%
\includegraphics[width=0.23\textwidth]{images/gt_M4_3cl/glob_min_gt(rcut).png}} 
\hfill 
\subfigure{%
\includegraphics[width=0.23\textwidth]{images/gt_M4_3cl/global_min_all.png}} 
\hfill 
\subfigure{%
\includegraphics[width=0.23\textwidth]{images/gt_M4_3cl/glob_min_adhoc(rcut).png}} 
\hfill 
\subfigure{%
\includegraphics[width=0.23\textwidth]{images/gt_M4_3cl/glob_min_adhoc_2dim_not1(rcut).png}} 

\footnotesize (c) Global minimum
\end{minipage}

\begin{minipage}[t]{1\textwidth}
\centering
\hfill \hfill   \hfill 
\subfigure{%
\includegraphics[width=0.23\textwidth]{images/gt_M4_3cl/loc_min_gt(rcut).png}} 
\hfill 
\subfigure{%
\includegraphics[width=0.23\textwidth]{images/gt_M4_3cl/local_min_all.png}} 
\hfill 
\subfigure{%
\includegraphics[width=0.23\textwidth]{images/gt_M4_3cl/loc_adh_rcut.png}} 
\hfill 
\subfigure{%
\includegraphics[width=0.23\textwidth]{images/gt_M4_3cl/loc_min_adhoc_2dim_not1(rcut).png}} 

\footnotesize (d) Local minimum
\end{minipage}

\end{minipage}
\caption[RCut results for the video sequence ``Marple4`` for 3 clusters with the output segmentation as a starting point]{
{\bf RCut results for the video sequence ``Marple4`` for 3 clusters with the output segmentation as a starting point.}}
\label{fig:RCut_3}
\end{figure}

For this purpose we re-annotated the frames and distinguished the shrubbery as the third object.
It can be seen that the above observations are also true for 3 clusters. The only difference is that the solution for spectral clustering with 2 eigenvectors 
is closer to the global optimum.
\section{Discussion}
\label{ch4:disc}
The conducted experiments showed that 1-spectral clustering falls behind in the performance in comparison with the standard relaxation due to its main drawback - recursive splitting scheme for multipartitioning.
Unexpectedly, the ratio cuts yielded better result for the 1-norm relaxation than the normalized ones and even outperformed the standard 2-norm relaxation for a higher number of clusters. Also it was observed that
1-spectral clustering performs better on more balanced video sequences, especially for the RCut criterion.

Moreover, it was revealed that the human annotated segmentation does not give the optimal in the sense of balanced graph cut criteria solution 
and that could be another reason for the 2-norm relaxation outperforming the 1-norm relaxation.
Nevertheless, the 1-spectral clustering method showed great potential and given the feasibility of computation of higher-order eigenvectors one could assume high improvement in the performance.

It was observed that for more challenging video sequences with occlusions,
changes of illumination and contrast the ground truth partition might not coincide with the minimum value of the cut criterion. Furthermore, it was seen that the solution of algorithms does not always converge
to the global optimum and sometimes a better partition could be found by a trivial greedy search. 

It is worth mentioning that the balanced graph cut criteria have also drawbacks. All known balanced terms take into account only size of clusters and discard balancing over time. 
But in video segmentation temporal consistency of clusters is also required. Hence it would be useful to have spatio-temporal balancing in the objective function.
So far this is an open problem in video segmentation.
\clearpage

\chapter{Experiments}
\label{Chapter5}
\settocdepth{subsection}
The majority of our experiments focuses on weighting the watershed locations. That corresponds to the \textbf{Structured voting (SV)} stage of the pipeline SE-SV-UCM, which we propose in \cref{Chapter4} for the task of going from edges to contours.

\textbf{Dataset:} For the evaluation of our segmentation results we work on the Berkeley Segmentation Data Set (BSDS500)~\cite{Arbelaez11}. Since its introduction in 2001, it has by now become a standard dataset for both the task of edge detection as well as that of image segmentation.

\textbf{Benchmark:} We report results on the benchmark~\cite{Galasso13Benchmark} introduced in~\cite{Galasso13} which can evaluate segmentation hierarchies against given ground-truth segmentations. It demonstrates the tradeoff between an oversegmentation and a more accurate object-centric segmentation.

\textbf{Watershed weighting strategy:} The Structured voting requires a choice of a watershed weighting strategy. The purpose of the weighting is associating a \textbf{score} with each of the watershed locations pixels. That score must faithfully reflect the strength of the underlying boundary. So we want to evaluate how good is the boundary evidence presented by the most likely segmentation determined by the structured forest. 

The first aspect of our voting strategy is making the structured forest patch and the watershed locations patch comparable. The watershed patch is an oversegmentation, and in this sense, contains much more information, not exclusively about the location of the boundary that we would like to evaluate. So we strive to simplify the watershed patch, keeping only important information about it - the shape of the boundary under consideration, or the constitution of the segmentation in the patch. Such a simplification in the context of our algorithm we call {\bf watershed patch transformation}. %``watershed patch transformation''. 

The second particular to a watershed weighting strategy is the choice of a scoring function. We view the task as a segmentation benchmark problem, where one of the patches is the ground truth segmentation, and the other - the segmentation under test. We analyse and apply a selection of boundary- and region- based metrics.

In the rest of the chapter we briefly describe the dataset and evaluation metrics used, in order to help understand the experiments. Afterwards, we give a detailed account of our most important experiments and the conclusions we draw based on them.

\section{Evaluation setup}
\subsection{Dataset}
\label{sec:ch5-BSDS500-dataset}
The Berkeley Segmentation Data Set (BSDS), introduced in~\cite{Martin01}, is a large dataset of natural images that have been manually segmented by multiple participants. It, therefore, provides the ground truth label for each pixel as being on- or off-boundary. Initially the dataset featured 300 images (BSDS300). It was later extended - in the new dataset BSDS500~\cite{Arbelaez11} the original 300 images are used for training (200) and validation (100), and 200 new human-annotated images are added for testing. Again, each image is segmented by different subjects. See \fref{fig:BSDS-annotations} for an example image and two of its annotations by different people.

\begin{figure}[ht!]
\begin{center}
  \begin{tabular}{ c c c }
  \includegraphics[width=0.3\textwidth]{images/examples/starfish/starfish.png} &
  \includegraphics[width=0.3\textwidth,frame]{images/examples/starfish/starfish_bdry_coarse.png} &
  \includegraphics[width=0.3\textwidth]{images/examples/starfish/starfish_segm_coarse.png} \\
  &
  \includegraphics[width=0.3\textwidth,frame]{images/examples/starfish/starfish_bdry_detail.png} &
  \includegraphics[width=0.3\textwidth]{images/examples/starfish/starfish_segm_detail.png} \\
  Input image & Boundaries & Segmentation \\
  \end{tabular}
\end{center}
\caption[BSDS500 dataset - 2 annotations]{Image from the validation subset of~\cite{BSDS500resources} and two of its annotations - subject~1 on the first row, subject~2 on the second row. {\bf Human-marked boundaries} are the central column, and their corresponding {\bf segmentation} reconstructions are given next to them.}
\label{fig:BSDS-annotations}
\end{figure}

\subsection{Metrics}
The benchmark that we use provides, among others, two precision-recall metrics - a boundary and a region oriented one.

\subsubsection{BPR}
The Boundary Precision-Recall (BPR)~\cite{Arbelaez11} is a boundary-based metric and emphasises the correct placement of image edges. Section~\ref*{sec:ch4-boundary-and-region-metrics-maths}~\ref{par:ch4-BPR-maths} % avoid having both links by using ref*
gives mathematical account on the metric and its properties. In case of segmentation, BPR is a good indicator of the {\bf localisation of the region boundaries}.

\textbf{Impact of %small 
local change in the boundary on the score:} A difference in the score of a single region boundary pixel should not greatly affect the edge detector output. Therefore, it correctly has only a small impact on the BPR metric. For the task of image segmentation however, a change in the strength of a single pixel could result in merging neighbouring regions. A ``weaker'' pixel among a strong intervening boundary could be thought as a leakage, which will cause the UCM algorithm (featured in \sref{sec:ch3-UCM}) % TODO give the algorithm in an appendix
to merge the two regions. In the hierarchical image segmentation framework, that means multiple levels of the segmentations hierarchy would change. So a rigorous image segmentation benchmark metric should not be oblivious to such changes.

\subsubsection{VPR}
To address the above issue, the other metric that we report is the Volume Precision-Recall (VPR) introduced by Galasso \etal~\cite{Galasso13} to evaluate the accuracy of video segmentation algorithms. For images (or video still frames) the metric is a region-based metric, operating % applied
into a precision-recall framework. It is measuring the size of the regions and the overlap between the ground truth segmentation regions and the segmentation regions produced by the algorithm under test. 
See Section~\ref*{sec:ch4-boundary-and-region-metrics-maths}~\ref{par:ch4-VPR-maths} for the formulae and discussion on the need for normalisation when evaluating segmentations.

% \section{Weighting strategies} % Exploration of the Space of Weighting Strategies}
% \section{Oracle} %  - Experiments with Ground Truth}
% \subsection{Oracle definition} % description}
% \subsection{Ranking of oracles}
% % \subsubsection{Confirms Correct Weighting Strategies}
% % \subsubsection{Failure Cases}

\subsubsection{Reported numbers}
\paragraph{For the precision-recall metrics BPR and VPR}\mbox{}\\\mbox{}\\
Both BPR and VPR are able to evaluate {\bf individual segmentations} (which on the plots are depicted as a single dot - the model error), as well as {\bf segmentation hierarchies}, represented using the UCM data structure (which on the plots constitute a curve). As stated in \sref{sec:ch4-boundary-and-region-metrics-maths}, for a single segmentation instance, the harmonic mean of the precision and recall, called the {\it F measure}, is reported. $F=\frac{2PR}{P+R}$.

In case of a probability of boundary, or a hierarchy of segmentations, which is in fact the case in the majority of our experiments, globally optimal scores %- best over the whole hierarchy, 
are reported. The scores are 3 in total: a best F-score (according to two criteria), as well as Average precision (\textbf{AP}). 

Optimal dataset score ({\bf ODS}) is the highest F score achieved while having a fixed scale for all images in the test set. 
Optimal image scale ({\bf OIS}) for BPR, or Optimal segmentation scale (\textbf{OSS}) for VPR is the average of the best F scores when allowing optimal scale {\it per image}. Hence, OIS\slash OSS is no lower than ODS. 
{\bf AP} is the precision averaged on the recall range $R\in[0,1]$, or, alternatively, the area under the precision-recall curve (AUC).

% TODO check if the following statement about ROC is indeed true
Note that in the case of edge detection benchmarked with BPR, P is not a function of R. Both values are {\bf functions of the segmentation threshold}. Thus it is possible to have different precision and same recall - on different threshold of detail, \ie different locations along the curve.
Not to be confused with %Compare with 
the receiver operating characteristic (ROC), used in statistics for comparing true-positive rate (TPR) against false-positive rate (FPR) of a binary classifier at various thresholds. In the ROC curve the TPR %sensitivity 
is a function of the FPR. %fall-out

\paragraph{Further region metrics}\mbox{}\\\mbox{}\\
Beside the aggregate measures for BPR and VPR, for the segmentation algorithms we also report the following region statistics. %(SC, PRI, VI)
\subparagraph{Segmentation covering of ground truth (SC):} This metric is an estimation of the best covering of the ground truth by the machine segments. We reviewed the metric in Section~\ref*{sec:ch4-boundary-and-region-metrics-maths}~\ref{par:ch4-SC-maths}.

\subparagraph{Probabilistic Rand index (PRI):} The PRI~\cite{UnnikrishnanPH07} is an extension of Rand index (RI)~\cite{rand1971objective}. It allows to assess the consistency of labelling of pixel pairs between the segmentation algorithm under test on one hand, and {\it multiple} ground truth segmentations on the other hand. Further, PRI partially addresses the issue of small dynamic range that RI displays. See Section~\ref*{sec:ch4-boundary-and-region-metrics-maths}~\ref{par:ch4-PRI-maths} for the formulae.

\subparagraph{Variation of information (VoI):} The Variation of information (VoI), introduced in~\cite{Meila05} measures the distance between two clusters of data - in our case, the human annotated and the machine-generated segmentation. The distance is measured \wrt %, in terms of 
their average conditional entropy and mutual information. This is the only metric that we report, which has preference for lower score, 0 being the theoretical best in case of equivalent segmentations.

\textbf{General preference:} For all metrics but VoI the general preference is ``higher is better''.

\paragraph{Example comparison of SE and gPb-OWT-UCM}\mbox{}\\\mbox{}\\
\tref{tab:SE_vs_gPb_OWT_UCM} has all the scores we just described, and \fref{fig:SE_vs_gPb_OWT_UCM} - the BPR and VPR plots for the two methods that we previously dissected - Structured edge~\cite{DollarICCV13edges} in \cref{Chapter2}, and gPb-OWT-UCM~\cite{Arbelaez11} in \cref{Chapter3}. 

Note that as SE is an edge detection algorithm, none of the region metrics or the VPR are applicable to its output. Since gPb-OWT-UCM is an image segmentation algorithm, the boundaries of the segments in a segmentation constitute closed contours, so boundary-based metrics, such as BPR, are applicable.

\begin{figure}[ht!]
\centering
 \subfigure[BPR]{%
  \includegraphics[trim=1.5cm 0cm 1.9cm 0cm, clip=true, width=0.48\textwidth]{images/plots/SE_vs_gPb_OWT_UCM_BPR.png}
 }
 \subfigure[VPR]{%
  \includegraphics[trim=1.5cm 0cm 1.9cm 0cm, clip=true, width=0.48\textwidth]{images/plots/SE_vs_gPb_OWT_UCM_VPR.png}
 }
\caption[SE and gPb-OWT-UCM plots]{We demonstrate the boundary precision recall metric (BPR) and the volume precision recall metric (VPR). Given are the edge detection algorithm that we utilise {\bf Structured edge}~\cite{DollarICCV13edges}, and the image segmentation algorithm {\bf gPb-OWT-UCM}~\cite{Arbelaez11}.}
\label{fig:SE_vs_gPb_OWT_UCM}
\end{figure}

\begin{table}[htbp]
\renewcommand{\arraystretch}{1.3}
\centering
\scriptsize
\begin{tabular}{l|c|c|c||c|c|c||c|c|c|}
\cline{2-10} % ZZ
\multirow{2}{*}{} & \multicolumn{3}{c||}{\textbf{BPR}} & \multicolumn{3}{c||}{\textbf{VPR}}& \multicolumn{3}{c|}{\textbf{Region}}\\
\cline{2-10}
& \textbf{ODS}  & \textbf{OIS} & \textbf{AP} % <- BPR
& \textbf{ODS} & \textbf{OSS} & \textbf{AP} % <- VPR
& \textbf{SC} & \textbf{PRI} & \textbf{VoI} \\
\hline
\multicolumn{1}{|c|}{Human} & .79 & .79 & - & - & - & - & .72 & .88 & 1.17 \\ % actually, we had .80 for humans on BPR from \cite{Arbelaez11}; % TODO for VPR - we don't know ODS = OIS for humans
\hline
\hline
\multicolumn{1}{|c|}{\cite{DollarICCV13edges} Structured edge (SE)} & .70 & .72 & .63 & - & - & - & - & - & - \\
\hline
\multicolumn{1}{|c|}{\cite{Arbelaez11} gPb-OWT-UCM} & .73 & .76 & .77 & .73 & .76 & .78 & .59 & .83 & 1.69 \\
\hline
\end{tabular}
\caption[SE and gPb-OWT-UCM boundary and region comparison]{SE and gPb-OWT-UCM boundary and region comparison.}
%The table shows aggregate measures (ODS, OSS, AP) for boundary precision-recall (BPR), volume precision-recall (VPR) and 
%includes region statistics (SC, PRI, VoI).}
\label{tab:SE_vs_gPb_OWT_UCM}
\end{table}

\subsubsection{Benchmark}
We use the benchmark MATLAB code from~\cite{Galasso13Benchmark}. %, the metric was introduced in this work~\cite{Galasso13}.
It unifies benchmarks for boundary detection (BPR) and image segmentation (VPR, SC, %ground truth segmentation covering, 
PRI, VoI) and allows the testing of coarse-to-fine methods. %, capturing the tradeoff in a precision-recall framework.

\section{From edges to contours - a proof of concept}
We apply a {\bf vanilla watershed} algorithm~\cite{beucher1992morphological,najman1996geodesic,PINKlibrary} to the SE output, as described in \sref{sec:ch3-watershed}. The result is a single segmentation. 

In our benchmark plots (\fref{fig:SE-watershed}) the outcome of the experiment is not a Precision-Recall curve, but a single point, %dot, 
indicative of the model error. Since the watershed transformation provides an oversegmentation of the image, the dot is located in the high-recall, low-precision range on the BPR plot (lower right). In contrast, oversegmentations occupy the low-recall, high-precision part of the VPR domain (upper left). 

\begin{figure}[ht!]
\centering
 \subfigure[BPR]{%
  \includegraphics[trim=1.5cm 0cm 1.9cm 0cm, clip=true, width=0.48\textwidth]{images/plots/SE-watershed_BPR.png}
 }
 \subfigure[VPR]{%
  \includegraphics[trim=1.5cm 0cm 1.9cm 0cm, clip=true, width=0.48\textwidth]{images/plots/SE-watershed_VPR.png}
 }
\caption[SE-watershed and baseline: SE-UCM plots]{{\bf SE-watershed} provides a single oversegmentation. Our {\bf baseline - SE-UCM} is a UCM hierarchy built on top of the probability of boundary output of the SE edge detector.}
\label{fig:SE-watershed}
\end{figure}

\textbf{A point at issue % problem, trouble 
with non-maximum suppressed edges:} Note that the SE algorithm implements non-maximum suppression on the edge detection output to provide thinned edges. Non-maximum suppression is a method first introduced as a means of reducing thick edge responses to thin lines for the task of edge detection in greyscale images~\cite{rosenfeld1976digital}. Non-maximum suppression considers only the maxima in the gradient direction. As a consequence, the final output of the SE often has only single regional minimum. In the presence of a unique lake, the watershed is empty. To circumvent this problem, we use the SE detector \textit{before non-maxima suppression} as a topographic surface for the flooding.

\section{Baseline: SE-UCM}
As a baseline, we create a hierarchy of segmentations on top of the SE detector result. This is in the spirit of~\cite{Arbelaez2006boundary} who, however, use the edge detector of Martin, Fowlkes, and Malik (MFM)~\cite{martin2004learning}. What we do for this baseline could also be thought as our pipeline SE-SV-UCM without structured voting. Instead, the values from the probability of boundary, which is the output of the edge detector are directly transfered as values for the watershed pixels. Benchmark plots are on \fref{fig:SE-watershed}.

We observe the problem of strong edges ``bleeding'' into non-salient ones, despite lack of good local boundary evidence, as on the tikis examples (see \fref{fig:SE-UCM-tikis-bleeding-sub2}) between the heads of the middle and right statues. This issue is one of the motivations for the Structured voting (SV) described in \sref{sec:ch4-SE-SV-UCM_SV_details}.% Similarly

\begin{figure}[ht!]
\centering
\subfigure[Input image]{%
 \includegraphics[width=0.3\textwidth]{images/examples/tikis/tikis.jpg}
 \label{fig:SE-UCM-tikis-bleeding-sub1}
}
\subfigure[UCM]{%
 \includegraphics[width=0.3\textwidth,frame]{images/examples/tikis/SE-UCM-tikis-ucm-problem.png}
 \label{fig:SE-UCM-tikis-bleeding-sub2}
}
\caption[SE-UCM drawback - ``bleeding'' of strong edges towards unimportant ones]{{\bf SE-UCM} result. \protect\subref{fig:SE-UCM-tikis-bleeding-sub1} - an image from the validation subset of~\cite{BSDS500resources}. Notice how in the SE-UCM output \protect\subref{fig:SE-UCM-tikis-bleeding-sub2} unimportant horizontal edges between the statues' heads are {\it incorrectly %wrongly 
up-voted} due to strong vertical boundary in their vicinity (the outline of the statues).}
\label{fig:SE-UCM-tikis-bleeding}
\end{figure}
% BPR edge detector MFM 0.65, MFM-UCM 0.67 ; SE 0.70, % SE_no_nms_single_scale_repeat
% SE-UCM 0.69 (0.70)

\section{SE+sPb-UCM}
This experiment shows us that globalisation could easily be introduced to our method. Here we use the same affinity matrix as the spectral Pb of Arbel\'aez \etal~\cite{Arbelaez11}. Extending our algorithm to adopt a globalisation step could be beneficial, since it could pick up on improvements in the realm of spectral clustering, as for example spectral reduction~\cite{Galasso14}. % check the plots, check MCG paper - they did exactly this

\begin{figure}[ht!]
\centering
 \subfigure[BPR]{%
  \includegraphics[trim=1.5cm 0cm 1.9cm 0cm, clip=true, width=0.48\textwidth]{images/plots/SE_nnms_sPb-UCM_BPR.png}
 }
 \subfigure[VPR]{%
  \includegraphics[trim=1.5cm 0cm 1.9cm 0cm, clip=true, width=0.48\textwidth]{images/plots/SE_nnms_sPb-UCM_VPR.png}
 }
\caption[(SE and spectralPb)-UCM plots]{SE+sPb-UCM.}
\label{fig:SE_nnms_sPb-UCM}
\end{figure}

% TODO here is a table of all results we will show. What to do with it, can't just dump all the data without proper analysis / conclusion?
\begin{table}[htbp]
\renewcommand{\arraystretch}{1.3}
\centering
\scriptsize
\begin{tabular}{l|c|c|c||c|c|c||c|c|c|}
\cline{2-10} % ZZ
\multirow{2}{*}{} & \multicolumn{3}{c||}{\textbf{BPR}} & \multicolumn{3}{c||}{\textbf{VPR}}& \multicolumn{3}{c|}{\textbf{Region}}\\
\cline{2-10}
& \textbf{ODS}  & \textbf{OIS} & \textbf{AP} % <- BPR
& \textbf{ODS} & \textbf{OSS} & \textbf{AP} % <- VPR
& \textbf{SC} & \textbf{PRI} & \textbf{VoI} \\
\hline
\multicolumn{1}{|l|}{Human} & .79 & .79 & - & - & - & - & .72 & .88 & 1.17 \\ % actually, we had .80 for humans on BPR from \cite{Arbelaez11}; % TODO for VPR - we don't know ODS = OIS for humans
\hline
\hline
%%%%%%%%%%%%%%%%%%%%%%%%%%%%%%%%%%%%%%%%%%%%%
% SoA. More, perhaps for here: Sketch tokens, SCG, DeepNet, 
\multicolumn{1}{|l|}{\cite{Felzenszwalb04} Felz-Hutt [2004]} & .61 & .64 & .56 & - & - & - & .52 & .80 & 2.21 \\ % from \cite{Arbelaez11}
\hline
\multicolumn{1}{|l|}{\cite{Arbelaez11} gPb [2011]} & .71 & .74 & .65 & - & - & - & - & - & - \\ % from cite{Hallman2014}
\hline
\multicolumn{1}{|l|}{\cite{Arbelaez11} gPb-OWT-UCM [2011]} & .73 & .76 & .73 & .73 & .76 & .78 & .59 & .83 & 1.69 \\ % from \cite{Arbelaez11} % our measured AP was .77
\hline
\multicolumn{1}{|l|}{\cite{DollarICCV13edges} SE-multi [2013]} & .74 & .76 & .78 & - & - & - & - & - & - \\ % from cite{Hallman2014}
\hline
\multicolumn{1}{|l|}{\cite{Dollar2015PAMI} SE-sharp [2015]} & {\bf .75} & .77 & .80 & - & - & - & - & - & - \\ % from cite{Hallman2014}
\hline
\multicolumn{1}{|l|}{\cite{Arbelaez2014multiscale} MCG [2014]} & {\bf .75} & {\bf .78} & .76 & - & - & - & - & - & - \\ % from cite{Hallman2014}
\hline
\multicolumn{1}{|l|}{\cite{Ganin2014n4fields} {$N^4$}-{F}ields [2014]} & {\bf .75} & .77 & .78 & - & - & - & - & - & - \\ % their output, our benchmark
\hline
\multicolumn{1}{|l|}{\cite{Isola2014crisp} Crisp (PMI) [2014]} & .74 & .77 & .78 & - & - & - & - & - & - \\ % from cite{Hallman2014}
\hline
\multicolumn{1}{|l|}{\cite{Hallman2014} OEF [2014]} & {\bf .75} & .77 & {\bf .82} & - & - & - & - & - & - \\ % from cite{Hallman2014}
%%%%%%%%%%%%%%%%%%%%%%%%%%%%%%%%%%%%%%%%%%%%%
\hline
\hline
\multicolumn{1}{|l|}{\cite{DollarICCV13edges} SE-single [2013]} & .70 & .72 & .63 & - & - & - & - & - & - \\
\hline
\multicolumn{1}{|l|}{SE-watershed} & .39 & .39 & - & .34 & .34 & - & .20 & .75 & 6.26 \\
\hline
\multicolumn{1}{|l|}{SE-UCM (baseline)} & .69 &.73 & .75 & .72 & .75 & .77 & .58 & .82 & 1.80 \\
\hline
\multicolumn{1}{|l|}{(SE+sPb)-UCM} & .72 & .75 & .76 & .73 & .76 & .78 & .59 & .82 & 1.68 \\ % SE_nnms_sPb-UCM
\hline % watershed arc
\hline
\multicolumn{1}{|l|}{fitted line} &  .67 & .69 & .70 & .69 & .72 & .71 & .55 & .80 & 1.88 \\ % line fitting (ends) BPR 3
\hline
\multicolumn{1}{|l|}{watershed arc} & .63 & .65 & .60 & .66 & .67 & .62 & .51 & .79 & 2.13 \\ % watershed arc (a.k.a. contour) BPR 3
% line fitting
\hline
\hline % quadratic fitting:
\multicolumn{1}{|l|}{linear LLS} & .68 & .71 & .71 & .69 & .71 & .71 & .55 & .81 & 1.90 \\% linear model - linear least squares fitting}
\hline
\multicolumn{1}{|l|}{quadratic LLS} & .54 & .56 & .40 & .40 & .40 & .25 & .37 & .55 & 2.45 \\
\hline
\hline
\end{tabular}
\caption[Benchmark scores for SoA and ours]{State-of-the-art edge detection and segmentation methods, and all our experiments - benchmark scores on BSDS500~\cite{BSDS500resources}.}
%The table shows aggregate measures (ODS, OSS, AP) for boundary precision-recall (BPR), volume precision-recall (VPR) and 
%includes region statistics (SC, PRI, VoI).}
\label{tab:all-results}
\end{table}

% quadratic
% Boundary PR global
%    G-ODS: F( R 0.57, P 0.52 ) = 0.54   [th = 0.12]
%    G-OIS: F( R 0.57, P 0.54 ) = 0.56
%    Area_PR = 0.40
% Volume PR global
%    G-ODS: F( R 0.79, P 0.27 ) = 0.40   [th = 0.02]            <- NOTE: very low threshold! can't recall all boundaries, that explains the plot
%    G-OSS: F( R 0.79, P 0.26 ) = 0.40
%    G-Area_PR = 0.25
% Region
%    GT covering: ODS = 0.34 [th = 0.21]. OSS = 0.35. Best = 0.37
% Region
%    Rand Index: ODS = 0.55 [th = 0.02]. OSS = 0.55.
%    Var. Info.: ODS = 2.45 [th = 0.73]. OSS = 2.37.


% SE-UCM baseline
% Boundary PR global
%    G-ODS: F( R 0.70, P 0.69 ) = 0.69   [th = 0.27]
%    G-OIS: F( R 0.73, P 0.73 ) = 0.73
%    Area_PR = 0.75
% Volume PR global
%    G-ODS: F( R 0.71, P 0.73 ) = 0.72   [th = 0.25]
%    G-OSS: F( R 0.73, P 0.77 ) = 0.75
%    G-Area_PR = 0.77
% Region
%    GT covering: ODS = 0.58 [th = 0.29]. OSS = 0.64. Best = 0.73
% Region
%    Rand Index: ODS = 0.82 [th = 0.17]. OSS = 0.86.
%    Var. Info.: ODS = 1.80 [th = 0.52]. OSS = 1.57.

% SE_nnms_sPb-UCM
% Boundary PR global
%    G-ODS: F( R 0.72, P 0.72 ) = 0.72   [th = 0.08]
%    G-OIS: F( R 0.74, P 0.76 ) = 0.75
%    Area_PR = 0.76
% Volume PR global
%    G-ODS: F( R 0.70, P 0.76 ) = 0.73   [th = 0.10]
%    G-OSS: F( R 0.74, P 0.77 ) = 0.76
%    G-Area_PR = 0.78
% Region
%    GT covering: ODS = 0.59 [th = 0.12]. OSS = 0.64. Best = 0.74
% Region
%    Rand Index: ODS = 0.82 [th = 0.08]. OSS = 0.85.
%    Var. Info.: ODS = 1.68 [th = 0.15]. OSS = 1.48.

% linear LLS
% Boundary PR global
%    G-ODS: F( R 0.69, P 0.67 ) = 0.68   [th = 0.35]
%    G-OIS: F( R 0.74, P 0.67 ) = 0.71
%    Area_PR = 0.71
% Volume PR global
%    G-ODS: F( R 0.68, P 0.70 ) = 0.69   [th = 0.27]
%    G-OSS: F( R 0.70, P 0.73 ) = 0.71
%    G-Area_PR = 0.71
% Region
%    GT covering: ODS = 0.55 [th = 0.37]. OSS = 0.60. Best = 0.67
% Region
%    Rand Index: ODS = 0.81 [th = 0.17]. OSS = 0.83.
%    Var. Info.: ODS = 1.90 [th = 0.67]. OSS = 1.72.

% quadratic LLS
% Boundary PR global
%    G-ODS: F( R 0.57, P 0.52 ) = 0.54   [th = 0.12]
%    G-OIS: F( R 0.57, P 0.54 ) = 0.56
%    Area_PR = 0.40
% Volume PR global
%    G-ODS: F( R 0.79, P 0.27 ) = 0.40   [th = 0.02]
%    G-OSS: F( R 0.79, P 0.26 ) = 0.40
%    G-Area_PR = 0.25
% Region
%    GT covering: ODS = 0.34 [th = 0.21]. OSS = 0.35. Best = 0.37


\section[Structured voting]{Structured voting - experimental study of watershed weighting strategies}
\label{sec:ch5-structured-voting}
All experiments presented here are instances of the general algorithm described in \sref{sec:ch4-SE-SV-UCM_SV_details}.

% \settocdepth{section}
% \settocdepth{subsection}
\settocdepth{section}
\subsection{Superpixels and Rand index}
We leave the watershed patch to be an oversegmentation, which in fact it is. The output of the watershed transform that we use has explicit the boundaries between segments, which would hinder a region-based metric. Therefore, we transform a watershed patch to have implicit segment boundaries - the locations of transition between differently labelled segments. 
The patches from the decision forest already constitute segmentation labelling with implicit segment boundaries. That, of course, is due to the fact that the structured forest patches are taken unmodified from the ground truth segmentations of the training subset of BSDS500, which has a ``labelling with implicit segment boundaries'' format. We conduct the comparison between watershed and a tree leaf patch using as a scoring function:

\begin{itemize}
 \item{\bf Rand Index (RI):} a count of the number of pairs of locations that belong to the same segment in both patches. For a $16\times16$ segmentation patch, that means $32 640$ pairs of locations.
 \item{\bf Rand Index Monte Carlo (RIMC):} ours randomised subsample version of RI, which takes only a fraction $\rho$ of the pairs of locations into consideration. We experience no reduction in performance \wrt RI for a fraction as small as $\rho\approx\frac{1}{128}$, \ie, 256 out of the $32 640$ possible pairs of locations in a $16 \times 16$ patch. This scoring function is inspired from the way features are subsampled to introduce randomness when training a decision tree in~\cite{DollarICCV13edges,Dollar2013toolbox}.
\end{itemize}

The above experiments (both having a result of $F=0.55$ on BPR) led us to two conclusions. First, we need to have a closer look into the properties of our \textbf{scoring functions}. Section~\ref{sec:ch4-boundary-and-region-metrics-maths} of the previous chapter gives mathematical formulae and a detailed explanation on the metrics we considered for this task. Second, a \textbf{simplification of the watershed patch} is desirable, due to the discrepancy %mismatch between 
in makeup %constitution
of watershed and decision tree patches.

\subsection{Na\"{\i}ve greedy merge of watershed patch}
We first address the second of our conclusions from the previous experiment. We merge segments in the watershed patch according to each of the $T$ leaf patches (where $T$ is the number of trees in the decision forest). Thus we end up with $T$ distinct ``merged'' watershed patches. This approach seems to be too greedy, however. The watershed patch eventually becomes overly adapted to the tree leaf patch it is being compared to. As a consequence, this watershed transformation is not discriminative enough.

\subsubsection{Fair greedy merge}
To remedy this shortcoming of the na\"{\i}ve greedy merge, we introduce what we call ``fairness'' in the greedy merge approach. As described previously, we cast votes only on the watershed locations. That means, the patches that we consider contain a potential boundary location at their central pixel. It is the strength of this boundary that we strive to quantify. We enforce the greedy merge to respect a boundary-at-centre-location condition by preventing excessive merge of the segments around the central pixel of the patch.
% TODO image of the patches

\fref{fig:segs-to-greedy-merge-RIMC} shows the improvement we get over watershed oversegmentation with the last two experiments.

\begin{figure}[ht!]
\centering
 \subfigure[BPR]{%
  \includegraphics[trim=1.5cm 0cm 1.9cm 0cm, clip=true, width=0.48\textwidth]{images/plots/segs-to-greedy-merge-RIMC-BPR.png}
 }
 \subfigure[VPR]{%
  \includegraphics[trim=1.5cm 0cm 1.9cm 0cm, clip=true, width=0.48\textwidth]{images/plots/segs-to-greedy-merge-RIMC-VPR.png}
 }
\caption[Greedy merge experiments]{Greedy merge experiments. In all cases the scoring function used for patches comparison was the RIMC (Section~\ref*{sec:ch4-boundary-and-region-metrics-maths}~\ref{par:ch4-RIMC-maths} contains a description of the RIMC metric).} % (\hyperref[par:ch4-RIMC-maths]{RIMC description in Chapter 4}).} % that looks bad on paper, still useful for .pdf, as it contains the link
\label{fig:segs-to-greedy-merge-RIMC}
\end{figure}

\subsection{Watershed region boundary}
We would like our watershed weighting strategy to take into account fine changes in the shape of the region boundary. To this end, we transform the watershed patch to contain exclusively the part of the watershed on which we are to cast our vote, and discard all other region boundaries present in the patch.

This approach does not guarantee closed contours - the part of the region boundary present in the patch would often be \textbf{just an image edge}. So we cannot use as a scoring function a region metric but must instead use a boundary-based one. We must, therefore, transform the segmentation patch from the tree leaf to a boundary patch. It is trivial~\cite{Arbelaez11} to obtain an edge map, given a segmentation. our transformed tree leaf patch is a binary edge map. As a scoring function we use the our boundary-based evaluation metric - BPR.

% BPR on contours - watershed arc and region boundary
Our conclusion from this experiment is that the combination of only the edge with the BPR provides poor means of judging the evidence of boundary in the leaves of the structured forest. Watershed region boundary is a very brittle cue. Further, BPR is parametrised on the pixel distance for which a match between the two segmentation boundaries is to be made. There is not a generic way to correctly choose such a distance for all forest segmentation patches and watershed locations. Accurate localisation of boundaries seems to be crucial for this weighting strategy, and this is not the case with the leaf segmentations and the watershed.

\subsection{Line fitting}
Our best performing weighting strategy. We implemented three line fitting algorithms:
\begin{enumerate}
  \item parametric, based on the derivative direction of the end-points of the watershed edge we vote on,
  \item as above, but enforcing adherence to the centre of the image patch,
  \item linear least squares fitting to all the watershed edge pixels.
 % also possible - PCA-based fit
\end{enumerate}

Discuss performance \wrt different scoring functions. Notable that all 3 types of scoring functions - VPR, RI, BPR perform reasonably well with this watershed transformation. Normalisation of VPR. Asymmetry of normalisation and how it affects us.

\subsection{Quadratic fitting} % conic n=2; Polynomial
Conic sections - parabola, hyperbola and ellipse that fit the data. Too complex a model, thwarted by degenerate cases (best fitting parameters yield a 3-dimensional surface that doesn't intersect the $Z=0$ plane).


\begin{figure}[ht!]
\centering
 \subfigure[BPR]{%
  \includegraphics[trim=1.5cm 0cm 1.9cm 0cm, clip=true, width=0.48\textwidth]{images/plots/SE-quadratic_BPR.png}
 }
 \subfigure[VPR]{%
  \includegraphics[trim=1.5cm 0cm 1.9cm 0cm, clip=true, width=0.48\textwidth]{images/plots/SE-quadratic_VPR.png}
 }
\caption[Quadratic linear least squares fitting compared to the linear model - plots]{{\bf Quadratic} linear least squares fitting compared to the {\bf linear} model. The scoring function for the two experiments was the VPR normalised on the side of the watershed.}
\label{fig:SE-quadratic}
\end{figure}

\settocdepth{subsection}
\section[Oracle for Structured voting]{Oracle for SV - experiments using ground truth}
\label{sec:ch5-oracle}
To evaluate the correctness of our weighting strategies, we've implemented an oracle for our pipeline. The question we wanted to answer is ``how well could we perform segmentation in the presence of perfect information?'' Our Structured voting lends itself easily to such an experiment using the ground truth segmentation. 
When scoring a given pixel on the watershed regions boundary, we use a ground truth segmentation patch, rather than the most likely segmentation learnt by the structured forest. The second patch, as in the regular experiments, is taken from the same pixel location in the watershed locations image.

\section{Hardest negative mining}
Help determine where the voting fails the most. Conclusion: with so few votes per location, our approach would need much better leaves. We observed a lack of strong agreement in the leaves of the decision forest. The medoid segmentation patch, which is the only one casting a vote, is not necessarily representative of the set of segmentations that reached the leaf node of the tree.
% TODO figure of decision forest leaf, pref. different leaves

\section{Voting scope}
% TODO two types of voting scope experiments - reducing and expanding; two plots
  \begin{itemize}
    \item{\bf Degraded baseline SE-UCM:} Have the SF output a pixel, or a $2\times2$, $4\times4$, $8\times8$ patch.
    \item{\bf Reduced vote scope:} This series of experiments aims at proving that casting a vote on a larger area is desireable, by doing exactly the opposite. We cast just $T$ votes ($T$ - number of trees in the decision forest) on a single pixel of the watershed location. As expected, that severely diminishes performance (\wrt averaging on the region boundary) since the voting scope is decreased to 1 pixel. Excessive localisation therefore hinders performance.
  \end{itemize}
Discussion: It would be useful to see the effect of a ``mixed'' scope of voting - cast vote on the whole region boundary, but use subdivided region boundary - the watershed arc in order to do the patch transformation.

\begin{figure}[ht!]
\centering
 \subfigure[BPR]{%
  \includegraphics[trim=1.5cm 0cm 1.9cm 0cm, clip=true, width=0.48\textwidth]{images/plots/scope_of_voting_oracle_disagreement__line_centre_VPR_ws_BPR.png}
 }
 \subfigure[VPR]{%
  \includegraphics[trim=1.5cm 0cm 1.9cm 0cm, clip=true, width=0.48\textwidth]{images/plots/scope_of_voting_oracle_disagreement__line_centre_VPR_ws_VPR.png}
 }
\caption[Voting scope experiments.]{Voting scope experiments. The three segmentation methods and their oracles are variants of our best performing weighting strategy - central line fitting, cobined with VPR normalised on the side of the watershed. A description of VPR is given in Section~\ref*{sec:ch4-boundary-and-region-metrics-maths}~\ref{par:ch4-VPR-maths}).}
\label{fig:voting-scope-line-centre-VPR-ws}
\end{figure}

\section{Discussion}

Important conclusions of the experiments:
\begin{enumerate}
 \item both watershed transformations and scoring functions are important,
 % \item scoring functions matter,
 \item a suitable %smart
 watershed transformation could greatly aid a ``weaker'' scoring function (\eg the benefit of greedy merge when using RI scoring function),
 \item our best watershed transformation has reasonable performance with all the scoring functions tried,
 \item oracle confirms the ranking of our experiments,
 \item simpler models work better for transforming the watershed patch (\eg quadratic and polynomial fitting are the worst performing experiments regardless of the scoring function used),
 \item voting scope is very important; a decrease in the voting scope seriously damages results; successfully increasing the voting scope is not trivial.
\end{enumerate}
\clearpage

\chapter{Conclusions and open questions} % Problems}
\label{Chapter6}
In this project we revisited the design decisions made by a standard pipeline \cite{Arbelaez11} for the task of hierarchical segmentation. In particular, our work focused on the problem of obtaining weighted oversegmentation contours out of a binary image indicating %with 
their locations. To this end we proposed a new approach: {\tt Structured voting (SV)}, see \fref{fig:weighting-oversegm-contours}. {\tt Structured voting} utilises discriminatively trained local segmentation patches in a voting framework based on patches comparisons. We analysed the theoretical properties of multiple dimensions which determine out weighting strategy. We conducted extensive experiments and build an oracle to try and reason about the intrisic limitations of our method.

\begin{figure}[t]
\centering
\subfigure[Input image]{%
 \includegraphics[width=0.32\textwidth]{images/examples/tikis/tikis.jpg}
 %\label{fig:SE-UCM-tikis-bleeding-sub1}
}
\subfigure[UCM]{%
 \includegraphics[width=0.32\textwidth,frame]{images/conclusions/weighting-oversegm-contours.png}
 %\label{fig:SE-UCM-tikis-bleeding-sub2}
}
\caption[From edges to contours: a new approach]{We proposed a new approach for obtaining contours, given image edges.}
\label{fig:weighting-oversegm-contours}
\end{figure}


\section{Open questions} % be more positive - don't say 'Problems' :)


While our best-performing experiments still narrowly miss on improving over the baseline that we defined (

\section{Conclusions}

%\section{Potential improvements}


% \section{Future work}
% % future work
% Current video segmentation research is limited by lack of strong features. Galasso \etal show \cite{Galasso13} that a simple baseline using image segmentation in combination with optical flow outperforms a state SoA % all the tested 
% % video segmentation methods.

\clearpage

% Thanks to Gaurav :-) - good idea
% *************** List of figures ************
\listoffigures
\clearpage
% *************** Bibliography ***************
\bibliographystyle{abbrv}
{\small\bibliography{references.bib}}
\clearpage
% 
% % *************** Appendixes ***************
% \addtocontents{toc}{\vspace{2em}}
% \appendix
% %\appendixpage*
% \chapter{Detailed Random Forest Algorithm}
\label{AppendixA}
%\lhead{Appendix~\ref{AppendixA}. \emph{Detailed Random Forest Algorithm}}
Here is the full algorithm for building Random Forests. {\bf TrainRandomForest} (Algorithm~\ref{alg:TrainRandomForest}) function gets the whole 
bunch of training samples ${(x_i, y_i)}_{i = 1}^{N}$
and builds $M$ trees each of depth $D$. The main problem here is that the whole sample should be allocated in the memory at once which may require sometimes
considerable amount of memory. But on the other hand each tree can be build independently in a separate thread allowing for easy parallelization.

{\bf BuildTree} (Algorithm~\ref{alg:BuildTree}) is a function which builds each Random tree independently in a recursive manner. This algorithm can be
altered easily to better reflect the specific needs of the particular application of the Random Forest. This is a general algorithm.

{\bf GetDistribution} (Algorithm~\ref{alg:GetDistribution}) is a function which takes a particular sample and traverses it through a tree until terminating
in a leaf node, then it returns the distribution stored in the leaf node. Again, this is a general algorithm which can be altered to store anything apart
from distribution, \eg label patches or just a single label.

\begin{algorithm}
 \SetAlgoLined
 \KwData{a set of training points ${(x_i, y_i)}_{i = 1}^{N}$, maximum depth $D$, 
 number of trees $M$, number of random tests per node $K$, }
 \KwResult{final classifier $\mathcal{F} = \lbrace f_1, f_2,\dotsc, f_M \rbrace$}
 \For{$m \leftarrow 1$ \KwTo $M$}
 {
  1. compute per class weights $w$ \\
  2. Subsample the set of training points to $S = {(x_i, y_i)}_{i = 1}^{N'}$, so that $N' < N$ \\
  3. $f_m \leftarrow \text{{\bf BuildTree}}(S, D, K, w)$
 }
 \caption{{\bf TrainRandomForest} function}
 \label{alg:TrainRandomForest}
\end{algorithm}

\begin{algorithm}
 \SetAlgoLined
 \KwData{test sample $x$}
 \KwResult{distribution $p(c | x), c \in \mathcal{Y}$}
 \While{$node.is\_leaf \neq true$}
 {
  perform feature test $g(x)$\;
  \lIf{$g(x) < \theta$}
  {
    proceed to the left subtree\;
  }
  \lElse
  {
    proceed to the right subtree\;
  }
 }
 \caption{{\bf GetDistribution} function}
 \label{alg:GetDistribution}
\end{algorithm}

\begin{algorithm}
 \SetAlgoLined
 \KwData{a set of training points $S = {(x_i, y_i)}_{i = 1}^{N'}$, current depth level $level$, number of random tests per node $K$,
 per class weights $w$}
 \KwResult{random tree $f(x)$}
 $distribution \leftarrow$ ComputeDistribution$(S, w)$\;
 $node\_impurity \leftarrow E(S, w)$\;
 \If{node\_impurity $<$ threshold or $\abs{S} <$ minN or level = 0}
 {
  $node.distribution \leftarrow distribution$\;
  terminate\;
 }
 $level \leftarrow level - 1$\;
 $score' \leftarrow +\inf$\;
 \For{$k \leftarrow 1$ \KwTo $K$}
 {
  generate random parameters for the split\;
  choose random feature $d$\;
  get responses as $S' \leftarrow \{x_d | (x, y) \in S\}$\;
  find $a = \min S'$ and $b = \max S'$ values\;
  randomly uniformly sample $c$ from $[a, b]$\;
  separate $S$ into $L = \{(x, y) | x_d \leq c, (x, y) \in S\}$ and $R = \{(x, y) | x_d > c, (x, y) \in S\}$\;
  compute score as $score \leftarrow \frac{\abs{R}}{\abs{S}}E(R, w) + \frac{\abs{L}}{\abs{S}}E(L, w)$\;
  \If{$score < score'$}
  {
    $score' \leftarrow score$\;
    save the splitting parameters into node\;
    remember best split as $L' \leftarrow L$ and $R' \leftarrow R$\;
  }
 }
 \If{$\abs{L'} > 0$}
 {
  \bf{BuildTree}$(L', level, K, w)$\;
 }
 \If{$\abs{R'} > 0$}
 {
  \bf{BuildTree}$(R', level, K, w)$\;
 }
 \caption{{\bf BuildTree} function}
 \label{alg:BuildTree}
\end{algorithm}

% *************** Back matter ***************
%\backmatter
%\input{back.tex}

\end{document}