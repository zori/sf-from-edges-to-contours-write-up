\chapter{A New More Challenging Road Scenes Dataset}
\label{Chapter4}

In order to study the problem of adaptation we needed to find a dataset, which would exhibit considerable amount of appearance variation still
representing the same scenario. Image segmentation is a very old and very well studied field of Computer Vision, but we were not able to find
a dataset which would suit the above mentioned conditions. Conventional datasets are either just a collection of different images depicting
different objects which have very little in common and such images cannot by put into one scenario, or the images in a dataset do not 
exhibit sufficient variations of visual conditions. Eventually, we decided to collect our own dataset of road scenes' images, which would
feature the visual conditions we were looking for and represent one coherent scenario. We decided to take images of roads in different conditions,
because this can allow direct application of our research in real-life.

\section{Existing Datasets}
There exist a large number of various datasets compiled on purpose for testing different segmentation approaches representing
different conditions. The most popular datasets for testing general purpose segmentation algorithms are the MSRC~\cite{MSRC} and
VOC~\cite{VOC} datasets. In our work we made use of the MSRC dataset, as it was also used as the test set in all the original papers
of the methods we implemented, to check the quality of our implementations.

We think that the most suitable setting for investigating into adaptive techniques is road scenes datasets. First of all, such datasets 
represent the field of application of our research which benefits most of all from the ability to adapt to changing visual conditions, which can
help make, for example, driving safer. Besides, such datasets exhibit the inherent process of evolution of visual conditions through time
which is the ideal setting for any adaptive algorithm, as the adaptation cannot be done instantly.

\subsection{The Microsoft Research Cambridge (MSRC) dataset}
The MSRC~\cite{MSRC} dataset is a publicly available collection of photographs taken at different parts of the world and
capturing a large number of objects classes. It contains 591 images and 23 classes, but all researchers leave out horse
and mountain classes, as there are
too few samples of them for training, so it is difficult for any classifier to faithfully learn them. But even the resulting 21
classes represents a wide variety of objects of different nature: building, grass, tree, cow, sheep, sky, airplane, water,
face, car, bicycle, flower, sign, bird, book, chair, road, cat, dog, body, boat. The main disadvantage of this dataset
is the fact that the provided ground truth images show very rough labeling of the objects. Besides, not all pixels are labeled
and one has to account for background (``void'') class both while training and testing. For certain methods 
(like~\cite{Kontschieder2011}) this may require particular handling of such ``void'' pixels. When testing, the pixels, marked
as background in the ground truth images, are not considered when computing error rates.

We use this dataset to show that our implementations of the methods described in Chapter~\ref{Chapter3} work exactly (up to the used features and certain
parameters of the classifiers) as stated in the original articles. We use the standard for this dataset splitting into train and
test images from~\cite{Krahenbuhl2011} which can be found here
\url{http://graphics.stanford.edu/projects/densecrf/textonboost/split.zip}, with the only difference that we combine Train and
Validation set together.

\begin{figure}[t]
 \centering
 \begin{subfigure}[c]{0.16\textwidth}
  \centering
  \includegraphics[width=\textwidth]{images/MSRC/1_9_s}
 \end{subfigure}
 \begin{subfigure}[c]{0.16\textwidth}
  \centering
  \includegraphics[width=\textwidth]{images/MSRC/1_9_s_GT}
 \end{subfigure}
 \begin{subfigure}[c]{0.16\textwidth}
  \centering
  \includegraphics[width=\textwidth]{images/MSRC/8_23_s}
 \end{subfigure}
 \begin{subfigure}[c]{0.16\textwidth}
  \centering
  \includegraphics[width=\textwidth]{images/MSRC/8_23_s_GT}
 \end{subfigure}
 \begin{subfigure}[c]{0.16\textwidth}
  \centering
  \includegraphics[width=\textwidth]{images/MSRC/3_26_s}
 \end{subfigure}
 \begin{subfigure}[c]{0.16\textwidth}
  \centering
  \includegraphics[width=\textwidth]{images/MSRC/3_26_s_GT}
 \end{subfigure}
 \\
 \begin{subfigure}[c]{0.16\textwidth}
  \centering
  \includegraphics[width=\textwidth]{images/MSRC/17_22_s}
 \end{subfigure}
 \begin{subfigure}[c]{0.16\textwidth}
  \centering
  \includegraphics[width=\textwidth]{images/MSRC/17_22_s_GT}
 \end{subfigure}
 \begin{subfigure}[c]{0.16\textwidth}
  \centering
  \includegraphics[width=\textwidth]{images/MSRC/20_12_s}
 \end{subfigure}
 \begin{subfigure}[c]{0.16\textwidth}
  \centering
  \includegraphics[width=\textwidth]{images/MSRC/20_12_s_GT}
 \end{subfigure}
 \begin{subfigure}[c]{0.16\textwidth}
  \centering
  \includegraphics[width=\textwidth]{images/MSRC/20_2_s}
 \end{subfigure}
 \begin{subfigure}[c]{0.16\textwidth}
  \centering
  \includegraphics[width=\textwidth]{images/MSRC/20_2_s_GT}
 \end{subfigure}
 \\
 \begin{subfigure}[c]{0.16\textwidth}
  \centering
  \includegraphics[width=\textwidth]{images/MSRC/19_2_s}
 \end{subfigure}
 \begin{subfigure}[c]{0.16\textwidth}
  \centering
  \includegraphics[width=\textwidth]{images/MSRC/19_2_s_GT}
 \end{subfigure}
 \begin{subfigure}[c]{0.16\textwidth}
  \centering
  \includegraphics[width=\textwidth]{images/MSRC/18_22_s}
 \end{subfigure}
 \begin{subfigure}[c]{0.16\textwidth}
  \centering
  \includegraphics[width=\textwidth]{images/MSRC/18_22_s_GT}
 \end{subfigure}
 \begin{subfigure}[c]{0.16\textwidth}
  \centering
  \includegraphics[width=\textwidth]{images/MSRC/18_25_s}
 \end{subfigure}
 \begin{subfigure}[c]{0.16\textwidth}
  \centering
  \includegraphics[width=\textwidth]{images/MSRC/18_25_s_GT}
 \end{subfigure}
 \\
 \begin{subfigure}[c]{0.16\textwidth}
  \centering
  \includegraphics[width=\textwidth]{images/MSRC/14_9_s}
 \end{subfigure}
 \begin{subfigure}[c]{0.16\textwidth}
  \centering
  \includegraphics[width=\textwidth]{images/MSRC/14_9_s_GT}
 \end{subfigure}
 \begin{subfigure}[c]{0.16\textwidth}
  \centering
  \includegraphics[width=\textwidth]{images/MSRC/10_16_s}
 \end{subfigure}
 \begin{subfigure}[c]{0.16\textwidth}
  \centering
  \includegraphics[width=\textwidth]{images/MSRC/10_16_s_GT}
 \end{subfigure}
 \begin{subfigure}[c]{0.16\textwidth}
  \centering
  \includegraphics[width=\textwidth]{images/MSRC/7_24_s}
 \end{subfigure}
 \begin{subfigure}[c]{0.16\textwidth}
  \centering
  \includegraphics[width=\textwidth]{images/MSRC/7_24_s_GT}
 \end{subfigure}
 \caption{Some example images with the corresponding ground truth images from the MSRC~\cite{MSRC} dataset}\label{fig:MSRC_examples}
\end{figure}

\subsection{Road scenes dataset}
As our main interest lies in the field of image segmentation of road scenes we used the following dataset from~\cite{Wojek2008}
to evaluate implemented classification algorithms and features. This dataset was taken by the authors using an auto-mounted
camera and driving along some German highways and roads, capturing real-time road scenarios. The dataset is available at
\url{http://www.d2.mpi-inf.mpg.de/tudds}. This dataset contains 88 training labeled images and 88 testing labeled images.
It also has a number of possible object class to color mappings, but we considered only the mapping which contains 3 classes:
road, sky and all other category classes were merged into one class called background. We believe that this is a reasonable
assumption, because we are mainly interested in correct detection of the ``road'' (blue) and all other classes in this case 
may be
considered as just ``background'' (green), and we also add ``sky'' (red) class as it occupies a lot of space in the provided 
images and helps not to turn our task into a binary problem.

We decided to take this dataset as an initial point in our research, because it was captured in real-life situations and we are
particularly interested in real-life road scenarios. Besides, already this dataset exhibits some visually difficult situations
like changes in object appearances due to motion blur effect, deep shadows which appear and disappear suddenly, changes in
lightning conditions like over- or under-saturated regions.

There also exists a dataset from~\cite{Alvarez2010}, but it has very limited labeling, which consists of just
two classes (``road'' and ``background''), which we found to be not corresponding to our aims. And besides that dataset does not
show any really difficult visual conditions.

\begin{figure}[t]
  \centering
  \begin{subfigure}[c]{0.24\textwidth}
    \centering
    \includegraphics[width=\textwidth]{images/EasySet/00000}
  \end{subfigure}
  \begin{subfigure}[c]{0.24\textwidth}
    \centering
    \includegraphics[width=\textwidth]{images/EasySet/00000_GT}
  \end{subfigure}
  \begin{subfigure}[c]{0.24\textwidth}
    \centering
    \includegraphics[width=\textwidth]{images/EasySet/00046}
  \end{subfigure}
  \begin{subfigure}[c]{0.24\textwidth}
    \centering
    \includegraphics[width=\textwidth]{images/EasySet/00046_GT}
  \end{subfigure}
  \\
  \begin{subfigure}[c]{0.24\textwidth}
    \centering
    \includegraphics[width=\textwidth]{images/EasySet/00053}
  \end{subfigure}
  \begin{subfigure}[c]{0.24\textwidth}
    \centering
    \includegraphics[width=\textwidth]{images/EasySet/00053_GT}
  \end{subfigure}
  \begin{subfigure}[c]{0.24\textwidth}
    \centering
    \includegraphics[width=\textwidth]{images/EasySet/00063}
  \end{subfigure}
  \begin{subfigure}[c]{0.24\textwidth}
    \centering
    \includegraphics[width=\textwidth]{images/EasySet/00063_GT}
  \end{subfigure}
  \\
  \begin{subfigure}[c]{0.24\textwidth}
    \centering
    \includegraphics[width=\textwidth]{images/EasySet/00076}
  \end{subfigure}
  \begin{subfigure}[c]{0.24\textwidth}
    \centering
    \includegraphics[width=\textwidth]{images/EasySet/00076_GT}
  \end{subfigure}
  \begin{subfigure}[c]{0.24\textwidth}
    \centering
    \includegraphics[width=\textwidth]{images/EasySet/00085}
  \end{subfigure}
  \begin{subfigure}[c]{0.24\textwidth}
    \centering
    \includegraphics[width=\textwidth]{images/EasySet/00085_GT}
  \end{subfigure}
  \caption{Some example images with the corresponding ground truth images from the~\cite{Wojek2008} dataset}\label{fig:Wojek_examples}
\end{figure}

\section{The New Dataset}

As we mentioned before, MSRC~\cite{MSRC} dataset is just a collection of pictures of a variety of object classes and it does not
suit our aim at all. We use it only for evaluation of the implemented methods, as all of them were evaluated on the
original papers on this dataset. The dataset from~\cite{Wojek2008} represents real-world road scenarios and exhibits visual
conditions we are looking for, but still it was too little for us. As this dataset has only pictures of asphalted 
relatively clean highways going through forests or fields and we would like to look into even more challenging conditions.

For this purpose we searched over the Internet, and particularly considered Flickr\textsuperscript{\textregistered}
(\url{http://www.flickr.com/}), looking for images depicting roads mostly in conditions which we called ``autumn'' and
``winter'', as ordinary datasets try to avoid such situations. We used the search engine of Flickr\textsuperscript{\textregistered}
and used such tags as ``dirty roads'', ``autumn roads'', ``roads with mud'', ``winter roads''.
The result of our searching is 220 images about half of which
represent roads in autumn conditions and another half roads in winter conditions. Figures~\ref{fig:autumn_examples} 
and~\ref{fig:winter_examples} show a bunch of examples from each of the ``seasons''. 
We performed hand labeling of the gathered images using the GNU Image Manipulation Program and a tablet into three classes: 
road (blue), sky (red), and background (green).
Unlike in the MSRC~\cite{MSRC} dataset, we treat background as a meaningful class and both learn and
predict it.

We believe that the introduced dataset introduces new challenges into the segmentation community, the ones which previously
were omitted. Fore example, our autumn images may have leaves on the road (which are on their own of yellow or red color,
which is different from the green color seen in the training set), different road cover (like asphalt of different colors
or simply no any asphalt at all, just ground), sometimes there dirt or snow (or even no any road visible, but rather just
path in the snow). If autumn images have at least something in common with the train set (and in all our experiments we use
train set from~\cite{Wojek2008}), then winter images have totally different visual appearance, which must create problems
for most ordinary off-line methods.

For notational convenience, from here on we use the word \emph{old} when we refer to the training and testing datasets from~\cite{Wojek2008},
and \emph{new} when we mean the introduced more challenging dataset. And in all the following experiments (apart from the ones on the 
MSRC~\cite{MSRC} dataset) we always use the \emph{old} training set for training.



\begin{figure}[t]
 \centering
 \begin{subfigure}[c]{0.4\textwidth}
  \centering
  \includegraphics[width=\textwidth]{images/NewSet/00000}
 \end{subfigure}
 \begin{subfigure}[c]{0.4\textwidth}
  \centering
  \includegraphics[width=\textwidth]{images/NewSet/00000_GT}
 \end{subfigure}
 \\
 \begin{subfigure}[c]{0.4\textwidth}
  \centering
  \includegraphics[width=\textwidth]{images/NewSet/00002}
 \end{subfigure}
 \begin{subfigure}[c]{0.4\textwidth}
  \centering
  \includegraphics[width=\textwidth]{images/NewSet/00002_GT}
 \end{subfigure}
 \\
 \begin{subfigure}[c]{0.4\textwidth}
  \centering
  \includegraphics[width=\textwidth]{images/NewSet/00003}
 \end{subfigure}
 \begin{subfigure}[c]{0.4\textwidth}
  \centering
  \includegraphics[width=\textwidth]{images/NewSet/00003_GT}
 \end{subfigure}
 \\
 \begin{subfigure}[c]{0.4\textwidth}
  \centering
  \includegraphics[width=\textwidth]{images/NewSet/00014}
 \end{subfigure}
 \begin{subfigure}[c]{0.4\textwidth}
  \centering
  \includegraphics[width=\textwidth]{images/NewSet/00014_GT}
 \end{subfigure}
 \\
 \begin{subfigure}[c]{0.4\textwidth}
  \centering
  \includegraphics[width=\textwidth]{images/NewSet/00065}
 \end{subfigure}
 \begin{subfigure}[c]{0.4\textwidth}
  \centering
  \includegraphics[width=\textwidth]{images/NewSet/00065_GT}
 \end{subfigure}
 \caption{Some example images with the corresponding ground truth images from ``autumn'' part of the
 proposed dataset.}\label{fig:autumn_examples}
\end{figure}

\begin{figure}[t]
 \centering
 \begin{subfigure}[c]{0.4\textwidth}
  \centering
  \includegraphics[width=\textwidth]{images/NewSet/00128}
 \end{subfigure}
 \begin{subfigure}[c]{0.4\textwidth}
  \centering
  \includegraphics[width=\textwidth]{images/NewSet/00128_GT}
 \end{subfigure}
 \\
 \begin{subfigure}[c]{0.4\textwidth}
  \centering
  \includegraphics[width=\textwidth]{images/NewSet/00129}
 \end{subfigure}
 \begin{subfigure}[c]{0.4\textwidth}
  \centering
  \includegraphics[width=\textwidth]{images/NewSet/00129_GT}
 \end{subfigure}
 \\
 \begin{subfigure}[c]{0.4\textwidth}
  \centering
  \includegraphics[width=\textwidth]{images/NewSet/00169}
 \end{subfigure}
 \begin{subfigure}[c]{0.4\textwidth}
  \centering
  \includegraphics[width=\textwidth]{images/NewSet/00169_GT}
 \end{subfigure}
 \\
 \begin{subfigure}[c]{0.4\textwidth}
  \centering
  \includegraphics[width=\textwidth]{images/NewSet/00188}
 \end{subfigure}
 \begin{subfigure}[c]{0.4\textwidth}
  \centering
  \includegraphics[width=\textwidth]{images/NewSet/00188_GT}
 \end{subfigure}
 \\
 \begin{subfigure}[c]{0.4\textwidth}
  \centering
  \includegraphics[width=\textwidth]{images/NewSet/00210}
 \end{subfigure}
 \begin{subfigure}[c]{0.4\textwidth}
  \centering
  \includegraphics[width=\textwidth]{images/NewSet/00210_GT}
 \end{subfigure}
 \caption{Some example images with the corresponding ground truth images from ``winter'' part of the
 proposed dataset.}\label{fig:winter_examples}
\end{figure}