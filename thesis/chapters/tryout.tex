Test orig: \cite{LimZD13}.

We use the random decision forest framework implemented in the toolbox of Piotr Doll\'ar~\cite{Dollar2013toolbox}.

Check capitals: \cite{Arbelaez09}. % these are one after the other
Arxiv: \cite{Hallman2014}

Dollar's: \cite{Dollar2015PAMI,DollarICCV13edges}

Reference with question mark: \cite{Fowlkes04}.

%%
%
% ready to copy: with subfigures
\begin{figure}[ht!]
\centering
 \subfigure[Input foo]{%
  \includegraphics[width=0.5\textwidth]{images/foo.png}
 }
\caption{{\bf Foo} is important.}
\label{fig:foo}
\end{figure}

%%
%
\begin{figure}[ht!]
\centering
 \subfigure[Input image]{%
  \includegraphics[width=0.3\textwidth]{images/examples/hawaii/arbelaez2011-035.png}
 }
 \subfigure[Edge map]{%
  \includegraphics[width=0.3\textwidth]{images/examples/hawaii/edge_map_arbelaez2011-039.png}
  \label{fig:subfigure1}
 }
 \subfigure[Probability of boundary]{%
  \includegraphics[width=0.3\textwidth]{images/examples/hawaii/Pb_arbelaez2011-039.png}
  \label{fig:subfigure2}
 }
% \quad
% \qquad
% \caption[Caption that doesn't show?]{{\bf Boundary detection} (courtesy~\cite{Arbelaez11}).}
\caption{Referencing subfigure: \protect\subref{fig:subfigure2} some text \protect\subref{fig:subfigure1} some other text}
\label{fig:edge_detection-tryout}
\end{figure}

%%
%
% simple figure, no labels
\begin{figure}[ht!]
 \centering
 \subfigure[Input image]{%
 \includegraphics[width=0.3\textwidth]{images/examples/hawaii/arbelaez2011-035.png}
 }
 \subfigure[Edge map]{%
 \includegraphics[width=0.3\textwidth]{images/examples/hawaii/edge_map_arbelaez2011-039.png}
 }
\subfigure[Probability of boundary]{%
 \includegraphics[width=0.3\textwidth]{images/examples/hawaii/Pb_arbelaez2011-039.png}
 }
 \caption[Caption that doesn't show?]{ % PS: it does show, it is the short figure or table title when you print the list of figures / tables
  {\bf Boundary detection} (courtesy of~\cite{Arbelaez11}).}
\end{figure}