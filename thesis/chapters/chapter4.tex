\chapter{Leveraging structured forest for segmentation}
\label{Chapter4}
Given an edge detection result, various methods could be employed to obtain image segmentation.  One approach is to try and identify all possible boundaries of segmentation regions, based on the probability of boundary score produced by the edge detector. Those boundaries finely partition the image pixel graph. The regions closed by them constitute an oversegmentation of the image. Different tasks require segmentation at different level of detail. It is useful, therefore, to be able to obtain coarser segmentations as well. By reasoning about the salience of region boundaries, one could construct a hierarchy of segmentations. Going up the hierarchy corresponds to coarsening the segmentation by merging neighbouring regions which are separated by a weak boundary. To establish the order in which to merge the regions, we must be able to tell weak from strong boundaries. We want to associate a score with every possible region boundary, which score should reflect the strength of that intervening boundary.

To score the region boundaries we take a patch comparison approach. The edge detector which we employ, Structured edge (SE) by Dollar et al.~\cite{DollarICCV13edges}, associates a segmentation patch centred around every pixel location in the image - its most likely segmentation. The possible region boundaries locations are obtained by a watershed transformation on the output of this detector. For the same location we attain a patch from the watershed locations image. We then explore strategies to compare those two patches and score the similarity between them, a higher score meaning higher local evidence for boundary. We call those watershed weighting strategies. The choice of such an approach defines how we conduct our \textbf{Structured voting (SV)}.

We identify two important points in order to be able to score the two patches just described. Firstly, one being a local segmentation, and the other - a local oversegmentation, they are quite dissimilar. Therefore, one patch needs to be sensibly transformed, so that the two patches are comparable. Secondly, given that the two patches are comparable, what scoring function should be employed? We cast the problem of comparing two segmentation patches as a segmentation benchmark problem. Various boundary- and region-based metrics have been proposed for the task of image segmentation. We analyse the theoretical properties of some, and use a reasonable subset of them in our practical experiments. So a watershed weighting strategy has two aspects - making the patches sufficiently similar to compare, and choosing a scoring function to compare them.

In the terms of the framework of Arbelaez et al.\cite{Arbelaez11} - gPb-OWT-UCM, we propose to use SE instead of gPb as an edge detector. Further, we replace the Oriented watershed transform (OWT) by SV to obtain scored region boundary locations. Our image segmentation pipeline could then be titled SE-SV-UCM.

\section{Edge detection - gPb vs. structured edge}
\section{Weighting the watershed locations}
\subsection{Comparing a structured forest patch to a watershed patch}
\subsubsection{Patch transformations}
\subsection{Scoring functions for patch similarity}
\subsubsection{Cast as a segmentation benchmark problem}
\subsubsection{Boundary and region metrics}
\paragraph{BPR}

\[
P=\frac{\left|S\cap\left(\bigcup\limits _{i=1}^{M}G_{i}\right)\right|}{|S|}
\]
\[
R=\frac{{\sum\limits _{i=1}^{M}\left|S\cap G_{i}\right|}}{\sum\limits _{i=1}^{M}\left|G_{i}\right|}
\]
\[
F=\frac{2PR}{P+R}
\]


where $S$ and $\{G_{i}\}_{i=1}^{M}$ are boundary maps and $\cap$
does a bipartite graph assignment between them.


\paragraph{RI (Rand Index)}


\subparagraph{RI}

between two segmentations

\begin{align*}
RI(S,G) & =\frac{1}{T}\sum\limits _{j<k}\left[\mathbb{I}\left(S(j)=S(k)\wedge G(j)=G(k)\right)+\mathbb{I}\left(S(j)\neq S(k)\wedge G(j)\neq G(k)\right)\right]\\
 & =\frac{1}{T}\sum\limits _{(j,k)\in A}\left[c_{jk}\mathbb{I}\left(G(j)=G(k)\right)+(1-c_{jk})\mathbb{I}\left(G(j)\neq G(k)\right)\right]
\end{align*}


where $\mathbb{I}$ - the identity function,

$S(j)$ - the label of pixel $j$ in the segmentation $S$,

$c_{jk}=\mathbb{I}\left(S(j)=S(k)\right)$ - the event of the pair
of pixels $j$ and $k$ having the same label in segmentation $S$,

$A=\{(j,k)|j<k\}$ - the set of unique pairs of pixels,

$N=\left|S\right|=\left|G\right|$ - the number of pixels in the image
(and each segmentation), and 

$T=|A|=\binom{N}{2}$ - the number of possible unique pairs among
$N$ pixels.


\subparagraph{RSRI (Random Subsample RI)}

previously CPD (Crude Patch Distance), which was a misnomer

Monte Carlo sample of the possible unique pairs of pixels.

between two segmentations

\[
RSRI(S,G)=\frac{1}{T}\sum\limits _{(j,k)\in B}\left[c_{jk}\mathbb{I}\left(G(j)=G(k)\right)+(1-c_{jk})\mathbb{I}\left(G(j)\neq G(k)\right)\right]
\]


where $B\subsetneq A$ - a (random) subset of the pairs of pixels.
In our experiments $|B|=256$.


\subparagraph{PRI (Probabilistic RI)}

between a test segmentation and multiple ground truths

\[
PRI(S,\{G_{i}\}_{i=1}^{M})=\frac{1}{T}\sum\limits _{j<k}\left[c_{jk}p_{jk}+\left(1-c_{jk}\right)\left(1-p_{jk}\right)\right]
\]


where $p_{jk}$ - probability of $c_{jk}$; possible estimator of
$p_{jk}$ is the sample mean of the corresponding Bernoulli distribution.


\paragraph{SC (Segmentation Covering)}


\subparagraph{Overlap, a.k.a. Jaccard Index}

Overlap of two regions $s$ and $g$

\[
\mathcal{O}\left(s,g\right)=\frac{\left|s\cap g\right|}{\left|s\cup g\right|}
\]



\subparagraph{IoU (intersection over Union)}

How to normalise it in the case of two segmentations $S=\left\{ {s_{i}}\right\} _{i=1}^{p}$
and $G=\left\{ {g_{j}}\right\} _{j=1}^{q}$ ?
\[
IoU(S,G)=\frac{\sum\limits _{i=1}^{p}\sum\limits _{j=1}^{q}\mathcal{O}\left(s_{i},g_{j}\right)}{\Gamma_{G}}
\]
where $\Gamma_{S}=p$ and $\Gamma_{G}=q$ - number of segments in
each segmentation.

The above is not symmetric w.r.t. $S$ and $G$.

\[
IoU(S,G)=\frac{\sum\limits _{i=1}^{p}\sum\limits _{j=1}^{q}\mathcal{O}\left(s_{i},g_{j}\right)}{\Gamma_{S}\Gamma_{G}}
\]


The above would penalise two equal segmentations if they have larger
amount of segments.


\subparagraph{SC}

Asymmetric metric, therefore two possible uses:


%\subsubparagraph
\textbf{Covering of the Test Segmentation with the Ground Truths}

\[
C\left(\left\{ {G_{i}}\right\} _{i=1}^{M}\longrightarrow S\right)=\frac{1}{M}\sum\limits _{i=1}^{M}\frac{1}{N}\sum\limits _{s\in S}\left|s\right|\max_{g\in G_{i}}\frac{\left|s\cap g\right|}{\left|s\cup g\right|}
\]

where $N=\left|S\right|=\left|G_{i}\right|$ - number of pixels in
the image.

%\subsubparagraph
\textbf{Covering of the Ground Truths with the Test Segmentation}

\[
C\left(S\longrightarrow\left\{ {G_{i}}\right\} _{i=1}^{M}\right)=\frac{1}{M}\sum\limits _{i=1}^{M}\frac{1}{N}\sum\limits _{g\in G_{i}}\left|g\right|\max_{g\in G_{i}}\frac{\left|s\cap g\right|}{\left|s\cup g\right|}
\]


\paragraph{VPR (Volumetric Precision-Recall)}


\subparagraph{VPR unnormalised}

\[
\tilde{P}=\frac{1}{M}\sum\limits _{i=1}^{M}\frac{\sum\limits _{s\in\mathbb{S}}\max\limits _{g\in\mathbb{G}_{i}}\left|s\cap g\right|}{\left|\mathbb{S}\right|}=\frac{\sum\limits _{i=1}^{M}\sum\limits _{s\in\mathbb{S}}\max\limits _{g\in\mathbb{G}_{i}}\left|s\cap g\right|}{M\left|\mathbb{S}\right|}
\]


\[
\tilde{R}=\sum\limits _{i=1}^{M}\frac{\sum\limits _{g\in\mathbb{G}_{i}}\max\limits _{s\in\mathbb{S}}\left|s\cap g\right|}{\sum\limits _{j=1}^{M}\left|\mathbb{G}_{j}\right|}=\frac{\sum\limits _{i=1}^{M}\sum\limits _{g\in\mathbb{G}_{i}}\max\limits _{s\in\mathbb{S}}\left|s\cap g\right|}{\sum\limits _{i=1}^{M}\left|\mathbb{G}_{i}\right|}
\]


\[
\tilde{F}=\frac{2\tilde{P}\tilde{R}}{\tilde{P}+\tilde{R}}
\]

where $\mathbb{S}$ and $\{\mathbb{G}_{i}\}_{i=1}^{M}$ - segmentations,

$\cap$ computes volume overlap between segments, and 

$\left|\centerdot\right|$ counts the pixels in the voxel.

\subparagraph{VPR normalised (lower bound)}

\[
P=\frac{\sum\limits _{i=1}^{M}\sum\limits _{s\in\mathbb{S}}\max\limits _{g\in\mathbb{G}_{i}}\left|s\cap g\right|-\boxed{\sum\limits _{i=1}^{M}\max\limits _{g\in\mathbb{G}_{i}}\left|g\right|}}{M\left|\mathbb{S}\right|-\boxed{\sum\limits _{i=1}^{M}\max\limits _{g\in\mathbb{G}_{i}}\left|g\right|}}
\]


\[
R=\frac{\sum\limits _{i=1}^{M}\sum\limits _{g\in\mathbb{G}_{i}}\max\limits _{s\in\mathbb{S}}\left|s\cap g\right|-\boxed{\sum\limits _{i=1}^{M}\Gamma_{\mathbb{G}_{i}}}}{\sum\limits _{i=1}^{M}\left|\mathbb{G}_{i}\right|-\boxed{\sum\limits _{i=1}^{M}\Gamma_{\mathbb{G}_{i}}}}
\]


\[
F=\frac{2PR}{P+R}
\]


\subparagraph{VPR normalised (new - according to model capacity of test segmentation)}

\[
\hat{P}=\frac{\sum\limits _{i=1}^{M}\sum\limits _{s\in\mathbb{S}}\max\limits _{g\in\mathbb{G}_{i}}\left|s\cap g\right|-\boxed{M\Gamma_{\mathbb{S}}}}{M\left|\mathbb{S}\right|-\boxed{M\Gamma_{\mathbb{S}}}}
\]


\[
\hat{{R}}=\frac{\sum\limits _{i=1}^{M}\sum\limits _{g\in\mathbb{G}_{i}}\max\limits _{s\in\mathbb{S}}\left|s\cap g\right|-\boxed{M\max_{s\in\mathbb{S}}\left|s\right|}}{\sum\limits _{i=1}^{M}\left|\mathbb{G}_{i}\right|-\boxed{M\max_{s\in\mathbb{S}}\left|s\right|}}
\]


\[
\hat{{F}}=\frac{2\hat{P}\hat{R}}{\hat{P}+\hat{R}}
\]
