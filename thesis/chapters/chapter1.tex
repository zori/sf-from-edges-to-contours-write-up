\chapter{Introduction}
\label{Chapter1}
\section{Motivation}
\section{Terminology}
\subsection{Edge Detection - Edge, Edge Map, Probability of Boundary}
\subsection{Image Segmentation - Contour, Segmentation, Region Boundary, Hierarchical Segmentation}
\section{Related Work}
% \subsubsection*{Edge Detection}
\subsection{From Edges to Contours}
\subsubsection{Edge Detection}
\subsubsection{Image Segmentation}
\subsection{State-of-the-Art Review}
\section{Outline}
The rest of this work is structured as follows:

% \begin{itemize}
% \item Chapter~\ref{SpectRelax} starts the thesis with a brief introduction to balanced graph cuts and spectral relaxation techniques.
% Section~\ref{sec:ch2_clgrpart} shows that clustering can be seen as a graph partitioning problem. The minimum cut approach often yields useless results where clusters are highly unbalanced, hence
% the balanced graph cut criteria are described in Section~\ref{sec:ch2_balgrcut}.  To solve the NP-hard balanced graph cut problem
% the relaxation methods are applied. Section~\ref{sec:ch2_spectclus} presents the standard spectral clustering approach, which is known to be loose.
% The tight relaxation, called 1-spectral clustering, is described in Section~\ref{sec:ch2_1spectclus}. 
% Section~\ref{ch2:disc} concludes the chapter and discusses the relevance of proposed methods to video segmentation.
% \item Chapter~\ref{chapter3} gives an overview of the video segmentation framework and provides the analysis of low-level features.
% Section~\ref{sec:ch3_framework} introduces the proposed video segmentation model, which employs a two-step approach:
% a graph is constructed on pre-computed superpixels and then a spectral clustering technique is applied. 
% %In graph-based algorithms, in order to produce high-quality segmentation results powerful superpixel similarity measures must be defined.
% Section~\ref{sec:ch3_affinities} gives a description of the graph affinities used in this work.
% To evaluate video segmentation performance and analyze the features of the proposed model we chose the Berkeley motion segmentation dataset, which is presented in Section~\ref{sec:ch3_dataset}.  
% The examination of the quality of the low-level video features is reported in Section~\ref{sec:ch3_aff} and the results are discussed in Section~\ref{ch3:disc}.
% \item Chapter~\ref{Chapter4} provides an experimental comparison of spectral relaxations and analyzes the effects of different balanced graph cuts applied to video segmentation. 
% We start with a brief recap of the main theoretical aspects of 1-norm and 2-norm relaxations in Section~\ref{ch4:recap}.
% Section~\ref{ch4:bench} presents the evaluation benchmark for video segmentation.
% Section~\ref{sec:ch4_1sc_vs_sc} shows the comparison of the performance of spectral clustering and 1-spectral clustering with different balanced graph cut objectives in the task of video segmentation. 
% In order to explore further the balanced cut criteria and the quality of the solutions obtained from the relaxation techniques, we tried to find a better partition by a trivial greedy search optimizing different balanced graph cut functions and see if the ground truth corresponds
% with the minimum cut criterion. The results of the experiments are reported in Section~\ref{sec:ch4_GTexp}.
% Section~\ref{ch4:disc} gives the discussion of the obtained results.
% \item In Chapter~\ref{Chapter5} a methodology for discriminative learning of must-link constraints and their incorporation in the video segmentation framework are proposed.
% Section~\ref{sec:ch5_cosc} shows a way of integrating prior information in the form of must-link constraints into spectral clustering while preserving all the
% balanced graph cuts.
% Section~\ref{sec:llf} presents evaluation of the low-level features as must-link constraints and the connections in the graph based on the ground truth.
% In Section~\ref{sec:ch4_ML} we propose to learn must-links with Random Forest from the affinities.
% We show that even a naive learning approach on the restricted feature space improves video segmentation performance for both relaxations: spectral clustering and 1 spectral clustering. 
% The proposed model is compared to state-of-the-art methods.
% Section~\ref{ch5:disc} discusses the achieved results.
% \item Chapter~\ref{Chapter6} concludes the thesis summarizing all the results of our work and proposing directions for possible improvements.
% \end{itemize}

\chapter{From Edges to Contours - Current State-of-the-Art}
\label{Chapter2}
\section{gPb-owt-ucm Algorithm Pipeline}
\subsection{Quantised Oriented Probability of Boundary}
\subsection{Weighted Watershed}
\section{Flaws of Quantisation}

\chapter{Structured Random Decision Forests}
\label{Chapter3}
\section{Algorithm Outline}
\section{Training - Growing a Decision Forest}
\section{Inference}
\subsection{Sliding Window}
\subsection{Averaging of Overlapping Decisions}

\chapter{Leveraging Structured Forest for Segmentation}
\label{Chapter4}
\section{Edge Detection - gPb vs. Structured Edge}
\section{Weighting the Watershed Locations}
\subsection{Comparing a Structured Forest Patch to a Watershed Patch}
\subsubsection{Patch Transformations}
\subsection{Scoring Functions for Patch Similarity}
\subsubsection{Cast as a Benchmark Problem}
\subsubsection{Boundary and Region Metrics}

\chapter{Experimental Study of Watershed Weighting Strategies}
\label{Chapter5}
\section{Evaluation}
\subsection{Dataset}
\subsection{Metrics}
\section{Exploration of the Space of Weighting Strategies}
\section{Oracle - Experiments with Ground Truth}
\subsection{Oracle Description}
\subsection{Ranking of Oracles}
\subsubsection{Confirms Correct Weighting Strategies}
\subsubsection{Failure Cases}

\chapter{Conclusions and Open Problems}
\label{Chapter6}
\section{Conclusions}
\section{Open Problems}
