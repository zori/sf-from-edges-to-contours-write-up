\chapter{Conclusions and Future Work}
\label{Chapter6}

{\bf Summary:} \quad Image segmentation is a very old field of Computer Vision and has been studied for many years already. But despite of all the 
achieved progress, there are still some particular fields which require further researching. One such fields can be Image segmentation in changing
visual conditions. We introduced a new road scenes dataset in this work, which shows that in situations when the testing sets exhibits high
extent of visual appearance variation, which considerably differs from the conditions contained in the training set, standard approaches, which
rely on the assumption of the constant class distribution over sets, show considerable degrade in accuracy or simply fail. For example, the evaluation
of the current state-of-the-art method for general purpose scene segmentation by Kr\"ahenb\"uhl \etal~\cite{Krahenbuhl2011} on the proposed dataset and 
the old dataset from~\cite{Wojek2008} shows at least ten times worse accuracy on the proposed set. And the method based on Random Forest with Texton 
features from~\cite{Shotton2009} fails on the proposed dataset showing totally unacceptable results. But using advanced features for road detection allows 
even a simple Random Forest classifier to show good accuracy on the new testing set.

We showed in this work that on-line learning can be used as a powerful tool for performing adaptation to changing visual conditions giving 
the possibility for tracking the correspondingly changing underlying class distribution. Even the naive approach, when new samples are added
just based on their confidence scores, already shows improvement of more than 1\% in comparison with the off-line classifier, the one which didn't
use test set for refining its predicting abilities. We experimented with the Prior-constrained method from~\cite{Alvarez2012}, which considerably
decreases error for one class (``road''), but at the same time increases error for other (like ``background''). As a result the total error improve 
is not better than for the naive on-line approach.

To get better results for on-line learning based approaches and get better adaptation to changing conditions, we proposed Bayesian Model Update
under Structured Scene Prior for tracking the evolving class distribution. This method allowed to get a considerable improvement of 3\% for the 
total error in comparison with the off-line classifier, and more than 10\% improvement for the error rates of ``road'' and ``sky''.

{\bf Future work:} \quad Based on our experiments several directions for the future work can be outlined.
First direction is further improvement of the new dataset. Current dataset exhibits considerable amount of visual variations,
but its main disadvantage is that the images it contains are not captured by a car-mounted camera, and therefore show no temporal consistency between 
frames. Though such conditions suit the purposes of evaluation and comparisons of different methods (same as conventional MSRC or VOC datasets),
but it will be interesting to have a dataset captured in real-life conditions and which will have temporal coherency between adjacent frames. This
will also allow to ensure following application of adaptive algorithms into real-world and industry, because we think that it is exactly automobile
industry that will benefit most of all from application of such systems.

Additional research is necessary for improving Bayesian Model Update approach. First of all, batched learning should be substituted by a real on-line 
learning
with should considerably improve the overall running time of the system. This will also allow to increase the number of particles, which represent
the distribution over models being tracked  in our proposed Bayesian Model Update approach, which in turn can ensure that the tracking algorithm 
will not miss the target distribution due to lack of particles' representation. Another important concern is the problem of numerics, because computing
likelihoods at the weights update stage of the Bayesian Model Update algorithm requires multiplication of a lot of probabilities which may result in
underflow errors.

We used only a pure Random Forest as a particle in our proposed Bayesian Model Update approach, but this is a general framework which allows to 
incorporate any other classifier. It can be Boosting, Support Vector Machine, or anything else. If the number of particles is set to a sufficiently
large number, all this classifiers can be made work together in a unified framework, which will ensure accurate tracking of the evolving 
target distribution.
